% =========================
% CHƯƠNG 2 - GIỚI THIỆU
% =========================
\chapter{Giới thiệu}

\section{Bài toán nghiên cứu}

\subsection{Tác động của ùn tắc giao thông đô thị}
Ùn tắc giao thông là vấn đề phổ biến tại các trung tâm đô thị, gây tổn thất về thời gian, nhiên liệu và môi trường, đồng thời giảm chất lượng dịch vụ giao thông. Sự biến động theo thời gian và không đồng đều giữa các hướng khiến các giải pháp cố định khó đáp ứng được yêu cầu thực tế, đặc biệt trong giờ cao điểm hoặc khi xuất hiện sự kiện bất thường.

\subsection{Giới hạn của điều khiển thời gian cố định}
Các hệ thống điều khiển tín hiệu truyền thống dựa trên chu kỳ cố định (fixed‑time) thường được thiết kế theo mẫu lưu lượng trung bình. Nhược điểm chính là thiếu khả năng thích ứng với biến động thực tế: khi lưu lượng giảm, thời gian xanh có thể bị lãng phí; khi lưu lượng tăng, hàng chờ dễ hình thành và kéo dài. Các phương pháp semi/fully‑actuated cải thiện phần nào nhưng vẫn bị hạn chế bởi các tham số cố định và khả năng tối ưu trên mạng lưới lớn.

\subsection{Yêu cầu ưu tiên cho phương tiện khẩn cấp}
Một yêu cầu thiết yếu trong quản lý tín hiệu là đảm bảo phục vụ phương tiện ưu tiên (xe cứu thương, cứu hỏa, cảnh sát). Hệ thống cần phát hiện kịp thời và điều chỉnh pha sao cho phương tiện ưu tiên được thông suốt, đồng thời giảm thiểu tác động tiêu cực lên lưu lượng chung.

\subsection{Vấn đề rẽ trái và xung đột pha}
Xung đột giữa luồng rẽ trái và luồng đi thẳng đối diện là nguyên nhân phổ biến dẫn tới tắc nghẽn cục bộ và nguy cơ tai nạn. Thiết kế pha rẽ trái cố định thường không linh hoạt khi điều kiện thay đổi, do đó cần cơ chế phát hiện rẽ trái bị chặn và tạo pha bảo vệ động để giải phóng luồng.

\section{Mục tiêu nghiên cứu}

Mục tiêu chính của luận văn là phát triển một hệ thống điều khiển đèn giao thông lai, kết hợp bộ điều khiển pha thích nghi (APC) với học tăng cường (RL), nhằm:
\begin{itemize}
    \item Tăng khả năng thích ứng với biến động lưu lượng thời gian thực,
    \item Giảm thời gian chờ trung bình và độ dài hàng chờ,
    \item Hỗ trợ ưu tiên phương tiện khẩn cấp và xử lý rẽ trái bị chặn,
    \item Đảm bảo tính an toàn (chèn pha vàng, ngăn xung đột) và khả năng phục hồi khi mất kết nối với hệ thống lưu trữ.
\end{itemize}

\section{Phạm vi và giới hạn}

Công trình được thực nghiệm trong môi trường mô phỏng SUMO và tập trung chính vào tối ưu cho một nút giao hoặc cụm nút nhỏ. Những giới hạn chính bao gồm:
\begin{itemize}
    \item Sử dụng Q‑learning rời rạc cho tác tử RL (chưa triển khai Deep RL cho không gian trạng thái liên tục).
    \item Kết quả dựa trên dữ liệu mô phỏng; việc chuyển sang triển khai thực tế (sim‑to‑real) yêu cầu bổ sung domain randomization và tích hợp cảm biến thực.
    \item Khả năng điều phối đa nút ở mức sơ khai; cần mở rộng thêm để xử lý mạng lưới lớn với độ trễ giao tiếp thực.
\end{itemize}

\section{Đóng góp chính}

Luận văn đóng góp những điểm chính sau:
\begin{itemize}
    \item Mô hình lai APC–RL kết hợp quy tắc an toàn và cơ chế học để điều chỉnh thời lượng pha theo reward đa mục tiêu.
    \item Cơ chế quản lý yêu cầu ưu tiên (pending requests) cho phép phục vụ xe khẩn cấp, starvation và congestion theo thứ tự ưu tiên.
    \item Kiến trúc lưu trữ lai (local pickle cho Q‑table và Supabase cho log/state) đảm bảo khả năng tái tạo, phân tích và phục hồi.
    \item Bộ công cụ đánh giá trong SUMO cho phép so sánh với baseline fixed‑time và phân tích các kịch bản congestion, priority và blocked left.
\end{itemize}

\section{Cấu trúc chương tiếp theo}
Tài liệu được tổ chức như sau:
\begin{itemize}
    \item Chương 3: Tổng quan tài liệu liên quan và khoảng trống nghiên cứu.
    \item Chương 4–7: Thiết kế hệ thống, APC, thành phần RL và chiến lược điều khiển lai.
    \item Chương 8–10: Quản lý phương tiện ưu tiên, quản lý dữ liệu và chi tiết triển khai.
    \item Chương 11–13: Thiết lập thí nghiệm, đánh giá hiệu năng và thảo luận kết quả.
    \item Chương 14–16: Trực quan hóa, hướng phát triển tương lai và kết luận.
\end{itemize}