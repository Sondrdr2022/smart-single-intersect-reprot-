
\documentclass[12pt,a4paper,oneside]{report}

% ===== Vietnamese + Encoding =====
\usepackage[T5]{fontenc}
\usepackage[utf8]{inputenc}
\usepackage[vietnamese]{babel}

% ===== Page layout & typography =====
\usepackage[a4paper,margin=2.5cm]{geometry}
\usepackage{setspace}
\onehalfspacing
\usepackage{microtype}

% ===== Graphics, tables, math =====
\usepackage{graphicx}
\usepackage{float}
\usepackage{subcaption}
\usepackage{booktabs}
\usepackage{siunitx}
\sisetup{detect-all,round-mode=places,round-precision=2}
\usepackage{amsmath,amssymb,mathtools}

% ===== Lists & formatting =====
\usepackage{enumitem}
\setlist{noitemsep,leftmargin=2em}

% ===== TOC & Lists of figures/tables =====
\usepackage{tocloft}
\setlength{\cftbeforechapskip}{0.5em}
\renewcommand{\cftchapleader}{\cftdotfill{\cftdotsep}}

% ===== Hyperlinks & clever references =====
\usepackage{xcolor}
\definecolor{Link}{RGB}{0,0,128}
\definecolor{Cite}{RGB}{0,102,0}
\definecolor{URL}{RGB}{128,0,0}
\usepackage[
    unicode,
    pdfencoding=auto,
    colorlinks=true,
    linkcolor=Link,
    citecolor=Cite,
    urlcolor=URL
]{hyperref}
\usepackage{cleveref}

% ===== Header/Footer =====
\usepackage{fancyhdr}
\pagestyle{fancy}
\fancyhf{}
\lhead{\leftmark}
\rhead{\thepage}

% ===== Code listings (Python) =====
\usepackage{listings}
\lstdefinestyle{py}{
  language=Python,
  basicstyle=\ttfamily\footnotesize,
  numbers=left,
  numberstyle=\tiny,
  stepnumber=1,
  frame=single,
  breaklines=true
}

% ===== Glossaries / Acronyms (optional) =====
\usepackage[acronym]{glossaries}
\makeglossaries
\newacronym{apc}{APC}{Adaptive Phase Control}
\newacronym{rl}{RL}{Reinforcement Learning}
\newacronym{sumo}{SUMO}{Simulation of Urban MObility}
\newacronym{traci}{TraCI}{Traffic Control Interface}

% ===== Bibliography (biblatex with IEEE style) =====
\usepackage[backend=biber,style=ieee,sorting=none]{biblatex}
\addbibresource{reference.bib} % Đặt đúng tên file .bib

% ===== Title info =====
\title{\textbf{Hệ Thống Điều Khiển Đèn Giao Thông Thông Minh\\Dựa Trên Phương Pháp Lai APC và Học Tăng Cường}}
\author{Tên Sinh viên}
\date{\today}

\begin{document}
% ===== Title page =====
\maketitle

% ===== Front matter =====
\pagenumbering{roman}
\tableofcontents
\listoffigures
\listoftables

% ===== Danh mục từ viết tắt và thuật ngữ chuyên ngành =====
\addcontentsline{toc}{chapter}{Danh mục từ viết tắt và thuật ngữ chuyên ngành}

% In danh mục từ viết tắt (acronyms)
\printglossary[type=\acronymtype,title={Từ viết tắt phổ biến}]

% Danh sách thuật ngữ chuyên ngành
\section*{Thuật ngữ chuyên ngành}
\begin{description}[leftmargin=2.5em,style=nextline]
    \item[\textbf{Spillback}] Hiện tượng ùn tắc dội ngược: Khi hàng chờ tại một nút giao thông kéo dài vượt qua điểm đầu nút giao, làm cản trở hoặc chặn luôn dòng xe ở nút giao phía trước. Spillback gây ra hiệu ứng dây chuyền, lan rộng tắc nghẽn trong mạng lưới đô thị.
    \item[\textbf{Gridlock}] Kẹt lưới giao thông: Toàn bộ các nút giao đều bị kẹt, xe không thể di chuyển qua bất kỳ hướng nào do xung đột dòng xe cắt nhau.
    \item[\textbf{Starvation}] Đói phục vụ: Một làn hoặc hướng giao thông bị bỏ qua quá lâu, dẫn đến hàng chờ kéo dài không được phục vụ, thường xảy ra khi chính sách ưu tiên quá mức cho các hướng khác.
    \item[\textbf{Protected left}] Pha rẽ trái bảo vệ: Pha đèn tín hiệu được thiết kế riêng để cho phép rẽ trái an toàn, không bị xung đột với dòng đi thẳng đối diện.
    \item[\textbf{Adaptive Phase Control (APC)}] Điều khiển pha thích nghi: Phương pháp điều chỉnh thời lượng các pha đèn dựa trên trạng thái giao thông thực tế (hàng chờ, số xe dừng, v.v.).
    \item[\textbf{Reinforcement Learning (RL)}] Học tăng cường: Kỹ thuật học máy cho phép hệ thống điều khiển tín hiệu học từ phần thưởng/hình phạt dựa trên kết quả thực tế của hành động.
    \item[\textbf{Green wave}] Làn sóng xanh: Chuỗi các đèn tín hiệu được điều phối để xe chạy liên tục qua nhiều nút giao mà không phải dừng lại.
    \item[\textbf{Phase duration}] Thời lượng pha: Khoảng thời gian một pha đèn (xanh/vàng/đỏ) được duy trì trước khi chuyển sang pha tiếp theo.
    \item[\textbf{Supabase}] Nền tảng cơ sở dữ liệu đám mây, dùng lưu trữ trạng thái, nhật ký sự kiện và bảng Q-learning cho hệ thống điều khiển.
    \item[\textbf{SUMO}] Simulation of Urban Mobility: Phần mềm mô phỏng giao thông vi mô, dùng kiểm thử các thuật toán điều khiển tín hiệu.
    \item[\textbf{TraCI}] Traffic Control Interface: Giao thức kết nối giữa Python và SUMO, cho phép điều khiển và lấy dữ liệu mô phỏng theo thời gian thực.
    \item[\textbf{Q-table}] Bảng giá trị Q trong Q-learning: Lưu trữ giá trị kỳ vọng cho mỗi trạng thái và hành động, dùng để ra quyết định tối ưu cho agent học tăng cường.
    \item[\textbf{Pending requests}] Hàng đợi yêu cầu: Danh sách các yêu cầu chuyển pha đèn đang chờ xử lý, thường được xếp theo mức độ ưu tiên và thời gian.
\end{description}

% ===== Abstract =====
\chapter*{Tóm tắt}
\addcontentsline{toc}{chapter}{Tóm tắt}
Đèn giao thông đóng vai trò then chốt trong việc điều tiết lưu thông đô thị, song các hệ thống truyền thống thường thiếu khả năng thích ứng với biến động giao thông thực tế. Trong những năm gần đây, các phương pháp điều khiển thích nghi đã được nghiên cứu, nhưng vẫn còn tồn tại những hạn chế như hiện tượng nhấp nháy pha, xử lý tình huống khẩn cấp chưa hiệu quả và khó mở rộng cho nhiều nút giao.

Để khắc phục những vấn đề này, nghiên cứu đề xuất một bộ điều khiển nút giao thông thông minh được triển khai trên môi trường mô phỏng \gls{sumo}, kết nối qua \gls{traci}, kết hợp kỹ thuật học tăng cường và cơ chế lưu trữ trạng thái trên nền tảng đám mây Supabase. Bộ điều khiển có khả năng tự động điều chỉnh thời lượng đèn xanh dựa trên số lượng xe dừng, chiều dài hàng chờ và lưu lượng tức thời, đồng thời bảo đảm an toàn thông qua việc chèn pha vàng và duy trì tối thiểu thời gian đèn xanh.

Ngoài ra, hệ thống còn tích hợp cơ chế ưu tiên cho phương tiện khẩn cấp, quản lý ùn tắc và hỗ trợ rẽ trái an toàn thông qua hàng đợi yêu cầu có cấu trúc. Kết quả mô phỏng thực nghiệm cho thấy mô hình đề xuất giúp giảm đáng kể thời gian chờ trung bình và cải thiện lưu lượng so với phương pháp điều khiển tín hiệu cố định. Điều này chứng minh tính hiệu quả, linh hoạt và tiềm năng ứng dụng của hệ thống trong quản lý giao thông đô thị thông minh.

\clearpage
\pagenumbering{arabic}

% =========================
% CHƯƠNG 1
% =========================
% =========================
% CHƯƠNG 1
% =========================
\chapter{Tổng quan dự án}

\section{Giới thiệu dự án}
Trong bối cảnh đô thị hóa nhanh chóng, các thành phố lớn ngày càng đối mặt với tình trạng ùn tắc giao thông nghiêm trọng. Sự gia tăng đột biến của phương tiện cá nhân, kết hợp với tốc độ mở rộng hạ tầng giao thông chưa theo kịp, đã tạo ra những áp lực lớn đối với hệ thống điều tiết lưu lượng. Ùn tắc không chỉ gây ra sự lãng phí thời gian mà còn kéo theo nhiều hệ quả tiêu cực khác như tiêu hao nhiên liệu, phát thải khí thải nhà kính, gia tăng ô nhiễm môi trường và ảnh hưởng xấu đến sức khỏe cộng đồng. Bên cạnh đó, các tình huống khẩn cấp như xe cứu thương, cứu hỏa hoặc cảnh sát gặp khó khăn khi di chuyển qua các nút giao đông đúc càng làm bộc lộ hạn chế của hệ thống điều khiển tín hiệu hiện tại.

Các hệ thống đèn tín hiệu giao thông truyền thống thường được lập trình dựa trên chu kỳ cố định, được xác định sẵn từ dữ liệu trung bình lịch sử. Tuy nhiên, phương pháp này thiếu tính linh hoạt, không thể thích ứng kịp thời với biến động lưu lượng thực tế vốn thay đổi liên tục theo thời gian, vị trí và điều kiện giao thông. Kết quả là ở những hướng có mật độ phương tiện thấp, thời gian đèn xanh bị lãng phí, trong khi ở những hướng có lưu lượng cao lại hình thành hàng chờ dài và ùn tắc kéo dài. Sự cứng nhắc này khiến hiệu quả vận hành của mạng lưới giao thông giảm đáng kể và đặt ra yêu cầu cấp bách về một giải pháp điều khiển tín hiệu thông minh, thích ứng theo thời gian thực.

Để giải quyết vấn đề trên, dự án này phát triển một \textbf{hệ thống điều khiển đèn giao thông thông minh}, kết hợp giữa \textbf{điều khiển pha thích nghi (Adaptive Phase Control – APC)} và \textbf{học tăng cường (Reinforcement Learning – RL)}. APC cung cấp nền tảng điều khiển theo luật, cho phép điều chỉnh độ dài pha đèn dựa trên các ngưỡng hàng chờ hoặc số lượng xe dừng. Trong khi đó, RL mang lại khả năng học hỏi từ trải nghiệm, tối ưu dần chiến lược điều khiển thông qua cơ chế phần thưởng – hình phạt, giúp hệ thống thích nghi linh hoạt hơn với các tình huống giao thông phức tạp.

Hệ thống được kiểm thử trong môi trường mô phỏng \textbf{SUMO}, kết nối thông qua \textbf{TraCI}, đồng thời tích hợp \textbf{Supabase} để lưu trữ và đồng bộ dữ liệu trên nền tảng đám mây. Cách tiếp cận này không chỉ cho phép đánh giá hiệu năng của bộ điều khiển trong các kịch bản đa dạng mà còn mở ra khả năng mở rộng sang quy mô thực tế, hỗ trợ phân tích dữ liệu dài hạn và triển khai trong các hệ thống giao thông đô thị thông minh trong tương lai.

\section{Thành tựu chính}
\begin{itemize}
    \item Xây dựng bộ điều khiển lai \textbf{APC–RL} điều chỉnh chu kỳ đèn theo trạng thái giao thông.
    \item Cơ chế \textbf{ưu tiên phương tiện khẩn cấp}.
    \item Thuật toán \textbf{phát hiện và xử lý tắc nghẽn} (giảm spillback).
    \item \textbf{Rẽ trái bảo vệ} với pha bảo vệ động.
    \item Kết nối Supabase để \textbf{sao lưu trạng thái} và phân tích.
\end{itemize}

\section{Tóm tắt cải thiện hiệu năng}
Theo nghiên cứu của Eom và Kim \cite{Eom2020}, các hệ thống điều khiển tín hiệu thích nghi có thể cải thiện đáng kể hiệu suất giao thông. Kết quả thực nghiệm của dự án này cho thấy:

\begin{itemize}
    \item \textbf{Điều chỉnh pha:} trung bình +13.9s (kéo dài 87.1\%, rút ngắn 10.4\%), tối đa +88s / $-70$s.
    \item \textbf{Thời gian chờ:} giảm từ \textbf{2061.5s} (mặc định) xuống \textbf{1412.7s} (APC–RL) $\Rightarrow$ \textbf{-31.5\%}.
    \item \textbf{Hàng chờ:} 49.1 xe (APC–RL) so với 51.6 xe (mặc định).
    \item \textbf{Tốc độ trung bình:} 1.2 m/s (APC–RL) vs 0.8 m/s (mặc định).
\end{itemize}

\section{Tổng quan công nghệ sử dụng} 
Hệ thống điều khiển được xây dựng dựa trên sự tích hợp của nhiều công nghệ phần mềm và công cụ mô phỏng, đảm bảo khả năng xử lý dữ liệu thời gian thực, học tăng cường và trực quan hóa kết quả. Cụ thể: 
\begin{itemize} 
  \item \textbf{SUMO (Simulation of Urban MObility)}: công cụ mô phỏng giao thông mã nguồn mở, dùng để xây dựng mạng lưới đường phố, luồng xe, và kiểm thử các chiến lược điều khiển tín hiệu. 
   \item \textbf{TraCI (Traffic Control Interface)}: giao thức cầu nối giữa Python và SUMO, cho phép điều khiển đèn tín hiệu theo từng bước thời gian, lấy dữ liệu trạng thái như độ dài hàng chờ, thời gian chờ, tốc độ xe.
    \item \textbf{Python}: ngôn ngữ lập trình chính của hệ thống. Các thành phần quan trọng bao gồm: 
    \begin{itemize} 
      \item Bộ điều khiển pha thích nghi (Adaptive Phase Controller - APC) với logic tính toán thời lượng pha. 
      \item Agent Q-learning tăng cường (Enhanced Q-Learning Agent) cho việc ra quyết định tối ưu dựa trên trạng thái giao thông.
      \item Bộ quản lý vòng lặp chính để phối hợp giữa hai thành phần trên (hybrid APC–RL).
    \end{itemize} 
    \item \textbf{Supabase}: nền tảng cơ sở dữ liệu đám mây, được dùng để lưu trữ bảng Q-learning (Q-table), nhật ký các sự kiện điều khiển, dữ liệu trạng thái giao thông, đồng thời hỗ trợ đồng bộ dữ liệu trong thời gian thực.
    \item \textbf{Các thư viện Python hỗ trợ}:
    \begin{itemize} 
      \item \texttt{NumPy} và \texttt{Pandas} cho xử lý ma trận trạng thái, dữ liệu hàng chờ và tính toán thống kê. 
       \item \texttt{Matplotlib} để trực quan hóa kết quả mô phỏng: hàng chờ, thời gian chờ, tốc độ trung bình và phân tích điều chỉnh pha. 
       \item \texttt{Supabase-py} để kết nối và quản lý dữ liệu đám mây.
       \item \texttt{Pickle} dùng lưu trữ/tải lại bảng Q-learning dưới dạng file nhị phân. \end{itemize} 
    \end{itemize}


% =========================
% CHƯƠNG 2
% =========================
% =========================
% CHƯƠNG 2
% =========================
\chapter{Giới thiệu}

\section{Bài toán nghiên cứu}
\subsection{Thách thức tắc nghẽn giao thông đô thị}
Trong bối cảnh đô thị hoá nhanh chóng, hệ thống giao thông thành phố
 ngày càng chịu áp lực lớn bởi lưu lượng phương tiện gia tăng. Ùn tắc giao
 thông dẫn đến nhiều hệ quả tiêu cực như lãng phí thời gian, tiêu hao nhiên
 liệu, tăng phát thải khí nhà kính, và gây căng thẳng cho người tham gia
 giao thông. Việc quản lý luồng phương tiện tại các nút giao thông vì thế trở
 thành một thách thức trọng yếu trong quy hoạch đô thị thông minh \cite{Eom2020}.

\subsection{Hạn chế của hệ thống đèn tín hiệu cố định}
 Các hệ thống đèn tín hiệu truyền thống thường dựa trên chu kỳ cố định,
 được thiết kế sẵn dựa trên dữ liệu trung bình lịch sử. Tuy nhiên, phương
 pháp này không thể thích ứng với sự biến động liên tục của giao thông thực
 tế. Điều này dễ dẫn đến các tình trạng bất cập: pha đèn xanh bị lãng phí khi
 lưu lượng thấp, trong khi ở hướng lưu lượng cao lại hình thành hàng chờ dài.
 Ngoài ra, cơ chế cố định không phản ứng kịp với các tình huống bất thường
 như tai nạn hoặc tăng đột biến lưu lượng \cite{Eom2020}.

\subsection{Yêu cầu ưu tiên cho xe khẩn cấp}
Một vấn đề quan trọng khác trong quản lý tín hiệu giao thông là đảm bảo ưu
 tiên cho các phương tiện khẩn cấp như xe cứu thương, cứu hỏa và cảnh sát.
 Nếu không có cơ chế điều chỉnh kịp thời, xe khẩn cấp sẽ bị cản trở bởi tín
 hiệu đèn đỏ, làm chậm trễ việc tiếp cận hiện trường. Do đó, hệ thống điều
khiển thông minh cần có khả năng phát hiện và điều chỉnh pha tín hiệu nhằm
 tạo "làn sóng xanh" cho xe khẩn cấp di chuyển an toàn và nhanh chóng.

\subsection{Vấn đề điều khiển rẽ trái an toàn}
 Một trong những nguyên nhân thường xuyên gây ùn tắc và tai nạn tại nút
 giao là xung đột giữa dòng rẽ trái và dòng đi thẳng đối diện. Trong mô hình
 tín hiệu cố định, pha rẽ trái thường không được xử lý linh hoạt, dẫn đến tình
 trạng xe rẽ trái bị kẹt, gây cản trở dòng chính và hình thành ùn tắc cục bộ.
 Vì vậy, cần có một cơ chế phát hiện rẽ trái bị chặn và tự động chèn thêm
 pha bảo vệ để đảm bảo an toàn cũng như tối ưu hóa lưu lượng.

\section{Mục tiêu và phạm vi}
Mục tiêu là phát triển bộ điều khiển lai \gls{apc}–\gls{rl} tự điều chỉnh pha theo trạng thái, ưu tiên khẩn cấp, quản lý tắc nghẽn và rẽ trái bảo vệ; đồng bộ dữ liệu thực nghiệm lên Supabase.

\section{Phạm vi và giới hạn}
Hệ thống được kiểm chứng trong \textbf{môi trường mô phỏng} và \textbf{tối ưu cho một nút giao duy nhất}. Chưa có cơ chế điều phối đa nút giao. Agent RL hiện dùng \textbf{Q-learning cơ bản} với không gian trạng thái rời rạc, còn hạn chế về khả năng khái quát và dự đoán. Các hướng mở rộng gồm \textit{Deep RL}, dự báo lưu lượng ngắn hạn (RNN/LSTM), và phối hợp đa Agent.

\section{Cấu trúc tài liệu}

Luận văn được cấu trúc thành 16 chương chính và các phụ lục, được tổ chức theo logic từ tổng quan đến chi tiết, từ lý thuyết đến thực nghiệm. Cấu trúc này nhằm trình bày một cách hệ thống và toàn diện về hệ thống điều khiển đèn giao thông thông minh được phát triển.

\subsection*{Phần I: Giới thiệu và cơ sở lý thuyết (Chương 1-3)}

\begin{itemize}[leftmargin=1.5em]
    \item \textbf{Chương 1: Tổng quan dự án} -- Trình bày bối cảnh nghiên cứu, vấn đề cần giải quyết, các thành tựu chính đạt được và tóm tắt công nghệ được sử dụng trong dự án.
    
    \item \textbf{Chương 2: Giới thiệu} -- Phân tích chi tiết bài toán nghiên cứu, các thách thức của hệ thống giao thông đô thị, hạn chế của phương pháp điều khiển truyền thống và yêu cầu xử lý tình huống khẩn cấp.
    
    \item \textbf{Chương 3: Tổng quan tài liệu} -- Khảo sát các nghiên cứu liên quan từ phương pháp điều khiển cố định, hệ thống thích nghi đến ứng dụng học máy và phương pháp lai trong điều khiển tín hiệu. Chương này xác định khoảng trống nghiên cứu và định vị đóng góp của luận văn.
\end{itemize}

\subsection*{Phần II: Thiết kế và phát triển hệ thống (Chương 4-10)}

\begin{itemize}[leftmargin=1.5em]
    \item \textbf{Chương 4: Kiến trúc hệ thống} -- Mô tả thiết kế tổng thể, sơ đồ các thành phần, luồng dữ liệu và điểm tích hợp giữa các module. Chương này cũng trình bày chi tiết về công nghệ và công cụ được sử dụng bao gồm SUMO, TraCI, Python và Supabase.
    
    \item \textbf{Chương 5: Điều khiển pha thích nghi (APC)} -- Giải thích nguyên lý hoạt động của bộ điều khiển APC, bao gồm thuật toán điều chỉnh pha động, cơ chế phát hiện và xử lý tắc nghẽn, cũng như quản lý pha rẽ trái bảo vệ để đảm bảo an toàn giao thông.
    
    \item \textbf{Chương 6: Thành phần học tăng cường} -- Trình bày kiến trúc Agent Q-learning cải tiến, thiết kế không gian trạng thái, hành động và hàm thưởng. Chương này phân tích cơ chế học và cập nhật chính sách dựa trên phản hồi từ môi trường.
    
    \item \textbf{Chương 7: Chiến lược điều khiển lai} -- Mô tả cách tích hợp APC và RL thành một hệ thống thống nhất, cơ chế phối hợp giữa điều khiển dựa trên luật và học máy, cùng với chiến lược giải quyết xung đột khi hai phương pháp đưa ra quyết định khác nhau.
    
    \item \textbf{Chương 8: Quản lý xe ưu tiên} -- Chi tiết hóa cơ chế phát hiện và xử lý phương tiện khẩn cấp, thuật toán tạo "làn sóng xanh" và phân tích tác động của việc ưu tiên đến hiệu năng tổng thể của hệ thống.
    
    \item \textbf{Chương 9: Quản lý dữ liệu} -- Trình bày kiến trúc lưu trữ trên đám mây với Supabase, cơ chế đồng bộ Q-table, hệ thống ghi log sự kiện và các API để truy vấn và phân tích dữ liệu thời gian thực.
    
    \item \textbf{Chương 10: Chi tiết triển khai} -- Cung cấp hướng dẫn cài đặt, cấu hình tham số hệ thống, cấu trúc mã nguồn và các best practices trong quá trình phát triển và triển khai.
\end{itemize}

\subsection*{Phần III: Thực nghiệm và đánh giá (Chương 11-13)}

\begin{itemize}[leftmargin=1.5em]
    \item \textbf{Chương 11: Mô hình thí nghiệm} -- Mô tả thiết lập môi trường mô phỏng SUMO, cấu hình mạng lưới đường và nút giao, định nghĩa các kịch bản kiểm thử với mức độ phức tạp khác nhau.
    
    \item \textbf{Chương 12: Đánh giá hiệu năng và so sánh} -- Trình bày các metrics đánh giá (thời gian chờ, độ dài hàng chờ, thông lượng), so sánh định lượng với hệ thống cố định và phân tích đường cong học tập của Agent RL.
    
    \item \textbf{Chương 13: Kết quả và thảo luận} -- Tổng hợp các phát hiện chính, phân tích hành vi hệ thống trong điều kiện khác nhau, thảo luận về ưu điểm, hạn chế và các yếu tố ảnh hưởng đến hiệu năng.
\end{itemize}

\subsection*{Phần IV: Công cụ và triển vọng (Chương 14-16)}

\begin{itemize}[leftmargin=1.5em]
    \item \textbf{Chương 14: Trực quan hóa và giám sát} -- Giới thiệu dashboard theo dõi thời gian thực, các công cụ phân tích và visualization để đánh giá hiệu năng hệ thống một cách trực quan.
    
    \item \textbf{Chương 15: Hướng phát triển tương lai} -- Đề xuất các cải tiến tiềm năng: mở rộng sang điều phối đa nút giao, tích hợp Deep Reinforcement Learning, giao tiếp V2I, dự báo lưu lượng với mạng LSTM và triển khai trên nền tảng đám mây quy mô lớn.
    
    \item \textbf{Chương 16: Kết luận} -- Tóm tắt toàn diện kết quả nghiên cứu, nhấn mạnh đóng góp khoa học và thực tiễn, đánh giá tác động đến lĩnh vực quản lý giao thông thông minh và đưa ra khuyến nghị cho nghiên cứu tương lai.
\end{itemize}

\subsection*{Phần V: Phụ lục và tài liệu tham khảo}

Các phụ lục bao gồm mã nguồn chi tiết của hệ thống, hướng dẫn cài đặt và cấu hình môi trường, dữ liệu thí nghiệm và kịch bản mô phỏng, cùng với bảng thuật ngữ chuyên ngành. Phần tài liệu tham khảo liệt kê đầy đủ các nguồn được trích dẫn trong luận văn theo chuẩn IEEE.

% =========================
% CHƯƠNG 3
% =========================
% =========================
% CHƯƠNG 3
% =========================
\chapter{Tổng quan tài liệu}

\section{Phương pháp điều khiển giao thông truyền thống}

Các hệ thống điều khiển tín hiệu giao thông truyền thống chủ yếu gồm:
\begin{itemize}
    \item \textbf{Điều khiển chu kỳ cố định (Fixed-time):} Thời gian các pha được thiết kế trước dựa trên dữ liệu lưu lượng lịch sử, không thích ứng được với biến động thực tế. Ưu điểm là đơn giản, dễ triển khai; nhược điểm là dễ gây lãng phí hoặc tắc nghẽn khi lưu lượng thay đổi \cite{Webster1958, Urbanik2015}.
    \item \textbf{Điều khiển kích hoạt (Actuated):} Sử dụng cảm biến để phát hiện xe, điều chỉnh thời gian pha trong giới hạn định trước. Linh hoạt hơn, nhưng vẫn bị giới hạn bởi các tham số cứng \cite{Koonce2008}.
    \item \textbf{Điều khiển phối hợp (Coordinated):} Đồng bộ nhiều nút giao để tạo “làn sóng xanh”. Điển hình là hệ SCOOT, SCATS. Hiệu quả trên các tuyến chính, nhưng phức tạp khi áp dụng cho mạng lưới lớn \cite{Hunt1981, Lowrie1990}.
\end{itemize}

\section{Hệ thống điều khiển thích nghi, học tăng cường và phương pháp lai}

Để khắc phục hạn chế của các phương pháp truyền thống, các hệ thống thích nghi như SCOOT, SCATS, OPAC đã ra đời. Các hệ này điều chỉnh thời gian tín hiệu dựa trên trạng thái giao thông thực, tăng khả năng phản ứng với biến động \cite{Eom2020}. 

Gần đây, các kỹ thuật học tăng cường (Reinforcement Learning - RL) được ứng dụng mạnh mẽ, cho phép hệ thống tự học chính sách tối ưu thông qua tương tác và phần thưởng thực tế \cite{Abdulhai2003, Li2016}. RL giúp tối ưu hóa hiệu năng tổng thể và thích nghi với nhiều kịch bản khác nhau.

Xu hướng hiện nay là kết hợp các phương pháp: điều khiển luật, thích nghi và học máy (hybrid). Việc phối hợp này tận dụng ưu điểm từng cách tiếp cận, đảm bảo hệ thống vừa an toàn, vừa tối ưu hóa linh hoạt trong điều kiện phức tạp \cite{Wei2019}.

\section{Khoảng trống nghiên cứu}

Mặc dù có nhiều tiến bộ, vẫn tồn tại các vấn đề cần giải quyết:
\begin{itemize}
    \item \textbf{Khả năng mở rộng hạn chế:} RL truyền thống (Q-learning) khó khái quát cho mạng lưới lớn, thiếu hàm xấp xỉ như mạng nơ-ron sâu \cite{Li2016, Liang2019}.
    \item \textbf{Đánh giá chưa toàn diện:} Nhiều nghiên cứu thiếu benchmark chuẩn, kiểm định thống kê và phân tích đa chiều các chỉ số \cite{Shaikh2022}.
    \item \textbf{Phối hợp đa nút giao:} Ít giải pháp cho multi-agent coordination, chưa tối ưu cho mạng lưới lớn với độ trễ giao tiếp thực tế \cite{Kouvelas2011}.
    \item \textbf{Sim-to-real gap:} Kết quả mô phỏng khó chuyển giao sang ứng dụng thực tế, thiếu tích hợp cảm biến thực và domain randomization \cite{Lopez2018}.
    \item \textbf{Thiết kế reward và công bằng:} Hàm thưởng thường phức tạp, chưa phân tích ảnh hưởng các thành phần tới kết quả, khó đảm bảo fairness giữa các hướng giao thông \cite{Sutton2018}.
    \item \textbf{Đảm bảo an toàn:} Thiếu kiểm thử stress với các tình huống cực đoan, thiếu cơ chế xác minh ràng buộc cứng \cite{Mirchandani2001}.
\end{itemize}

\section{Kết luận chương}

Nhu cầu phát triển hệ thống điều khiển tín hiệu giao thông thông minh vẫn còn nhiều thách thức: khả năng mở rộng, phối hợp đa nút, chuyển giao thực tế, đánh giá chuẩn, và đảm bảo an toàn. Luận văn này hướng tới giải quyết một phần các khoảng trống trên bằng cách phát triển bộ điều khiển lai APC-RL với cơ chế ưu tiên, quản lý tắc nghẽn, rẽ trái bảo vệ và lưu trữ dữ liệu thực nghiệm đồng bộ, đặt nền móng cho các nghiên cứu mở rộng trong tương lai.
% =========================
% CHƯƠNG 4
% =========================
% =========================
% CHƯƠNG 4
% =========================
\chapter{Kiến trúc hệ thống}

\section{Thiết kế tổng thể}

Hệ thống điều khiển đèn giao thông thông minh được xây dựng theo mô hình phân tầng, kết hợp giữa điều khiển dựa trên luật (rule-based) và học tăng cường bằng máy (reinforcement learning). Kiến trúc này đảm bảo hệ thống vừa ổn định, an toàn, vừa có khả năng thích ứng và tối ưu theo trạng thái giao thông thực tế.

\subsection{Sơ đồ thành phần hệ thống}

\begin{figure}[H]
    \centering
    \includegraphics[width=1\linewidth]{Untitled diagram _ Mermaid Chart-2025-08-21-023904.png}
    \caption{Kiến trúc tổng thể hệ thống điều khiển đèn giao thông}
\end{figure}

Hệ thống gồm 4 tầng chính:
\begin{itemize}
    \item \textbf{Tầng mô phỏng}: Sử dụng SUMO làm môi trường giao thông vi mô, cung cấp dữ liệu thời gian thực về phương tiện, làn đường, đèn tín hiệu.
    \item \textbf{Tầng giao tiếp}: TraCI làm cầu nối hai chiều giữa SUMO và hệ điều khiển Python, cho phép truy vấn trạng thái và gửi lệnh điều khiển.
    \item \textbf{Tầng điều khiển}: UniversalSmartTrafficController quản lý một nút giao, điều phối các AdaptivePhaseController (APC) và RL agent để ra quyết định động, xử lý ưu tiên, tắc nghẽn, rẽ trái bảo vệ.
    \item \textbf{Tầng lưu trữ}: Supabase lưu trữ trạng thái, log sự kiện, lịch sử pha và Q-table; pickle file lưu Q-table cục bộ giúp agent khôi phục trạng thái học.
\end{itemize}

\section{Luồng dữ liệu và tích hợp}

Quy trình dữ liệu của hệ thống theo chu trình khép kín:
\begin{enumerate}
    \item TraCI đọc trạng thái giao thông từ SUMO (số lượng xe, hàng chờ, tốc độ...).
    \item UniversalSmartTrafficController tổng hợp, phân tích dữ liệu, đánh giá trạng thái nút giao.
    \item AdaptivePhaseController và RL agent quyết định chuyển pha, điều chỉnh thời lượng theo trạng thái và lịch sử học.
    \item Lệnh điều khiển gửi ngược về SUMO qua TraCI.
    \item Kết quả, phần thưởng, log sự kiện được ghi lên Supabase, Q-table được cập nhật để cải thiện hiệu năng lâu dài.
\end{enumerate}

Các điểm tích hợp chính:
\begin{itemize}
    \item \textbf{TraCI API}: Giao tiếp và thao tác trạng thái với SUMO, tối ưu hóa qua subscription.
    \item \textbf{Supabase REST API}: Đồng bộ trạng thái, log, Q-table lên cloud với batch write, retry logic.
    % \item \textbf{Corridor Coordinator}: Điều phối đa nút giao, phát hiện cluster congestion, tạo green wave.
    % Đã loại bỏ, vì hệ thống hiện tại chỉ điều khiển một nút giao.
\end{itemize}

\section{Các thành phần cốt lõi}

\subsection{UniversalSmartTrafficController}

Điều khiển trung tâm, quản lý APC cho một nút giao, thu thập và phân tích dữ liệu, nhận diện các tình huống đặc biệt (ưu tiên, tắc nghẽn, starvation).

\subsection{AdaptivePhaseController (APC)}

Điều khiển nút giao, thực hiện điều chỉnh động thời lượng pha, chèn pha vàng, bảo vệ rẽ trái, xử lý hàng đợi ưu tiên, phối hợp với RL agent và controller.

\subsection{EnhancedQLearningAgent}

Tác tử học tăng cường Q-learning, nhận vector trạng thái đa chiều, cập nhật Q-table, tối ưu hóa lựa chọn pha và thời lượng dựa trên reward đa mục tiêu.

\subsection{Giao tiếp SUMO–TraCI}

Các hàm bọc giúp kiểm soát pha an toàn, cache logic đèn tín hiệu (TTL 0.5s), quản lý subscription giảm overhead truyền tin, đảm bảo quá trình điều khiển ổn định.

\section{Công nghệ sử dụng}

\subsection{Python và thư viện}

Hệ thống phát triển trên Python 3.8+:
\begin{itemize}
    \item \textbf{Async/Await, Threading}: Xử lý bất đồng bộ, tăng hiệu suất
    \item \textbf{Type Hints, Dataclasses}: Quản lý dữ liệu rõ ràng, dễ bảo trì
    \item \textbf{NumPy, Pandas, Matplotlib}: Phân tích, xử lý và trực quan hóa dữ liệu
    \item \textbf{Supabase-py}: Kết nối cloud database
    \item \textbf{Pickle}: Lưu/đọc Q-table cục bộ
\end{itemize}

\subsection{Mô phỏng giao thông SUMO}

SUMO 1.22.0 với NETEDIT cho phép thiết kế mạng lưới, cấu hình traffic demand, xuất dữ liệu chi tiết phục vụ kiểm thử và đào tạo.

\subsection{Tích hợp API TraCI}

Các module chính:
\begin{itemize}
    \item \texttt{traci.trafficlight}: Điều khiển pha, logic đèn
    \item \texttt{traci.lane}: Đọc thông tin làn, hàng chờ
    \item \texttt{traci.vehicle}: Theo dõi phương tiện
    \item \texttt{traci.simulation}: Quản lý mô phỏng chung
\end{itemize}

\subsection{Cấu hình Supabase}

Supabase sử dụng PostgreSQL cloud với REST API và real-time. Hệ thống tối ưu hóa lưu trữ và truy vấn dữ liệu bằng cấu trúc bảng sau:

\subsubsection{Bảng apc\_states}

Lưu trữ trạng thái tổng thể APC, bao gồm cấu hình pha, hàng đợi sự kiện, trạng thái RL agent.

\begin{lstlisting}[style=sql,caption={Định nghĩa bảng apc\_states}]
CREATE TABLE apc_states (
    id BIGSERIAL PRIMARY KEY,
    tls_id TEXT NOT NULL,
    state_type TEXT NOT NULL,
    data JSONB NOT NULL,
    created_at TIMESTAMPTZ DEFAULT NOW(),
    updated_at TIMESTAMPTZ DEFAULT NOW()
);

CREATE INDEX idx_apc_states_tls_type ON apc_states(tls_id, state_type);
CREATE INDEX idx_apc_states_data ON apc_states USING gin(data);
\end{lstlisting}

Trường \texttt{state\_type} gồm:
\begin{itemize}
    \item \texttt{full}: Snapshot trạng thái toàn bộ APC
    \item \texttt{phase}: Thông tin chi tiết pha đèn
    \item \texttt{event}: Sự kiện điều khiển đơn lẻ
\end{itemize}
Sử dụng JSONB giúp mở rộng dữ liệu linh hoạt, truy vấn nhanh nhờ GIN index, dễ dàng tích hợp với Python.

\subsubsection{Bảng phase\_records}

Lưu lịch sử các lần điều chỉnh pha đèn, phục vụ phân tích hiệu năng và đào tạo RL agent.

\begin{table}[H]
\centering
\begin{tabular}{llp{7cm}}
\toprule
\textbf{Cột} & \textbf{Kiểu} & \textbf{Vai trò} \\
\midrule
tls\_id         & TEXT    & Định danh nút giao \\
phase\_idx      & INT     & Pha được điều chỉnh (0--11) \\
duration        & REAL    & Thời lượng thực tế \\
base\_duration  & REAL    & Thời lượng cơ sở \\
delta\_t        & REAL    & Mức điều chỉnh \\
extended\_time  & REAL    & Thời gian mở rộng \\
reward          & REAL    & Reward từ RL agent \\
sim\_time       & REAL    & Thời điểm mô phỏng \\
\bottomrule
\end{tabular}
\caption{Các trường cốt lõi của bảng phase\_records}
\end{table}

\subsubsection{Bảng simulation\_events}

Lưu mọi sự kiện đặc biệt: chuyển pha khẩn cấp, protected left, congestion, chuyển vàng, v.v.

\begin{itemize}
    \item \texttt{id}: Khóa chính
    \item \texttt{tls\_id}: ID đèn giao thông
    \item \texttt{event\_type}: Loại sự kiện
    \item \texttt{event\_date}: JSONB chi tiết sự kiện
    \item \texttt{sim\_time}: Thời điểm mô phỏng
    \item \texttt{created\_at}: Thời điểm tạo bản ghi
\end{itemize}

Ví dụ JSON:
\begin{lstlisting}[style=json]
{
    "action": "phase_duration_update",
    "phase": 209
}
\end{lstlisting}

\subsubsection{Tối ưu hiệu suất database}

\begin{itemize}
    \item \textbf{Indexing}: Composite index cho truy vấn nhanh
    \item \textbf{JSONB + GIN}: Truy vấn linh hoạt, hiệu suất cao
    \item \textbf{Row Level Security}: Bảo mật theo từng dòng dữ liệu
    \item \textbf{Batch operations}: Ghi dữ liệu theo lô giảm số lần round-trip
\end{itemize}

\begin{lstlisting}[style=sql]
ALTER TABLE apc_states ENABLE ROW LEVEL SECURITY;
CREATE POLICY "Enable all operations for authenticated users"
    ON apc_states FOR ALL USING (auth.role() = 'authenticated');
\end{lstlisting}

\subsubsection{Chiến lược đồng bộ dữ liệu}

\begin{itemize}
    \item Dữ liệu tạm thời gom vào \_pending\_db\_ops (tối đa 1000 bản ghi)
    \item AsyncSupabaseWriter ghi dữ liệu định kỳ lên Supabase (60s/batch)
    \item Retry logic thông minh, tối đa 6 lần, dùng exponential backoff
    \item Nếu Supabase offline, chuyển sang lưu local tạm thời
\end{itemize}

Thiết kế này giúp hệ thống an toàn, hiệu suất cao, sẵn sàng mở rộng cho thực nghiệm hoặc triển khai thực tế.

% =========================
% CHƯƠNG 5
% =========================
\chapter{Điều khiển pha thích nghi (APC)}

\section{Nguyên lý thiết kế và kiến trúc}

Bộ điều khiển pha thích nghi (Adaptive Phase Controller - APC) được thiết kế theo nguyên lý điều khiển phân tán với khả năng tự quyết định cục bộ cho từng nút giao thông. Kiến trúc này cho phép mỗi nút giao hoạt động độc lập trong khi vẫn có thể phối hợp với các nút lân cận thông qua cơ chế trao đổi thông tin. APC kết hợp giữa điều khiển dựa trên luật (rule-based control) để đảm bảo an toàn và điều khiển thích ứng (adaptive control) để tối ưu hiệu suất.

\begin{figure}[H]
    \centering
    \includegraphics[width=0.9\textwidth]{Untitled diagram _ Mermaid Chart-2025-08-21-074754.png}
    \caption{Kiến trúc tổng thể của bộ điều khiển APC với các thành phần chính và luồng dữ liệu}
    \label{fig:apc_architecture}
\end{figure}

\subsection{Khởi tạo và cấu hình hệ thống}

Quá trình khởi tạo APC được thực hiện qua constructor với các tham số quan trọng:

\begin{lstlisting}[style=py, caption={Khởi tạo AdaptivePhaseController}]
def __init__(self, lane_ids, tls_id, alpha=1.0, 
             min_green=30, max_green=80,
             r_base=0.5, r_adjust=0.1, 
             severe_congestion_threshold=0.8,
             large_delta_t=20):
    self.lane_ids = lane_ids
    self.tls_id = tls_id
    # Đăng ký subscription với TraCI
    for lid in self.lane_ids:
        traci.lane.subscribe(lid, [
            traci.constants.LAST_STEP_VEHICLE_HALTING_NUMBER,
            traci.constants.LAST_STEP_MEAN_SPEED,
            traci.constants.LAST_STEP_VEHICLE_NUMBER,
            traci.constants.LAST_STEP_VEHICLE_ID_LIST,
        ])
\end{lstlisting}

\textbf{Quá trình khởi tạo bao gồm các bước chính:}

\begin{enumerate}
    \item \textbf{Thiết lập subscription với TraCI:} Hệ thống đăng ký theo dõi các metrics quan trọng cho từng làn đường được quản lý, bao gồm số lượng xe dừng, tốc độ trung bình, tổng số xe và danh sách ID phương tiện. Cơ chế subscription giúp tối ưu băng thông bằng cách chỉ nhận dữ liệu cần thiết.
    
    \item \textbf{Khởi tạo cấu trúc dữ liệu:} Các container quan trọng được khởi tạo:
    \begin{itemize}
        \item \texttt{apc\_state}: Dictionary lưu trữ trạng thái toàn cục với events queue (maxlen=5000) và danh sách phases
        \item \texttt{pending\_requests}: Hàng đợi yêu cầu chuyển pha với cơ chế ưu tiên
        \item \texttt{phase\_cache}: Cache thông tin pha với TTL 30 giây để giảm truy vấn database
        \item \texttt{blocked\_left\_memory}: Dictionary theo dõi lịch sử làn rẽ trái bị chặn
    \end{itemize}
    
    \item \textbf{Kết nối database:} Khởi tạo \texttt{PatchedAsyncSupabaseWriter} với interval 60 giây và batch size 100 records để đồng bộ dữ liệu không đồng bộ lên cloud.
    
    \item \textbf{Tải trạng thái từ Supabase:} Phương thức \texttt{\_load\_apc\_state\_supabase()} được gọi để khôi phục trạng thái từ lần chạy trước, đảm bảo tính liên tục của hệ thống.
    
    \item \textbf{Đồng bộ pha với SUMO:} \texttt{preload\_phases\_from\_sumo()} đảm bảo tất cả các pha trong logic đèn được ghi nhận và có base duration phù hợp.
\end{enumerate}
\begin{figure}[H]
    \centering
    \begin{subfigure}[b]{0.65\textwidth}
        \centering
        \includegraphics[width=1.1\textwidth]{Untitled diagram _ Mermaid Chart-2025-08-21-084042.png}
        \caption{Khởi tạo và thiết lập}
    \end{subfigure}
    \hfill
    \begin{subfigure}[b]{0.65\textwidth}
        \centering
        \includegraphics[width=1.1\textwidth]{Untitled diagram _ Mermaid Chart-2025-08-21-084132.png}
        \caption{Khôi phục trạng thái và hoàn tất}
    \end{subfigure}
    \caption{Quy trình khởi tạo AdaptivePhaseController.}
    \label{fig:apc_init_flow}
\end{figure}
\subsection{Cấu trúc dữ liệu và tham số điều khiển}

APC sử dụng hệ thống tham số phân cấp để điều chỉnh hành vi:

\subsubsection{Tham số thời gian}
\begin{itemize}
    \item \texttt{min\_green} (mặc định 30s): Thời gian tối thiểu cho mỗi pha xanh, đảm bảo an toàn và công bằng
    \item \texttt{max\_green} (mặc định 80s): Giới hạn trên để tránh monopoly một hướng
    \item \texttt{cycle\_length} (mặc định 90s): Chu kỳ tham chiếu cho điều phối corridor
    \item \texttt{low\_demand\_extend\_cap} (4s): Giới hạn mở rộng khi nhu cầu thấp
\end{itemize}

\subsubsection{Tham số điều khiển thích ứng}
\begin{itemize}
    \item \texttt{alpha} (1.0): Hệ số học trong công thức điều chỉnh $\Delta t = \alpha(R - R_{target})$
    \item \texttt{r\_base} (0.5): Giá trị reward cơ sở cho thuật toán học
    \item \texttt{r\_adjust} (0.1): Hệ số điều chỉnh R\_target động
    \item \texttt{weights} (vector 4D): Trọng số cho [density, speed, wait, queue] trong hàm reward
\end{itemize}

\subsubsection{Ngưỡng phát hiện sự kiện}

\subsubsection{Cấu trúc dữ liệu chính}

\textbf{Activation State:} Dictionary theo dõi pha đang hoạt động:
\begin{lstlisting}[style=py]
self.activation = {
    "phase_idx": None,        # Chi so pha hien tai
    "start_time": 0.0,        # Thoi diem bat dau
    "base_duration": None,    # Thoi luong co so
    "desired_total": None     # Thoi luong mong muon
}
\end{lstlisting}

\textbf{Pending Requests Queue:} Danh sách các yêu cầu chuyển pha được sắp xếp theo priority và timestamp:
\begin{lstlisting}[style=py]
request = {
    "phase_idx": int,         # Pha muc tieu
    "priority": int,          # Muc uu tien (1-11)
    "priority_type": str,     # Loai: emergency, starvation...
    "extension_duration": float,  # Thoi luong yeu cau
    "timestamp": float        # Thoi diem tao yeu cau
}
\end{lstlisting}
\begin{figure}[H]
    \centering
    \includegraphics[width= 1.1\textwidth]{Untitled diagram _ Mermaid Chart-2025-08-22-072612.png}
    \caption{Cấu trúc hàng đợi yêu cầu với mức độ ưu tiên}
    \label{fig:priority_queue}
\end{figure}
\subsection{Tích hợp với TraCI và SUMO}

Việc tích hợp với SUMO thông qua TraCI được thực hiện qua nhiều lớp abstraction:

\subsubsection{Subscription Mechanism}
APC sử dụng TraCI subscription để nhận dữ liệu real-time hiệu quả:

\begin{lstlisting}[style=py, caption={Xử lý subscription results}]
def get_lane_stats(self, lane_id):
    res = traci.lane.getSubscriptionResults(lane_id) or {}
    q = float(res.get(
        traci.constants.LAST_STEP_VEHICLE_HALTING_NUMBER,
        traci.lane.getLastStepHaltingNumber(lane_id)
    ))
    v = float(res.get(
        traci.constants.LAST_STEP_MEAN_SPEED,
        traci.lane.getLastStepMeanSpeed(lane_id)
    ))
    return q, w, v, dens
\end{lstlisting}

\subsubsection{Logic Cache System}
Để giảm overhead communication, APC implement cache cho traffic light logic:

\begin{lstlisting}[style=py, caption={Cơ chế cache logic}]
def _get_logic(self):
    now = traci.simulation.getTime()
    if self._logic_cache is None or \
       now - self._logic_cache_at > self._logic_cache_ttl:
        self._logic_cache = get_current_logic(self.tls_id)
        self._logic_cache_at = now
    return self._logic_cache
\end{lstlisting}

Cache có TTL 0.5 giây và được invalidate khi có thay đổi cấu trúc pha. Hệ thống còn hỗ trợ shared cache ở controller level để tối ưu cho nhiều APC instances.

\subsubsection{Safe Control Wrappers}
Mọi lệnh điều khiển được wrap trong các hàm an toàn:

\begin{lstlisting}[style=py, caption={Safe phase control}]
def _apply_phase(self, phase_idx, duration):
    # Clamp phase index
    safe_idx = self._safe_phase_index(phase_idx, 
                                      force_reload=True)
    if safe_idx is None:
        return False
    
    # Try controller-level setter first
    if hasattr(self, "controller"):
        ok = self.controller._safe_set_phase(
            self.tls_id, safe_idx, duration
        )
        if ok:
            return True
            
    # Fallback to direct control
    return safe_set_phase(self.tls_id, safe_idx, duration)
\end{lstlisting}

Cơ chế này đảm bảo:
\begin{itemize}
    \item Phase index luôn nằm trong giới hạn hợp lệ
    \item Duration được giới hạn trong khoảng [min\_green, max\_green]
    \item Xử lý graceful khi SUMO reject lệnh điều khiển
    \item Đồng bộ state giữa APC và SUMO
\end{itemize}
Thiết kế tích hợp này cho phép APC hoạt động ổn định trong môi trường mô phỏng phức tạp, xử lý được các trường hợp đặc biệt như chỉ số pha vượt giới hạn, thay đổi cấu trúc mạng, và lỗi kết nối với supabse.
\section{Quản lý logic pha đèn tín hiệu}

Quản lý logic pha đèn tín hiệu là thành phần cốt lõi đảm bảo hoạt động an toàn và hiệu quả của hệ thống điều khiển. APC implement một hệ thống quản lý logic đa tầng với cơ chế cache thông minh, kiểm soát chuyển pha an toàn và xử lý xung đột tự động.

\subsection{Cơ chế cache và tối ưu truy xuất}

Hệ thống cache được thiết kế theo mô hình hai tầng nhằm giảm thiểu overhead communication với SUMO trong khi vẫn đảm bảo tính nhất quán của dữ liệu.

\subsubsection{Cache cục bộ APC}

Mỗi instance APC duy trì cache riêng cho traffic light logic với TTL (Time To Live) 0.5 giây:

\begin{lstlisting}[style=py, caption={Implementation của logic cache cục bộ}]
def _get_logic(self):
    now = traci.simulation.getTime()
    # Kiểm tra cache validity
    if self._logic_cache is None or \
       now - self._logic_cache_at > self._logic_cache_ttl:
        try:
            # Fetch fresh logic từ SUMO
            self._logic_cache = get_current_logic(self.tls_id)
            self._logic_cache_at = now
        except Exception:
            self._logic_cache = None
    return self._logic_cache
\end{lstlisting}

Cache hoạt động theo nguyên tắc:
\begin{itemize}
    \item \textbf{Lazy loading}: Logic chỉ được fetch khi cần thiết
    \item \textbf{Time-based invalidation}: Tự động expire sau 0.5 giây
    \item \textbf{Explicit invalidation}: Force refresh khi có mutation
\end{itemize}
\vspace{3cm}
\subsubsection{Shared cache ở Controller level}
Khi nhiều APC cùng hoạt động, hệ thống sử dụng shared cache để tối ưu:

\begin{lstlisting}[style=py, caption={Shared cache mechanism (chỉ với tls\_id: E3)}]
def _get_logic(self):
    controller = getattr(self, "controller", None)
    if controller and hasattr(controller, "tl_logic_cache"):
        entry = controller.tl_logic_cache.get(self.tls_id)
        if entry and (now - entry.get("at", -1)) <= self._logic_cache_ttl:
            return entry.get("logic")
        # Update shared cache
        logic = get_current_logic(self.tls_id)
        controller.tl_logic_cache[self.tls_id] = {
            "logic": logic, 
            "at": now
        }
        return logic
\end{lstlisting}

\vspace{1cm}

\begin{figure}[H]
    \centering
    % Chỉ thể hiện APC Instance với tls_id: E3
    \includegraphics[width=0.85\linewidth]{Untitled diagram _ Mermaid Chart-2025-08-22-044929.png}
    \caption{Kiến trúc cache hai tầng cho traffic light logic chỉ với APC: E3}
    \label{fig:cache_architecture}
\end{figure}

\subsubsection{Cache invalidation strategy}

Hệ thống implement invalidation thông minh để đảm bảo consistency:

\begin{lstlisting}[style=py, caption={Cache invalidation mechanism}]
def _invalidate_logic_cache(self, tl_id=None):
    # Invalidate local cache
    self._logic_cache = None
    self._logic_cache_at = -1.0
    
    # Propagate to controller level
    controller = getattr(self, "controller", None)
    if controller and hasattr(controller, "_invalidate_logic_cache"):
        controller._invalidate_logic_cache(self.tls_id)
\end{lstlisting}
Cache bị invalidate trong các trường hợp:
\begin{enumerate}
    \item Sau khi thêm/xóa/sửa pha (logic mutation)
    \item Khi phát hiện inconsistency với SUMO
    \item Theo yêu cầu explicit từ coordinator
    \item Khi topology mạng thay đổi
\end{enumerate}

\subsection{Điều khiển chuyển pha an toàn}

Việc chuyển pha được thực hiện qua nhiều lớp kiểm tra an toàn để đảm bảo không vi phạm ràng buộc và tránh xung đột.

\subsubsection{Phase index validation}

Trước khi áp dụng bất kỳ phase nào, hệ thống validate index:

\begin{lstlisting}[style=py, caption={Safe phase index clamping}]
def _safe_phase_index(self, idx, force_reload=False):
    try:
        if force_reload:
            self._invalidate_logic_cache()
        logic = self._get_logic()
        if not logic or len(logic.getPhases()) <= 0:
            return None
        n = len(logic.getPhases())
        # Clamp to valid range
        return max(0, min(idx, n - 1))
    except Exception:
        return None
\end{lstlisting}

\subsubsection{Multi-layer phase application}

Hệ thống áp dụng phase qua nhiều lớp fallback:

\begin{lstlisting}[style=py, caption={Hierarchical phase application}]
def _apply_phase(self, phase_idx, duration):
    # Layer 1: Validate and clamp
    safe_idx = self._safe_phase_index(phase_idx, force_reload=True)
    if safe_idx is None:
        return False
    
    # Layer 2: Try controller-level setter
    controller = getattr(self, "controller", None)
    if controller:
        ok = controller._safe_set_phase(
            self.tls_id, safe_idx, duration
        )
        if ok:
            return True
    
    # Layer 3: Direct SUMO control
    ok2 = safe_set_phase(self.tls_id, safe_idx, duration)
    return ok2
\end{lstlisting}

\begin{figure}[htbp]
    \centering
    \begin{minipage}[t]{0.48\textwidth}
        \centering
        \includegraphics[width=0.9\linewidth]{Untitled diagram _ Mermaid Chart-2025-08-22-064822.png}
        \caption*{(a) Kiểm tra & xác thực chỉ số pha}
    \end{minipage}
    \hfill
    \begin{minipage}[t]{0.48\textwidth}
        \centering
        \includegraphics[width=0.9\linewidth]{Untitled diagram _ Mermaid Chart-2025-08-22-064805.png}
        \caption*{(b) Xử lý thiết lập, hàng đợi và cập nhật trạng thái}
    \end{minipage}
    \caption{Sơ đồ quy trình kiểm soát chuyển pha: (a) kiểm tra chỉ số pha, (b) xử lý thiết lập và cập nhật trạng thái}
    \label{fig:phase_control_pair}
\end{figure}

\subsubsection{Minimum green enforcement}

Hệ thống enforce thời gian xanh tối thiểu để đảm bảo an toàn:

\begin{lstlisting}[style=py, caption={Minimum green time enforcement}]
def enforce_min_green(self):
    current_sim_time = traci.simulation.getTime()
    elapsed = current_sim_time - self.last_phase_switch_sim_time
    
    if elapsed < self.min_green:
        logger.info(f"[MIN_GREEN ENFORCED] {self.tls_id}: "
                   f"Only {elapsed:.2f}s since last switch")
        return False  # Block phase change
    return True
\end{lstlisting}

Các exception cho min\_green:
\begin{itemize}
    \item Emergency vehicle detection
    \item Protected left turn activation
    \item Critical starvation (>3× max\_green wait time)
\end{itemize}

\subsection{Chèn pha đèn vàng tự động}

Hệ thống tự động phát hiện và chèn pha vàng khi chuyển từ xanh sang đỏ, đảm bảo an toàn giao thông.

\subsubsection{Yellow phase detection algorithm}

Thuật toán xác định khi nào cần pha vàng:

\begin{lstlisting}[style=py, caption={Yellow phase necessity detection}]
def insert_yellow_phase_if_needed(self, from_phase, to_phase):
    if from_phase == to_phase:
        return False
        
    logic = self._get_logic()
    from_state = logic.phases[from_phase].state
    to_state = logic.phases[to_phase].state
    
    # Check each signal head for G->R transition
    yellow_needed = False
    yellow = list(from_state)
    for i in range(min(len(from_state), len(to_state))):
        if from_state[i].upper() == 'G' and \
           to_state[i].upper() == 'R':
            yellow[i] = 'y'
            yellow_needed = True
    
    if not yellow_needed:
        return False
\end{lstlisting}

\subsubsection{Dynamic yellow phase creation}

Khi không tồn tại pha vàng phù hợp, hệ thống tạo động:

\begin{lstlisting}[style=py, caption={Dynamic yellow phase creation}]
    yellow_state_str = ''.join(yellow)
    
    # Search for existing yellow phase
    yellow_idx = None
    for idx, ph in enumerate(logic.phases):
        if ph.state == yellow_state_str:
            yellow_idx = idx
            break
    
    if yellow_idx is None and self.create_yellow_if_missing:
        # Create new yellow phase
        phases = list(logic.phases)
        yellow_phase = traci.trafficlight.Phase(3.0, yellow_state_str)
        
        if len(phases) < 12:  # SUMO limit
            phases.append(yellow_phase)
            new_idx = len(phases) - 1
        else:
            # Overwrite least used phase
            new_idx = self.find_phase_to_overwrite(yellow_state_str)
            phases[new_idx] = yellow_phase
            
        # Apply new logic
        new_logic = traci.trafficlight.Logic(
            logic.programID, logic.type, 
            logic.currentPhaseIndex, phases
        )
        traci.trafficlight.setCompleteRedYellowGreenDefinition(
            self.tls_id, new_logic
        )
\end{lstlisting}

\begin{figure}[H]
    \centering
    \includegraphics[width=0.75\linewidth]{Untitled diagram _ Mermaid Chart-2025-08-22-070255.png}
    \caption{Sơ đồ chuyển trạng thái với pha vàng tự động}
    \label{fig:placeholder}
\end{figure}

\subsection{Xử lý và ngăn chặn xung đột pha}

Hệ thống implement nhiều cơ chế để phát hiện và ngăn chặn xung đột pha nguy hiểm.

\subsubsection{Conflict detection matrix}

Xây dựng ma trận xung đột cho các movement:

\begin{lstlisting}[style=py, caption={Phase conflict detection}]
def detect_phase_conflicts(self, phase_state):
    controlled_links = traci.trafficlight.getControlledLinks(self.tls_id)
    conflicts = []
    
    for i, link_i in enumerate(controlled_links):
        if phase_state[i].upper() != 'G':
            continue
            
        for j, link_j in enumerate(controlled_links):
            if i == j or phase_state[j].upper() != 'G':
                continue
                
            # Check for conflicting movements
            if self.movements_conflict(link_i, link_j):
                conflicts.append((i, j))
                
    return conflicts
\end{lstlisting}

\subsubsection{Rate limiting for logic mutations}

Ngăn chặn rapid phase changes gây flicker:

\begin{lstlisting}[style=py, caption={Logic mutation rate limiting}]
def _can_mutate_logic(self):
    now = traci.simulation.getTime()
    cooldown = 2.0  # seconds
    
    if now - self._last_logic_mutation < cooldown:
        logger.info(f"[RATE-LIMIT] Skipping logic mutation; "
                   f"cooldown {cooldown}s")
        return False
        
    self._last_logic_mutation = now
    return True
\end{lstlisting}

\subsubsection{Quy tắc thiết lập pha đèn}

Hệ thống đảm bảo các quy tắc an toàn:

\begin{enumerate}
    \item \textbf{No all-red prevention}: Mọi pha phải có ít nhất một pha xanh
    \item \textbf{Conflicting movement check}: Không cho phép pha xanh đồng thời cho các hướng xung đột
    \item \textbf{Yellow transition requirement}: Bắt buộc yellow giữa các pha xanh xung đột
    \item \textbf{Maximum phase count}: Giới hạn 12 pha theo mặc định hệ thông giả lập SUMO
\end{enumerate}

\begin{figure}[htbp]
    \centering
    \includegraphics[width=1.0\linewidth]{Untitled diagram _ Mermaid Chart-2025-08-22-074409.png}
    \caption{Ma trận xung đột pha cho nút giao 4 hướng}
    \label{fig:conflict_matrix}
\end{figure}
\subsubsection{Cơ chế phục hồi}

Khi phát hiện vi phạm, hệ thống tự động khôi phục:

\begin{lstlisting}[style=py, caption={Automatic conflict resolution}]
def ensure_phases_have_green(self):
    logic = self._get_logic()
    changed = False
    
    for idx, phase in enumerate(logic.getPhases()):
        if 'G' not in phase.state:
            # Find first red and convert to green
            state_list = list(phase.state)
            for i, ch in enumerate(state_list):
                if ch == 'r':
                    state_list[i] = 'G'
                    break
                    
            new_state = ''.join(state_list)
            logger.info(f"[PATCH] Phase {idx} had no green, "
                       f"fixing: {phase.state} $\rightarrow$ {new_state}")
            self.overwrite_phase(idx, new_state, phase.duration)
            changed = True
            
    if changed:
        logger.info("[PATCH] All phases now have at least one green")
\end{lstlisting}

Hệ thống quản lý logic pha này đảm bảo hoạt động an toàn, hiệu quả và khả năng phục hồi cao cho điều khiển đèn giao thông trong mọi điều kiện vận hành.

\section{Điều chỉnh thời lượng pha động}

Mục tiêu của phần này là mô tả rõ cách bộ điều khiển APC điều chỉnh thời lượng pha xanh theo thời gian thực dựa trên phần thưởng vận hành, ràng buộc an toàn, nhu cầu tức thời và cơ chế tránh nhấp nháy. Phần cài đặt tương ứng trong mã nguồn nằm chủ yếu ở các hàm:
\begin{itemize}
    \item \texttt{calculate\_reward()}, \texttt{compute\_reward\_and\_bonus()} -- tính toán phần thưởng tổng hợp từ mật độ, tốc độ, thời gian chờ và hàng chờ.
    \item \texttt{calculate\_delta\_t\_and\_penalty()} -- chuyển phần thưởng thành \(\Delta t\) (thời gian mở rộng/rút ngắn) với làm mượt và phạt điều chỉnh lớn.
    \item \texttt{apply\_extension\_delta()} và \texttt{\_maybe\_update\_phase\_remaining()} -- áp dụng \(\Delta t\) vào pha đang hoạt động, cập nhật remaining time một cách an toàn.
    \item \texttt{enforce\_min\_green()} -- đảm bảo ràng buộc thời gian xanh tối thiểu; các ngoại lệ cho ưu tiên.
\end{itemize}

\begin{figure}[H]
    \centering
    \fbox{\rule[0pt]{0pt}{0.35\textwidth}\rule[0.8\textwidth]{0pt}{0pt}}
    \caption{Placeholder: Sơ đồ tổng quan pipeline điều chỉnh thời lượng pha động (Reward $\rightarrow$ Delta-t $\rightarrow$ Extension)}
    \label{fig:dynamic_timing_pipeline}
\end{figure}

\subsection{Thuật toán điều chỉnh delta-t}

Thuật toán chuyển đổi phần thưởng (\(R\)) thành giá trị điều chỉnh thời lượng \(\Delta t\) được thiết kế để:
\begin{enumerate}
    \item Phản ứng theo sai lệch so với mục tiêu (\(R - R_{\text{target}}\)),
    \item Làm mượt bằng hàm \(\tanh\) để tránh bước nhảy lớn,
    \item Áp dụng phạt khi điều chỉnh quá lớn so với ngưỡng an toàn đã định.
\end{enumerate}

\paragraph{Công thức chính}
\[
\text{raw\_}\Delta t = \alpha \cdot (R - R_{\text{target}}), \quad
\Delta t = s \cdot \tanh\left(\frac{\text{raw\_}\Delta t}{s}\right), \quad
\Delta t \in [\Delta t_{\min}, \Delta t_{\max}]
\]
Trong đó \(\alpha\) là hệ số học/điều chỉnh, \(s\) là thang làm mượt (ví dụ 20), và \([\Delta t_{\min}, \Delta t_{\max}]\) là giới hạn kẹp để tránh điều chỉnh cực đoan. Phạt cho điều chỉnh lớn:
\[
\text{penalty} = \max \left(0, \left| \text{raw\_}\Delta t \right| - \Delta t_{\text{large}} \right)
\]

\begin{lstlisting}[style=py, caption={Trích đoạn: Tính delta-t và penalty}, label={lst:delta_t}]
def calculate_delta_t_and_penalty(self, R):
    raw_delta_t = self.alpha * (R - self.R_target)
    penalty = max(0, abs(raw_delta_t) - self.large_delta_t)
    ext_t = 20 * np.tanh(raw_delta_t / 20)
    delta_t = np.clip(ext_t, -20, 20)
    return raw_delta_t, delta_t, penalty
\end{lstlisting}

Giá trị \(R_{\text{target}}\) được cập nhật động để bám theo xu hướng hiệu năng gần đây:
\begin{lstlisting}[style=py, caption={Cập nhật R_target động}]
def update_R_target(self, window=10):
    if len(self.reward_history) < window or self.phase_count % 10 != 0:
        return
    avg_R = np.mean(list(self.reward_history)[-window:])
    self.R_target = self.r_base + self.r_adjust * (avg_R - self.r_base)
\end{lstlisting}

\begin{figure}[H]
    \centering
    \fbox{\rule[0pt]{0pt}{0.35\textwidth}\rule[0.8\textwidth]{0pt}{0pt}}
    \caption{Placeholder: Biểu đồ ánh xạ R $\rightarrow$ raw-$\Delta t$ $\rightarrow$ $\Delta t$ sau làm mượt và kẹp biên}
    \label{fig:delta_t_mapping}
\end{figure}

\subsection{Cơ chế extension và cập nhật remaining time}

Khi \(\Delta t\) đã được xác định, APC áp dụng thời lượng mới theo hai lớp:
\begin{enumerate}
    \item Tính \emph{desired\_total} cho kích hoạt hiện tại: \(\text{desired\_total} = \text{base\_duration} + \Delta t\), có kẹp về \([\texttt{min\_green}, \texttt{max\_green}]\).
    \item Cập nhật \emph{remaining time} chỉ khi khác biệt đáng kể (vượt \emph{buffer}) để tránh spam lệnh.
\end{enumerate}

Cơ chế ưu tiên ngăn kéo đuôi khi nhu cầu hiện tại thấp (\texttt{low\_demand\_extend\_cap}):
\begin{lstlisting}[style=py, caption={Áp dụng extension tương ứng với delta-t}, label={lst:apply_ext}]
def apply_extension_delta(self, delta_t, buffer=0.5):
    base = self.activation["base_duration"] or self.min_green
    desired_total = float(np.clip(base + float(delta_t), self.min_green, self.max_green))
    if self._phase_has_low_current_demand(min_total_halted=self.low_demand_min_halted):
        elapsed = self._get_phase_elapsed()
        desired_total = min(desired_total, elapsed + self.low_demand_extend_cap)
    self._maybe_update_phase_remaining(desired_total, buffer=0.3)
    return desired_total
\end{lstlisting}

Cập nhật remaining time có kiểm soát:
\begin{lstlisting}[style=py, caption={Cập nhật remaining time an toàn}, label={lst:update_remaining}]
def _maybe_update_phase_remaining(self, desired_total, buffer=0.5):
    elapsed = self._get_phase_elapsed()
    remaining = self._get_phase_remaining()
    desired_remaining = max(0.0, float(desired_total) - elapsed)
    if abs(remaining - desired_remaining) > float(buffer):
        traci.trafficlight.setPhaseDuration(self.tls_id, desired_remaining)
        self.activation["desired_total"] = float(desired_total)
        # Cập nhật nhật ký và Supabase
        current_phase = traci.trafficlight.getPhase(self.tls_id)
        total_after_update = elapsed + desired_remaining
        phase_record = self.load_phase_from_supabase(current_phase)
        base = phase_record.get("base_duration", self.min_green) if phase_record else self.min_green
        extended_time = max(0.0, total_after_update - base)
        self.update_phase_duration_record(current_phase, total_after_update, extended_time)
\end{lstlisting}

\noindent APC cũng phát ra telemetry extension theo ngưỡng để phục vụ trực quan hóa:
\begin{lstlisting}[style=py, caption={Telemetry extension ngưỡng 0.5s}]
def emit_extension_telemetry(self, threshold=0.5):
    # ... tính extended và phát sự kiện nếu thay đổi đáng kể ...
\end{lstlisting}

\begin{figure}[H]
    \centering
    \fbox{\rule[0pt]{0pt}{0.35\textwidth}\rule[0.8\textwidth]{0pt}{0pt}}
    \caption{Placeholder: Lưu đồ áp dụng extension và cập nhật remaining time với buffer}
    \label{fig:extension_flow}
\end{figure}

\subsection{Ràng buộc \texttt{min\_green} và \texttt{max\_green}}

Mọi \emph{desired\_total} đều bị kẹp trong \([\texttt{min\_green}, \texttt{max\_green}]\). Đồng thời, bộ điều khiển chặn chuyển pha nếu chưa đủ \texttt{min\_green}, trừ các ngoại lệ ưu tiên:
\begin{itemize}
    \item Phát hiện phương tiện khẩn cấp,
    \item Kích hoạt \emph{protected left} thực sự,
    \item \emph{Critical starvation} (thời gian chờ vượt ngưỡng nghiêm trọng).
\end{itemize}

\begin{lstlisting}[style=py, caption={Enforce thời gian xanh tối thiểu với ngoại lệ}, label={lst:min_green}]
def enforce_min_green(self):
    current_sim_time = traci.simulation.getTime()
    elapsed = current_sim_time - self.last_phase_switch_sim_time
    if elapsed < self.min_green:
        return False
    return True
\end{lstlisting}

Ngoài ra, khi nhu cầu hiện tại xuống thấp, APC sẽ chỉ cho phép kéo đuôi ngắn (mặc định \texttt{low\_demand\_extend\_cap} = 4s) để giải phóng vòng nhanh, giảm \emph{empty greens}. Mọi cập nhật remaining time đều đi qua bộ đệm \emph{buffer} để tránh spam lệnh và nhấp nháy.

\subsection{Tính toán thời lượng tối ưu theo nhu cầu}

Bên cạnh cơ chế \(\Delta t\), APC cung cấp các phép tính thời lượng mang tính heuristics theo nhu cầu:
\begin{itemize}
    \item \textbf{Theo pha:} \texttt{calculate\_adaptive\_duration(phase\_idx)} dựa tổng hàng chờ trên các lane xanh của pha đó: pha trống sẽ rất ngắn; pha bận rộn mở rộng đến giới hạn.
    \item \textbf{Theo làn:} \texttt{calculate\_optimal\_green\_time(lane\_id)} dùng thời gian giải tỏa hàng chờ và dung lượng downstream:
\end{itemize}

\[
T_{\text{clear}} \approx 2.0 \cdot Q, \qquad
T_{\text{downstream}} \approx 2.0 \cdot C_{\downarrow}, \qquad
T^{\star} = \min\left(\texttt{max\_green}, \max\left(\texttt{min\_green}, T_{\text{clear}}, 2T_{\text{downstream}}\right) + k \cdot \lambda_{\text{arr}}\right)
\]
với \(Q\) là hàng chờ, \(C_{\downarrow}\) là dung lượng downstream, \(\lambda_{\text{arr}}\) là tốc độ đến, và \(k\) là hệ số nhỏ để phản ánh nhu cầu sắp tới.

\begin{lstlisting}[style=py, caption={Ví dụ: tính thời lượng pha theo nhu cầu}, label={lst:adaptive_duration}]
def calculate_adaptive_duration(self, phase_idx):
    base = self.min_green
    q_total = self._phase_green_total_queue(phase_idx)
    if q_total == 0:
        return max(3, self.min_green // 3)
    elif q_total < 3:
        return self.min_green
    else:
        return min(self.max_green, self.min_green + q_total * 2)
\end{lstlisting}

Để chọn pha tốt nhất khi chuyển pha, APC sử dụng \texttt{find\_best\_phase\_for\_traffic()} tổng hợp các yếu tố:
\begin{itemize}
    \item Hàng chờ lớn nhất (\emph{max queue}) và tổng cầu (\emph{sum queue}),
    \item Thời gian chờ, starvation, áp lực downstream,
    \item Phạt mạnh cho \emph{empty greens} và có \emph{hysteresis} (+10\%) để tránh dao động.
\end{itemize}

\begin{figure}[H]
    \centering
    \fbox{\rule[0pt]{0pt}{0.35\textwidth}\rule[0.8\textwidth]{0pt}{0pt}}
    \caption{Placeholder: Sơ đồ quyết định thời lượng tối ưu theo hàng chờ, tốc độ đến và dung lượng downstream}
    \label{fig:optimal_time_decision}
\end{figure}

\paragraph{Tóm tắt vận hành}
\begin{enumerate}
    \item Tính \(R\), cập nhật \(R_{\text{target}}\) định kỳ, suy ra \(\Delta t\) với làm mượt, kèm penalty.
    \item Từ \(\Delta t\), tính \emph{desired\_total}, kẹp theo \texttt{min\_green}/\texttt{max\_green}, xét \emph{low\_demand\_extend\_cap}.
    \item Cập nhật remaining time nếu chênh lệch vượt \emph{buffer}, ghi nhận vào Supabase và telemetry.
    \item Khi chuyển pha: chọn pha tối ưu theo nhu cầu, đảm bảo chèn vàng và duy trì \texttt{min\_green} trừ ngoại lệ ưu tiên.
\end{enumerate}
\section{Hệ thống quản lý yêu cầu ưu tiên}
\subsection{Cấu trúc hàng đợi yêu cầu phân cấp}
\subsection{Xử lý yêu cầu theo mức độ ưu tiên}
\subsection{Cơ chế stacked requests và batch processing}
\subsection{Giải quyết xung đột giữa các yêu cầu}

\section{Phát hiện và xử lý tắc nghẽn}
\subsection{Thuật toán phát hiện congestion patterns}
\subsection{Tính toán chỉ số severity đa yếu tố}
\subsection{Dự báo và phòng ngừa tắc nghẽn}
\subsection{Kích hoạt chế độ congestion mode}

\section{Quản lý rẽ trái bảo vệ thông minh}
\subsection{Phát hiện blocked left turn với xung đột}
\subsection{Tạo pha protected left động}
\subsection{Cơ chế memory và guard deadline}
\subsection{Tối ưu thời lượng protected left phase}

\section{Xử lý tình huống khẩn cấp}
\subsection{Phát hiện phương tiện ưu tiên}
\subsection{Emergency rebalancing cho lane imbalance}
\subsection{Xử lý starvation và critical congestion}
\subsection{Cơ chế cooldown và ngăn chặn flicker}

\section{Tạo và quản lý pha động}
\subsection{Thuật toán tạo pha mới tự động}
\subsection{Overwrite và tái sử dụng pha}
\subsection{Tối ưu số lượng pha với giới hạn SUMO}
\subsection{Sinh tập pha tối ưu cho topology}

\section{Tích hợp với hệ thống lưu trữ}
\subsection{Đồng bộ trạng thái với Supabase}
\subsection{Cơ chế batch write và retry logic}
\subsection{Lưu trữ lịch sử điều chỉnh pha}
\subsection{Event logging và telemetry}

\section{Vòng lặp điều khiển chính}
\subsection{Luồng xử lý control_step}
\subsection{Phối hợp với RL agent}
\subsection{Cập nhật trạng thái lane serving}
\subsection{Xử lý phase ending và transitions}

\section{Đánh giá hiệu năng và metrics}
\subsection{Tính toán reward đa mục tiêu}
\subsection{Điều chỉnh trọng số thích ứng}
\subsection{Cập nhật R_target động}
\subsection{Phân tích bonus và penalty}

\section{Tối ưu hóa và cải tiến}
\subsection{Hysteresis để tránh dao động}
\subsection{Rate limiting cho logic mutations}
\subsection{Decay mechanisms cho memory}
\subsection{Xử lý corner cases và edge conditions}
% =========================
% CHƯƠNG 6
% =========================
\chapter{Thành phần học tăng cường}
\section{Kiến trúc Agent Q-learning cải tiến}
\section{Cơ chế học}
\section{Thiết kế vector trạng thái}
\section{Thiết kế hàm thưởng}

% =========================
% CHƯƠNG 7
% =========================
\chapter{Chiến lược điều khiển lai}
\section{Khung tích hợp}
\section{Luồng điều khiển}
\section{Giải quyết xung đột}

% =========================
% CHƯƠNG 8
% =========================
\chapter{Quản lý xe ưu tiên}
\section{Phát hiện xe khẩn cấp}
\section{Chiến lược ưu tiên}
\section{Phân tích tác động hiệu năng}

% =========================
% CHƯƠNG 9
% =========================
\chapter{Quản lý dữ liệu}
\section{Tích hợp Supabase}
\section{Lưu trữ Q-Table}
\section{Hệ thống ghi log sự kiện}

% =========================
% CHƯƠNG 10
% =========================
\chapter{Chi tiết triển khai}
\section{Cài đặt và cấu hình}
\section{Tham số cấu hình}
\section{Cấu trúc mã nguồn}

% =========================
% CHƯƠNG 11
% =========================
\chapter{Mô hình thí nghiệm}
\section{Môi trường mô phỏng}
\section{Cấu hình cơ bản}
\section{Thí nghiệm kiểm soát}

% =========================
% CHƯƠNG 12
% =========================
\chapter{Đánh giá hiệu năng và so sánh}
\section{Các chỉ số đánh giá}
\section{So sánh định lượng và định tính}
\section{Đánh giá theo kịch bản}
\section{Phân tích đường cong học tập}

% =========================
% CHƯƠNG 13
% =========================
\chapter{Kết quả và thảo luận}
\section{Phát hiện chính}
\section{Phân tích hành vi hệ thống}
\section{Hạn chế và thách thức}

% =========================
% CHƯƠNG 14
% =========================
\chapter{Trực quan hóa và giám sát}
\section{Bảng điều khiển thời gian thực}
\section{Công cụ phân tích}

% =========================
% CHƯƠNG 15
% =========================
\chapter{Hướng phát triển tương lai}
\section{Điều phối đa nút giao}
\section{Tích hợp học tăng cường sâu}
\section{Giao tiếp xe--hạ tầng}
\section{Dự báo giao thông}
\section{Mở rộng dựa trên điện toán đám mây}

% =========================
% CHƯƠNG 16
% =========================
\chapter{Kết luận}
\section{Tóm tắt kết quả}
\section{Tác động đến quản lý giao thông}
\section{Ứng dụng thực tiễn}
\section{Khuyến nghị cuối cùng}

% ===== Bibliography =====
\cleardoublepage
\phantomsection
\addcontentsline{toc}{chapter}{Tài liệu tham khảo}
\printbibliography[title={Tài liệu tham khảo}]

% ===== Appendices =====
\appendix

\chapter{Tài liệu mã nguồn}
% Đảm bảo đã thêm \usepackage{listings} và cấu hình style trong phần preamble
% Ví dụ chèn file mã nguồn chính:
% \lstinputlisting[style=py,caption={Bộ điều khiển Lane7b.py}]{Lane7b.py}

\chapter{Hướng dẫn cài đặt}
% Hướng dẫn cài Python, SUMO, cấu hình TraCI, tham số...

\chapter{Dữ liệu thí nghiệm}
% Link/ mô tả dữ liệu, cấu hình traffic demand...

\chapter{Thuật ngữ}
% Giải thích các thuật ngữ chuyên ngành bổ sung.

% ======= Bảng phụ lục chi tiết schema =======
\chapter{Phụ lục: Lược đồ bảng phase\_records}
% File: appendix_phase_records2.tex
% Sử dụng file này như một chương phụ lục riêng sau \appendix trong main.tex

\chapter{Phụ lục: Lược đồ bảng \texttt{phase\_records}}

\section*{Lược đồ đầy đủ bảng \texttt{phase\_records}}
\addcontentsline{toc}{chapter}{Phụ lục: Lược đồ bảng phase\_records}
\label{app:phase_records_full}

Bảng \texttt{phase\_records} lưu lịch sử chi tiết mọi lần điều chỉnh pha trong hệ thống, bao gồm thông tin thời gian, điều khiển và chỉ số hiệu suất. Bảng này được sử dụng cho phân tích hậu kỳ, huấn luyện RL, kiểm toán và debug.

\vspace{0.5em}
\begin{table}[H]
\centering
\footnotesize
\begin{tabular}{@{}llp{7.2cm}@{}}
\toprule
\textbf{Cột} & \textbf{Kiểu dữ liệu} & \textbf{Mô tả} \\
\midrule
\multicolumn{3}{@{}l}{\textit{Thông tin cơ bản}}\\
\midrule
id            & BIGSERIAL      & Khóa chính tự tăng \\
tls\_id       & TEXT           & ID của nút giao thông \\
phase\_idx    & INTEGER        & Chỉ số pha (0--11) \\
sim\_time     & REAL           & Thời điểm trong mô phỏng (giây) \\
\midrule
\multicolumn{3}{@{}l}{\textit{Thông số thời gian}}\\
\midrule
duration      & REAL           & Thời lượng thực tế của pha (giây) \\
base\_duration& REAL           & Thời lượng cơ sở theo cấu hình \\
delta\_t      & REAL           & Mức điều chỉnh sau làm mượt \\
raw\_delta\_t & REAL           & Mức điều chỉnh ban đầu (trước smoothing) \\
extended\_time& REAL           & Thời gian mở rộng thêm so với base \\
\midrule
\multicolumn{3}{@{}l}{\textit{Thông tin điều khiển}}\\
\midrule
state\_str    & TEXT           & Chuỗi trạng thái (ví dụ ``GGrrGGrr'') \\
event\_type   & TEXT           & Loại sự kiện kích hoạt điều chỉnh (emergency, starvation, rl, ...) \\
weights       & JSONB          & Trọng số các yếu tố điều khiển (density, speed, wait, queue) \\
lanes         & JSONB          & Danh sách làn được phục vụ (dùng cho phân tích theo làn) \\
\midrule
\multicolumn{3}{@{}l}{\textit{Đánh giá hiệu suất}}\\
\midrule
reward        & REAL           & Phần thưởng tính cho hành động (RL) \\
penalty       & REAL           & Hình phạt cho điều chỉnh quá mức \\
bonus         & REAL           & Điểm thưởng/bonus (nếu có) \\
\midrule
\multicolumn{3}{@{}l}{\textit{Metadata \& auditing}}\\
\midrule
created\_at   & TIMESTAMPTZ    & Thời điểm tạo bản ghi \\
updated\_at   & TIMESTAMPTZ    & Thời điểm cập nhật cuối \\
\bottomrule
\end{tabular}
\caption{Lược đồ đầy đủ của bảng \texttt{phase\_records}}
\label{tab:phase_records_detailed_full}
\end{table}

\section*{Ghi chú triển khai}
\begin{itemize}
  \item \textbf{Trường JSONB (weights, lanes):} Dùng JSONB để hỗ trợ indexing (GIN) và mở rộng linh hoạt khi cần thêm trường không cố định.
  \item \textbf{Chỉ mục gợi ý:} \texttt{CREATE INDEX idx\_phase\_records\_tls\_phase ON phase\_records(tls\_id, phase\_idx);} \\
    \texttt{CREATE INDEX idx\_phase\_records\_simtime ON phase\_records(sim\_time);}
  \item \textbf{Lưu trữ/archival:} Khi triển khai thực tế, cân nhắc chính sách lưu trữ (giữ dữ liệu tần suất cao trong 3 tháng, sau đó nén/archive).
\end{itemize}

\end{document}