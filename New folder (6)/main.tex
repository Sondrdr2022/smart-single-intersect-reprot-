\documentclass[12pt,a4paper,oneside]{report}
\UseRawInputEncoding

% ===== Vietnamese + Encoding =====
\usepackage[T5]{fontenc}
\usepackage[utf8]{inputenc}
\usepackage[vietnamese]{babel}

% ===== Page layout & typography =====
\usepackage[a4paper,margin=2.5cm]{geometry}
\usepackage{setspace}
\onehalfspacing
\usepackage{microtype}
\setlength{\headheight}{18pt} % Fix fancyhdr warning
\usepackage{lmodern} % Latin Modern, scalable Computer Modern
% ===== Graphics, tables, math =====
\usepackage{graphicx}
\usepackage{float}
\usepackage{subcaption}
\usepackage{booktabs}
\usepackage{siunitx}
\sisetup{mode=match, propagate-math-font=true, round-mode=places, round-precision=2}
\usepackage{amsmath,amssymb,mathtools}

% ===== Lists & formatting =====
\usepackage{enumitem}
\setlist{noitemsep,leftmargin=2em}

% ===== TOC & Lists of figures/tables =====
\usepackage{tocloft}
\setlength{\cftbeforechapskip}{0.5em}
\renewcommand{\cftchapleader}{\cftdotfill{\cftdotsep}}

% ===== Hyperlinks & clever references =====
\usepackage{xcolor}
\definecolor{Link}{RGB}{0,0,128}
\definecolor{Cite}{RGB}{0,102,0}
\definecolor{URL}{RGB}{128,0,0}
\usepackage[
    unicode,
    pdfencoding=auto,
    colorlinks=true,
    linkcolor=Link,
    citecolor=Cite,
    urlcolor=URL
]{hyperref}
\usepackage{cleveref}

% ===== Header/Footer =====
\usepackage{fancyhdr}
\pagestyle{fancy}
\fancyhf{}
\lhead{\leftmark}
\rhead{\thepage}

% ===== Code listings (Python, JSON) =====
\usepackage{listings}
\lstdefinestyle{py}{
  language=Python,
  basicstyle=\ttfamily\footnotesize,
  numbers=left,
  numberstyle=\tiny,
  stepnumber=1,
  frame=single,
  breaklines=true
}
\lstdefinestyle{sql}{
  language=SQL,
  basicstyle=\ttfamily\footnotesize,
  numbers=left,
  numberstyle=\tiny,
  stepnumber=1,
  frame=single,
  breaklines=true
}
\lstdefinestyle{json}{
  language=,
  basicstyle=\ttfamily\footnotesize,
  numbers=left,
  numberstyle=\tiny,
  stepnumber=1,
  frame=single,
  breaklines=true
}

% ===== Glossaries / Acronyms (optional) =====
\usepackage[acronym]{glossaries}
\makeglossaries
\newacronym{apc}{APC}{Adaptive Phase Control}
\newacronym{rl}{RL}{Reinforcement Learning}
\newacronym{sumo}{SUMO}{Simulation of Urban MObility}
\newacronym{traci}{TraCI}{Traffic Control Interface}

% ===== Bibliography (biblatex with IEEE style) =====
\usepackage[backend=biber,style=ieee,sorting=none]{biblatex}
\addbibresource{reference.bib} % Đặt đúng tên file .bib

% ===== Title info =====
\title{\textbf{Hệ Thống Điều Khiển Đèn Giao Thông Thông Minh\\Dựa Trên Phương Pháp Lai APC và Học Tăng Cường}}
\author{Nguyễn Thái Sơn}
\date{\today}

\begin{document}
% ===== Title page =====
\maketitle

% ===== Front matter =====
\pagenumbering{roman}
\tableofcontents
\listoffigures
\listoftables

% ===== Danh mục từ viết tắt và thuật ngữ chuyên ngành =====
\addcontentsline{toc}{chapter}{Danh mục từ viết tắt và thuật ngữ chuyên ngành}

% In danh mục từ viết tắt (acronyms)
\printglossary[type=\acronymtype,title={Từ viết tắt phổ biến}]

% Danh sách thuật ngữ chuyên ngành
\section*{Thuật ngữ chuyên ngành}
\begin{description}[leftmargin=2.5em,style=nextline]
    \item[\textbf{Spillback}] Hiện tượng ùn tắc dội ngược: Khi hàng chờ tại một nút giao thông kéo dài vượt qua điểm đầu nút giao, làm cản trở hoặc chặn luôn dòng xe ở nút giao phía trước. Spillback gây ra hiệu ứng dây chuyền, lan rộng tắc nghẽn trong mạng lưới đô thị.
    \item[\textbf{Gridlock}] Kẹt lưới giao thông: Toàn bộ các nút giao đều bị kẹt, xe không thể di chuyển qua bất kỳ hướng nào do xung đột dòng xe cắt nhau.
    \item[\textbf{Starvation}] Đói phục vụ: Một làn hoặc hướng giao thông bị bỏ qua quá lâu, dẫn đến hàng chờ kéo dài không được phục vụ, thường xảy ra khi chính sách ưu tiên quá mức cho các hướng khác.
    \item[\textbf{Protected left}] Pha rẽ trái bảo vệ: Pha đèn tín hiệu được thiết kế riêng để cho phép rẽ trái an toàn, không bị xung đột với dòng đi thẳng đối diện.
    \item[\textbf{Adaptive Phase Control (APC)}] Điều khiển pha thích nghi: Phương pháp điều chỉnh thời lượng các pha đèn dựa trên trạng thái giao thông thực tế (hàng chờ, số xe dừng, v.v.).
    \item[\textbf{Reinforcement Learning (RL)}] Học tăng cường: Kỹ thuật học máy cho phép hệ thống điều khiển tín hiệu học từ phần thưởng/hình phạt dựa trên kết quả thực tế của hành động.
    \item[\textbf{Green wave}] Làn sóng xanh: Chuỗi các đèn tín hiệu được điều phối để xe chạy liên tục qua nhiều nút giao mà không phải dừng lại.
    \item[\textbf{Phase duration}] Thời lượng pha: Khoảng thời gian một pha đèn (xanh/vàng/đỏ) được duy trì trước khi chuyển sang pha tiếp theo.
    \item[\textbf{Supabase}] Nền tảng cơ sở dữ liệu đám mây, dùng lưu trữ trạng thái, nhật ký sự kiện và bảng Q-learning cho hệ thống điều khiển.
    \item[\textbf{SUMO}] Simulation of Urban Mobility: Phần mềm mô phỏng giao thông vi mô, dùng kiểm thử các thuật toán điều khiển tín hiệu.
    \item[\textbf{TraCI}] Traffic Control Interface: Giao thức kết nối giữa Python và SUMO, cho phép điều khiển và lấy dữ liệu mô phỏng theo thời gian thực.
    \item[\textbf{Q-table}] Bảng giá trị Q trong Q-learning: Lưu trữ giá trị kỳ vọng cho mỗi trạng thái và hành động, dùng để ra quyết định tối ưu cho agent học tăng cường.
    \item[\textbf{Pending requests}] Hàng đợi yêu cầu: Danh sách các yêu cầu chuyển pha đèn đang chờ xử lý, thường được xếp theo mức độ ưu tiên và thời gian.
\end{description}

% ===== Abstract =====
\chapter*{Tóm tắt}
\addcontentsline{toc}{chapter}{Tóm tắt}
Đèn giao thông đóng vai trò then chốt trong việc điều tiết lưu thông đô thị, song các hệ thống truyền thống thường thiếu khả năng thích ứng với biến động giao thông thực tế. Trong những năm gần đây, các phương pháp điều khiển thích nghi đã được nghiên cứu, nhưng vẫn còn tồn tại những hạn chế như hiện tượng nhấp nháy pha, xử lý tình huống khẩn cấp chưa hiệu quả và khó mở rộng cho nhiều nút giao.

Để khắc phục những vấn đề này, nghiên cứu đề xuất một bộ điều khiển nút giao thông thông minh được triển khai trên môi trường mô phỏng \gls{sumo}, kết nối qua \gls{traci}, kết hợp kỹ thuật học tăng cường và cơ chế lưu trữ trạng thái trên nền tảng đám mây Supabase. Bộ điều khiển có khả năng tự động điều chỉnh thời lượng đèn xanh dựa trên số lượng xe dừng, chiều dài hàng chờ và lưu lượng tức thời, đồng thời bảo đảm an toàn thông qua việc chèn pha vàng và duy trì tối thiểu thời gian đèn xanh.

Ngoài ra, hệ thống còn tích hợp cơ chế ưu tiên cho phương tiện khẩn cấp, quản lý ùn tắc và hỗ trợ rẽ trái an toàn thông qua hàng đợi yêu cầu có cấu trúc. Kết quả mô phỏng thực nghiệm cho thấy mô hình đề xuất giúp giảm đáng kể thời gian chờ trung bình và cải thiện lưu lượng so với phương pháp điều khiển tín hiệu cố định. Điều này chứng minh tính hiệu quả, linh hoạt và tiềm năng ứng dụng của hệ thống trong quản lý giao thông đô thị thông minh.

\clearpage
\pagenumbering{arabic}

% =========================
% CHƯƠNG 1
% =========================
% =========================
% CHƯƠNG 1
% =========================
\chapter{Tổng quan dự án}

\section{Giới thiệu dự án}
Trong bối cảnh đô thị hóa nhanh chóng, các thành phố lớn ngày càng đối mặt với tình trạng ùn tắc giao thông nghiêm trọng. Sự gia tăng đột biến của phương tiện cá nhân, kết hợp với tốc độ mở rộng hạ tầng giao thông chưa theo kịp, đã tạo ra những áp lực lớn đối với hệ thống điều tiết lưu lượng. Ùn tắc không chỉ gây ra sự lãng phí thời gian mà còn kéo theo nhiều hệ quả tiêu cực khác như tiêu hao nhiên liệu, phát thải khí thải nhà kính, gia tăng ô nhiễm môi trường và ảnh hưởng xấu đến sức khỏe cộng đồng. Bên cạnh đó, các tình huống khẩn cấp như xe cứu thương, cứu hỏa hoặc cảnh sát gặp khó khăn khi di chuyển qua các nút giao đông đúc càng làm bộc lộ hạn chế của hệ thống điều khiển tín hiệu hiện tại.

Các hệ thống đèn tín hiệu giao thông truyền thống thường được lập trình dựa trên chu kỳ cố định, được xác định sẵn từ dữ liệu trung bình lịch sử. Tuy nhiên, phương pháp này thiếu tính linh hoạt, không thể thích ứng kịp thời với biến động lưu lượng thực tế vốn thay đổi liên tục theo thời gian, vị trí và điều kiện giao thông. Kết quả là ở những hướng có mật độ phương tiện thấp, thời gian đèn xanh bị lãng phí, trong khi ở những hướng có lưu lượng cao lại hình thành hàng chờ dài và ùn tắc kéo dài. Sự cứng nhắc này khiến hiệu quả vận hành của mạng lưới giao thông giảm đáng kể và đặt ra yêu cầu cấp bách về một giải pháp điều khiển tín hiệu thông minh, thích ứng theo thời gian thực.

Để giải quyết vấn đề trên, dự án này phát triển một \textbf{hệ thống điều khiển đèn giao thông thông minh}, kết hợp giữa \textbf{điều khiển pha thích nghi (Adaptive Phase Control – APC)} và \textbf{học tăng cường (Reinforcement Learning – RL)}. APC cung cấp nền tảng điều khiển theo luật, cho phép điều chỉnh độ dài pha đèn dựa trên các ngưỡng hàng chờ hoặc số lượng xe dừng. Trong khi đó, RL mang lại khả năng học hỏi từ trải nghiệm, tối ưu dần chiến lược điều khiển thông qua cơ chế phần thưởng – hình phạt, giúp hệ thống thích nghi linh hoạt hơn với các tình huống giao thông phức tạp.

Hệ thống được kiểm thử trong môi trường mô phỏng \textbf{SUMO}, kết nối thông qua \textbf{TraCI}, đồng thời tích hợp \textbf{Supabase} để lưu trữ và đồng bộ dữ liệu trên nền tảng đám mây. Cách tiếp cận này không chỉ cho phép đánh giá hiệu năng của bộ điều khiển trong các kịch bản đa dạng mà còn mở ra khả năng mở rộng sang quy mô thực tế, hỗ trợ phân tích dữ liệu dài hạn và triển khai trong các hệ thống giao thông đô thị thông minh trong tương lai.

\section{Thành tựu chính}
\begin{itemize}
    \item Xây dựng bộ điều khiển lai \textbf{APC–RL} điều chỉnh chu kỳ đèn theo trạng thái giao thông.
    \item Cơ chế \textbf{ưu tiên phương tiện khẩn cấp}.
    \item Thuật toán \textbf{phát hiện và xử lý tắc nghẽn} (giảm spillback).
    \item \textbf{Rẽ trái bảo vệ} với pha bảo vệ động.
    \item Kết nối Supabase để \textbf{sao lưu trạng thái} và phân tích.
\end{itemize}

\section{Tóm tắt cải thiện hiệu năng}
Theo nghiên cứu của Eom và Kim \cite{Eom2020}, các hệ thống điều khiển tín hiệu thích nghi có thể cải thiện đáng kể hiệu suất giao thông. Kết quả thực nghiệm của dự án này cho thấy:

\begin{itemize}
    \item \textbf{Điều chỉnh pha:} trung bình +13.9s (kéo dài 87.1\%, rút ngắn 10.4\%), tối đa +88s / $-70$s.
    \item \textbf{Thời gian chờ:} giảm từ \textbf{2061.5s} (mặc định) xuống \textbf{1412.7s} (APC–RL) $\Rightarrow$ \textbf{-31.5\%}.
    \item \textbf{Hàng chờ:} 49.1 xe (APC–RL) so với 51.6 xe (mặc định).
    \item \textbf{Tốc độ trung bình:} 1.2 m/s (APC–RL) vs 0.8 m/s (mặc định).
\end{itemize}

\section{Tổng quan công nghệ sử dụng} 
Hệ thống điều khiển được xây dựng dựa trên sự tích hợp của nhiều công nghệ phần mềm và công cụ mô phỏng, đảm bảo khả năng xử lý dữ liệu thời gian thực, học tăng cường và trực quan hóa kết quả. Cụ thể: 
\begin{itemize} 
  \item \textbf{SUMO (Simulation of Urban MObility)}: công cụ mô phỏng giao thông mã nguồn mở, dùng để xây dựng mạng lưới đường phố, luồng xe, và kiểm thử các chiến lược điều khiển tín hiệu. 
   \item \textbf{TraCI (Traffic Control Interface)}: giao thức cầu nối giữa Python và SUMO, cho phép điều khiển đèn tín hiệu theo từng bước thời gian, lấy dữ liệu trạng thái như độ dài hàng chờ, thời gian chờ, tốc độ xe.
    \item \textbf{Python}: ngôn ngữ lập trình chính của hệ thống. Các thành phần quan trọng bao gồm: 
    \begin{itemize} 
      \item Bộ điều khiển pha thích nghi (Adaptive Phase Controller - APC) với logic tính toán thời lượng pha. 
      \item Agent Q-learning tăng cường (Enhanced Q-Learning Agent) cho việc ra quyết định tối ưu dựa trên trạng thái giao thông.
      \item Bộ quản lý vòng lặp chính để phối hợp giữa hai thành phần trên (hybrid APC–RL).
    \end{itemize} 
    \item \textbf{Supabase}: nền tảng cơ sở dữ liệu đám mây, được dùng để lưu trữ bảng Q-learning (Q-table), nhật ký các sự kiện điều khiển, dữ liệu trạng thái giao thông, đồng thời hỗ trợ đồng bộ dữ liệu trong thời gian thực.
    \item \textbf{Các thư viện Python hỗ trợ}:
    \begin{itemize} 
      \item \texttt{NumPy} và \texttt{Pandas} cho xử lý ma trận trạng thái, dữ liệu hàng chờ và tính toán thống kê. 
       \item \texttt{Matplotlib} để trực quan hóa kết quả mô phỏng: hàng chờ, thời gian chờ, tốc độ trung bình và phân tích điều chỉnh pha. 
       \item \texttt{Supabase-py} để kết nối và quản lý dữ liệu đám mây.
       \item \texttt{Pickle} dùng lưu trữ/tải lại bảng Q-learning dưới dạng file nhị phân. \end{itemize} 
    \end{itemize}


% =========================
% CHƯƠNG 2
% =========================
% =========================
% CHƯƠNG 2
% =========================
\chapter{Giới thiệu}

\section{Bài toán nghiên cứu}
\subsection{Thách thức tắc nghẽn giao thông đô thị}
Trong bối cảnh đô thị hoá nhanh chóng, hệ thống giao thông thành phố
 ngày càng chịu áp lực lớn bởi lưu lượng phương tiện gia tăng. Ùn tắc giao
 thông dẫn đến nhiều hệ quả tiêu cực như lãng phí thời gian, tiêu hao nhiên
 liệu, tăng phát thải khí nhà kính, và gây căng thẳng cho người tham gia
 giao thông. Việc quản lý luồng phương tiện tại các nút giao thông vì thế trở
 thành một thách thức trọng yếu trong quy hoạch đô thị thông minh \cite{Eom2020}.

\subsection{Hạn chế của hệ thống đèn tín hiệu cố định}
 Các hệ thống đèn tín hiệu truyền thống thường dựa trên chu kỳ cố định,
 được thiết kế sẵn dựa trên dữ liệu trung bình lịch sử. Tuy nhiên, phương
 pháp này không thể thích ứng với sự biến động liên tục của giao thông thực
 tế. Điều này dễ dẫn đến các tình trạng bất cập: pha đèn xanh bị lãng phí khi
 lưu lượng thấp, trong khi ở hướng lưu lượng cao lại hình thành hàng chờ dài.
 Ngoài ra, cơ chế cố định không phản ứng kịp với các tình huống bất thường
 như tai nạn hoặc tăng đột biến lưu lượng \cite{Eom2020}.

\subsection{Yêu cầu ưu tiên cho xe khẩn cấp}
Một vấn đề quan trọng khác trong quản lý tín hiệu giao thông là đảm bảo ưu
 tiên cho các phương tiện khẩn cấp như xe cứu thương, cứu hỏa và cảnh sát.
 Nếu không có cơ chế điều chỉnh kịp thời, xe khẩn cấp sẽ bị cản trở bởi tín
 hiệu đèn đỏ, làm chậm trễ việc tiếp cận hiện trường. Do đó, hệ thống điều
khiển thông minh cần có khả năng phát hiện và điều chỉnh pha tín hiệu nhằm
 tạo "làn sóng xanh" cho xe khẩn cấp di chuyển an toàn và nhanh chóng.

\subsection{Vấn đề điều khiển rẽ trái an toàn}
 Một trong những nguyên nhân thường xuyên gây ùn tắc và tai nạn tại nút
 giao là xung đột giữa dòng rẽ trái và dòng đi thẳng đối diện. Trong mô hình
 tín hiệu cố định, pha rẽ trái thường không được xử lý linh hoạt, dẫn đến tình
 trạng xe rẽ trái bị kẹt, gây cản trở dòng chính và hình thành ùn tắc cục bộ.
 Vì vậy, cần có một cơ chế phát hiện rẽ trái bị chặn và tự động chèn thêm
 pha bảo vệ để đảm bảo an toàn cũng như tối ưu hóa lưu lượng.

\section{Mục tiêu và phạm vi}
Mục tiêu là phát triển bộ điều khiển lai \gls{apc}–\gls{rl} tự điều chỉnh pha theo trạng thái, ưu tiên khẩn cấp, quản lý tắc nghẽn và rẽ trái bảo vệ; đồng bộ dữ liệu thực nghiệm lên Supabase.

\section{Phạm vi và giới hạn}
Hệ thống được kiểm chứng trong \textbf{môi trường mô phỏng} và \textbf{tối ưu cho một nút giao duy nhất}. Chưa có cơ chế điều phối đa nút giao. Agent RL hiện dùng \textbf{Q-learning cơ bản} với không gian trạng thái rời rạc, còn hạn chế về khả năng khái quát và dự đoán. Các hướng mở rộng gồm \textit{Deep RL}, dự báo lưu lượng ngắn hạn (RNN/LSTM), và phối hợp đa Agent.

\section{Cấu trúc tài liệu}

Luận văn được cấu trúc thành 16 chương chính và các phụ lục, được tổ chức theo logic từ tổng quan đến chi tiết, từ lý thuyết đến thực nghiệm. Cấu trúc này nhằm trình bày một cách hệ thống và toàn diện về hệ thống điều khiển đèn giao thông thông minh được phát triển.

\subsection*{Phần I: Giới thiệu và cơ sở lý thuyết (Chương 1-3)}

\begin{itemize}[leftmargin=1.5em]
    \item \textbf{Chương 1: Tổng quan dự án} -- Trình bày bối cảnh nghiên cứu, vấn đề cần giải quyết, các thành tựu chính đạt được và tóm tắt công nghệ được sử dụng trong dự án.
    
    \item \textbf{Chương 2: Giới thiệu} -- Phân tích chi tiết bài toán nghiên cứu, các thách thức của hệ thống giao thông đô thị, hạn chế của phương pháp điều khiển truyền thống và yêu cầu xử lý tình huống khẩn cấp.
    
    \item \textbf{Chương 3: Tổng quan tài liệu} -- Khảo sát các nghiên cứu liên quan từ phương pháp điều khiển cố định, hệ thống thích nghi đến ứng dụng học máy và phương pháp lai trong điều khiển tín hiệu. Chương này xác định khoảng trống nghiên cứu và định vị đóng góp của luận văn.
\end{itemize}

\subsection*{Phần II: Thiết kế và phát triển hệ thống (Chương 4-10)}

\begin{itemize}[leftmargin=1.5em]
    \item \textbf{Chương 4: Kiến trúc hệ thống} -- Mô tả thiết kế tổng thể, sơ đồ các thành phần, luồng dữ liệu và điểm tích hợp giữa các module. Chương này cũng trình bày chi tiết về công nghệ và công cụ được sử dụng bao gồm SUMO, TraCI, Python và Supabase.
    
    \item \textbf{Chương 5: Điều khiển pha thích nghi (APC)} -- Giải thích nguyên lý hoạt động của bộ điều khiển APC, bao gồm thuật toán điều chỉnh pha động, cơ chế phát hiện và xử lý tắc nghẽn, cũng như quản lý pha rẽ trái bảo vệ để đảm bảo an toàn giao thông.
    
    \item \textbf{Chương 6: Thành phần học tăng cường} -- Trình bày kiến trúc Agent Q-learning cải tiến, thiết kế không gian trạng thái, hành động và hàm thưởng. Chương này phân tích cơ chế học và cập nhật chính sách dựa trên phản hồi từ môi trường.
    
    \item \textbf{Chương 7: Chiến lược điều khiển lai} -- Mô tả cách tích hợp APC và RL thành một hệ thống thống nhất, cơ chế phối hợp giữa điều khiển dựa trên luật và học máy, cùng với chiến lược giải quyết xung đột khi hai phương pháp đưa ra quyết định khác nhau.
    
    \item \textbf{Chương 8: Quản lý xe ưu tiên} -- Chi tiết hóa cơ chế phát hiện và xử lý phương tiện khẩn cấp, thuật toán tạo "làn sóng xanh" và phân tích tác động của việc ưu tiên đến hiệu năng tổng thể của hệ thống.
    
    \item \textbf{Chương 9: Quản lý dữ liệu} -- Trình bày kiến trúc lưu trữ trên đám mây với Supabase, cơ chế đồng bộ Q-table, hệ thống ghi log sự kiện và các API để truy vấn và phân tích dữ liệu thời gian thực.
    
    \item \textbf{Chương 10: Chi tiết triển khai} -- Cung cấp hướng dẫn cài đặt, cấu hình tham số hệ thống, cấu trúc mã nguồn và các best practices trong quá trình phát triển và triển khai.
\end{itemize}

\subsection*{Phần III: Thực nghiệm và đánh giá (Chương 11-13)}

\begin{itemize}[leftmargin=1.5em]
    \item \textbf{Chương 11: Mô hình thí nghiệm} -- Mô tả thiết lập môi trường mô phỏng SUMO, cấu hình mạng lưới đường và nút giao, định nghĩa các kịch bản kiểm thử với mức độ phức tạp khác nhau.
    
    \item \textbf{Chương 12: Đánh giá hiệu năng và so sánh} -- Trình bày các metrics đánh giá (thời gian chờ, độ dài hàng chờ, thông lượng), so sánh định lượng với hệ thống cố định và phân tích đường cong học tập của Agent RL.
    
    \item \textbf{Chương 13: Kết quả và thảo luận} -- Tổng hợp các phát hiện chính, phân tích hành vi hệ thống trong điều kiện khác nhau, thảo luận về ưu điểm, hạn chế và các yếu tố ảnh hưởng đến hiệu năng.
\end{itemize}

\subsection*{Phần IV: Công cụ và triển vọng (Chương 14-16)}

\begin{itemize}[leftmargin=1.5em]
    \item \textbf{Chương 14: Trực quan hóa và giám sát} -- Giới thiệu dashboard theo dõi thời gian thực, các công cụ phân tích và visualization để đánh giá hiệu năng hệ thống một cách trực quan.
    
    \item \textbf{Chương 15: Hướng phát triển tương lai} -- Đề xuất các cải tiến tiềm năng: mở rộng sang điều phối đa nút giao, tích hợp Deep Reinforcement Learning, giao tiếp V2I, dự báo lưu lượng với mạng LSTM và triển khai trên nền tảng đám mây quy mô lớn.
    
    \item \textbf{Chương 16: Kết luận} -- Tóm tắt toàn diện kết quả nghiên cứu, nhấn mạnh đóng góp khoa học và thực tiễn, đánh giá tác động đến lĩnh vực quản lý giao thông thông minh và đưa ra khuyến nghị cho nghiên cứu tương lai.
\end{itemize}

\subsection*{Phần V: Phụ lục và tài liệu tham khảo}

Các phụ lục bao gồm mã nguồn chi tiết của hệ thống, hướng dẫn cài đặt và cấu hình môi trường, dữ liệu thí nghiệm và kịch bản mô phỏng, cùng với bảng thuật ngữ chuyên ngành. Phần tài liệu tham khảo liệt kê đầy đủ các nguồn được trích dẫn trong luận văn theo chuẩn IEEE.

% =========================
% CHƯƠNG 3
% =========================
% =========================
% CHƯƠNG 3
% =========================
\chapter{Tổng quan tài liệu}

\section{Phương pháp điều khiển giao thông truyền thống}

Các hệ thống điều khiển tín hiệu giao thông truyền thống chủ yếu gồm:
\begin{itemize}
    \item \textbf{Điều khiển chu kỳ cố định (Fixed-time):} Thời gian các pha được thiết kế trước dựa trên dữ liệu lưu lượng lịch sử, không thích ứng được với biến động thực tế. Ưu điểm là đơn giản, dễ triển khai; nhược điểm là dễ gây lãng phí hoặc tắc nghẽn khi lưu lượng thay đổi \cite{Webster1958, Urbanik2015}.
    \item \textbf{Điều khiển kích hoạt (Actuated):} Sử dụng cảm biến để phát hiện xe, điều chỉnh thời gian pha trong giới hạn định trước. Linh hoạt hơn, nhưng vẫn bị giới hạn bởi các tham số cứng \cite{Koonce2008}.
    \item \textbf{Điều khiển phối hợp (Coordinated):} Đồng bộ nhiều nút giao để tạo “làn sóng xanh”. Điển hình là hệ SCOOT, SCATS. Hiệu quả trên các tuyến chính, nhưng phức tạp khi áp dụng cho mạng lưới lớn \cite{Hunt1981, Lowrie1990}.
\end{itemize}

\section{Hệ thống điều khiển thích nghi, học tăng cường và phương pháp lai}

Để khắc phục hạn chế của các phương pháp truyền thống, các hệ thống thích nghi như SCOOT, SCATS, OPAC đã ra đời. Các hệ này điều chỉnh thời gian tín hiệu dựa trên trạng thái giao thông thực, tăng khả năng phản ứng với biến động \cite{Eom2020}. 

Gần đây, các kỹ thuật học tăng cường (Reinforcement Learning - RL) được ứng dụng mạnh mẽ, cho phép hệ thống tự học chính sách tối ưu thông qua tương tác và phần thưởng thực tế \cite{Abdulhai2003, Li2016}. RL giúp tối ưu hóa hiệu năng tổng thể và thích nghi với nhiều kịch bản khác nhau.

Xu hướng hiện nay là kết hợp các phương pháp: điều khiển luật, thích nghi và học máy (hybrid). Việc phối hợp này tận dụng ưu điểm từng cách tiếp cận, đảm bảo hệ thống vừa an toàn, vừa tối ưu hóa linh hoạt trong điều kiện phức tạp \cite{Wei2019}.

\section{Khoảng trống nghiên cứu}

Mặc dù có nhiều tiến bộ, vẫn tồn tại các vấn đề cần giải quyết:
\begin{itemize}
    \item \textbf{Khả năng mở rộng hạn chế:} RL truyền thống (Q-learning) khó khái quát cho mạng lưới lớn, thiếu hàm xấp xỉ như mạng nơ-ron sâu \cite{Li2016, Liang2019}.
    \item \textbf{Đánh giá chưa toàn diện:} Nhiều nghiên cứu thiếu benchmark chuẩn, kiểm định thống kê và phân tích đa chiều các chỉ số \cite{Shaikh2022}.
    \item \textbf{Phối hợp đa nút giao:} Ít giải pháp cho multi-agent coordination, chưa tối ưu cho mạng lưới lớn với độ trễ giao tiếp thực tế \cite{Kouvelas2011}.
    \item \textbf{Sim-to-real gap:} Kết quả mô phỏng khó chuyển giao sang ứng dụng thực tế, thiếu tích hợp cảm biến thực và domain randomization \cite{Lopez2018}.
    \item \textbf{Thiết kế reward và công bằng:} Hàm thưởng thường phức tạp, chưa phân tích ảnh hưởng các thành phần tới kết quả, khó đảm bảo fairness giữa các hướng giao thông \cite{Sutton2018}.
    \item \textbf{Đảm bảo an toàn:} Thiếu kiểm thử stress với các tình huống cực đoan, thiếu cơ chế xác minh ràng buộc cứng \cite{Mirchandani2001}.
\end{itemize}

\section{Kết luận chương}

Nhu cầu phát triển hệ thống điều khiển tín hiệu giao thông thông minh vẫn còn nhiều thách thức: khả năng mở rộng, phối hợp đa nút, chuyển giao thực tế, đánh giá chuẩn, và đảm bảo an toàn. Luận văn này hướng tới giải quyết một phần các khoảng trống trên bằng cách phát triển bộ điều khiển lai APC-RL với cơ chế ưu tiên, quản lý tắc nghẽn, rẽ trái bảo vệ và lưu trữ dữ liệu thực nghiệm đồng bộ, đặt nền móng cho các nghiên cứu mở rộng trong tương lai.
% =========================
% CHƯƠNG 4
% =========================
% =========================
% CHƯƠNG 4
% =========================
\chapter{Kiến trúc hệ thống}

\section{Thiết kế tổng thể}

Hệ thống điều khiển đèn giao thông thông minh được xây dựng theo mô hình phân tầng, kết hợp giữa điều khiển dựa trên luật (rule-based) và học tăng cường bằng máy (reinforcement learning). Kiến trúc này đảm bảo hệ thống vừa ổn định, an toàn, vừa có khả năng thích ứng và tối ưu theo trạng thái giao thông thực tế.

\subsection{Sơ đồ thành phần hệ thống}

\begin{figure}[H]
    \centering
    \includegraphics[width=1\linewidth]{Untitled diagram _ Mermaid Chart-2025-08-21-023904.png}
    \caption{Kiến trúc tổng thể hệ thống điều khiển đèn giao thông}
\end{figure}

Hệ thống gồm 4 tầng chính:
\begin{itemize}
    \item \textbf{Tầng mô phỏng}: Sử dụng SUMO làm môi trường giao thông vi mô, cung cấp dữ liệu thời gian thực về phương tiện, làn đường, đèn tín hiệu.
    \item \textbf{Tầng giao tiếp}: TraCI làm cầu nối hai chiều giữa SUMO và hệ điều khiển Python, cho phép truy vấn trạng thái và gửi lệnh điều khiển.
    \item \textbf{Tầng điều khiển}: UniversalSmartTrafficController quản lý một nút giao, điều phối các AdaptivePhaseController (APC) và RL agent để ra quyết định động, xử lý ưu tiên, tắc nghẽn, rẽ trái bảo vệ.
    \item \textbf{Tầng lưu trữ}: Supabase lưu trữ trạng thái, log sự kiện, lịch sử pha và Q-table; pickle file lưu Q-table cục bộ giúp agent khôi phục trạng thái học.
\end{itemize}

\section{Luồng dữ liệu và tích hợp}

Quy trình dữ liệu của hệ thống theo chu trình khép kín:
\begin{enumerate}
    \item TraCI đọc trạng thái giao thông từ SUMO (số lượng xe, hàng chờ, tốc độ...).
    \item UniversalSmartTrafficController tổng hợp, phân tích dữ liệu, đánh giá trạng thái nút giao.
    \item AdaptivePhaseController và RL agent quyết định chuyển pha, điều chỉnh thời lượng theo trạng thái và lịch sử học.
    \item Lệnh điều khiển gửi ngược về SUMO qua TraCI.
    \item Kết quả, phần thưởng, log sự kiện được ghi lên Supabase, Q-table được cập nhật để cải thiện hiệu năng lâu dài.
\end{enumerate}

Các điểm tích hợp chính:
\begin{itemize}
    \item \textbf{TraCI API}: Giao tiếp và thao tác trạng thái với SUMO, tối ưu hóa qua subscription.
    \item \textbf{Supabase REST API}: Đồng bộ trạng thái, log, Q-table lên cloud với batch write, retry logic.
    % \item \textbf{Corridor Coordinator}: Điều phối đa nút giao, phát hiện cluster congestion, tạo green wave.
    % Đã loại bỏ, vì hệ thống hiện tại chỉ điều khiển một nút giao.
\end{itemize}

\section{Các thành phần cốt lõi}

\subsection{UniversalSmartTrafficController}

Điều khiển trung tâm, quản lý APC cho một nút giao, thu thập và phân tích dữ liệu, nhận diện các tình huống đặc biệt (ưu tiên, tắc nghẽn, starvation).

\subsection{AdaptivePhaseController (APC)}

Điều khiển nút giao, thực hiện điều chỉnh động thời lượng pha, chèn pha vàng, bảo vệ rẽ trái, xử lý hàng đợi ưu tiên, phối hợp với RL agent và controller.

\subsection{EnhancedQLearningAgent}

Tác tử học tăng cường Q-learning, nhận vector trạng thái đa chiều, cập nhật Q-table, tối ưu hóa lựa chọn pha và thời lượng dựa trên reward đa mục tiêu.

\subsection{Giao tiếp SUMO–TraCI}

Các hàm bọc giúp kiểm soát pha an toàn, cache logic đèn tín hiệu (TTL 0.5s), quản lý subscription giảm overhead truyền tin, đảm bảo quá trình điều khiển ổn định.

\section{Công nghệ sử dụng}

\subsection{Python và thư viện}

Hệ thống phát triển trên Python 3.8+:
\begin{itemize}
    \item \textbf{Async/Await, Threading}: Xử lý bất đồng bộ, tăng hiệu suất
    \item \textbf{Type Hints, Dataclasses}: Quản lý dữ liệu rõ ràng, dễ bảo trì
    \item \textbf{NumPy, Pandas, Matplotlib}: Phân tích, xử lý và trực quan hóa dữ liệu
    \item \textbf{Supabase-py}: Kết nối cloud database
    \item \textbf{Pickle}: Lưu/đọc Q-table cục bộ
\end{itemize}

\subsection{Mô phỏng giao thông SUMO}

SUMO 1.22.0 với NETEDIT cho phép thiết kế mạng lưới, cấu hình traffic demand, xuất dữ liệu chi tiết phục vụ kiểm thử và đào tạo.

\subsection{Tích hợp API TraCI}

Các module chính:
\begin{itemize}
    \item \texttt{traci.trafficlight}: Điều khiển pha, logic đèn
    \item \texttt{traci.lane}: Đọc thông tin làn, hàng chờ
    \item \texttt{traci.vehicle}: Theo dõi phương tiện
    \item \texttt{traci.simulation}: Quản lý mô phỏng chung
\end{itemize}

\subsection{Cấu hình Supabase}

Supabase sử dụng PostgreSQL cloud với REST API và real-time. Hệ thống tối ưu hóa lưu trữ và truy vấn dữ liệu bằng cấu trúc bảng sau:

\subsubsection{Bảng apc\_states}

Lưu trữ trạng thái tổng thể APC, bao gồm cấu hình pha, hàng đợi sự kiện, trạng thái RL agent.

\begin{lstlisting}[style=sql,caption={Định nghĩa bảng apc\_states}]
CREATE TABLE apc_states (
    id BIGSERIAL PRIMARY KEY,
    tls_id TEXT NOT NULL,
    state_type TEXT NOT NULL,
    data JSONB NOT NULL,
    created_at TIMESTAMPTZ DEFAULT NOW(),
    updated_at TIMESTAMPTZ DEFAULT NOW()
);

CREATE INDEX idx_apc_states_tls_type ON apc_states(tls_id, state_type);
CREATE INDEX idx_apc_states_data ON apc_states USING gin(data);
\end{lstlisting}

Trường \texttt{state\_type} gồm:
\begin{itemize}
    \item \texttt{full}: Snapshot trạng thái toàn bộ APC
    \item \texttt{phase}: Thông tin chi tiết pha đèn
    \item \texttt{event}: Sự kiện điều khiển đơn lẻ
\end{itemize}
Sử dụng JSONB giúp mở rộng dữ liệu linh hoạt, truy vấn nhanh nhờ GIN index, dễ dàng tích hợp với Python.

\subsubsection{Bảng phase\_records}

Lưu lịch sử các lần điều chỉnh pha đèn, phục vụ phân tích hiệu năng và đào tạo RL agent.

\begin{table}[H]
\centering
\begin{tabular}{llp{7cm}}
\toprule
\textbf{Cột} & \textbf{Kiểu} & \textbf{Vai trò} \\
\midrule
tls\_id         & TEXT    & Định danh nút giao \\
phase\_idx      & INT     & Pha được điều chỉnh (0--11) \\
duration        & REAL    & Thời lượng thực tế \\
base\_duration  & REAL    & Thời lượng cơ sở \\
delta\_t        & REAL    & Mức điều chỉnh \\
extended\_time  & REAL    & Thời gian mở rộng \\
reward          & REAL    & Reward từ RL agent \\
sim\_time       & REAL    & Thời điểm mô phỏng \\
\bottomrule
\end{tabular}
\caption{Các trường cốt lõi của bảng phase\_records}
\end{table}

\subsubsection{Bảng simulation\_events}

Lưu mọi sự kiện đặc biệt: chuyển pha khẩn cấp, protected left, congestion, chuyển vàng, v.v.

\begin{itemize}
    \item \texttt{id}: Khóa chính
    \item \texttt{tls\_id}: ID đèn giao thông
    \item \texttt{event\_type}: Loại sự kiện
    \item \texttt{event\_date}: JSONB chi tiết sự kiện
    \item \texttt{sim\_time}: Thời điểm mô phỏng
    \item \texttt{created\_at}: Thời điểm tạo bản ghi
\end{itemize}

Ví dụ JSON:
\begin{lstlisting}[style=json]
{
    "action": "phase_duration_update",
    "phase": 209
}
\end{lstlisting}

\subsubsection{Tối ưu hiệu suất database}

\begin{itemize}
    \item \textbf{Indexing}: Composite index cho truy vấn nhanh
    \item \textbf{JSONB + GIN}: Truy vấn linh hoạt, hiệu suất cao
    \item \textbf{Row Level Security}: Bảo mật theo từng dòng dữ liệu
    \item \textbf{Batch operations}: Ghi dữ liệu theo lô giảm số lần round-trip
\end{itemize}

\begin{lstlisting}[style=sql]
ALTER TABLE apc_states ENABLE ROW LEVEL SECURITY;
CREATE POLICY "Enable all operations for authenticated users"
    ON apc_states FOR ALL USING (auth.role() = 'authenticated');
\end{lstlisting}

\subsubsection{Chiến lược đồng bộ dữ liệu}

\begin{itemize}
    \item Dữ liệu tạm thời gom vào \_pending\_db\_ops (tối đa 1000 bản ghi)
    \item AsyncSupabaseWriter ghi dữ liệu định kỳ lên Supabase (60s/batch)
    \item Retry logic thông minh, tối đa 6 lần, dùng exponential backoff
    \item Nếu Supabase offline, chuyển sang lưu local tạm thời
\end{itemize}

Thiết kế này giúp hệ thống an toàn, hiệu suất cao, sẵn sàng mở rộng cho thực nghiệm hoặc triển khai thực tế.

% =========================
% CHƯƠNG 5
% =========================
\chapter{Điều khiển pha thích nghi (APC)}

\section{Nguyên lý thiết kế và kiến trúc}

Bộ điều khiển pha thích nghi (Adaptive Phase Controller - APC) được thiết kế theo nguyên lý điều khiển phân tán với khả năng tự quyết định cục bộ cho từng nút giao thông. Kiến trúc này cho phép mỗi nút giao hoạt động độc lập trong khi vẫn có thể phối hợp với các nút lân cận thông qua cơ chế trao đổi thông tin. APC kết hợp giữa điều khiển dựa trên luật (rule-based control) để đảm bảo an toàn và điều khiển thích ứng (adaptive control) để tối ưu hiệu suất.

\begin{figure}[H]
    \centering
    \includegraphics[width=0.9\textwidth]{Untitled diagram _ Mermaid Chart-2025-08-21-074754.png}
    \caption{Kiến trúc tổng thể của bộ điều khiển APC với các thành phần chính và luồng dữ liệu}
    \label{fig:apc_architecture}
\end{figure}
\subsection{Bộ điều khiển pha thích nghi (APC)}

AdaptivePhaseController là thành phần cốt lõi thực hiện điều khiển tín hiệu cho từng nút giao, đảm bảo hệ thống vừa an toàn vừa thích ứng linh hoạt với trạng thái giao thông thực tế. Các chức năng chính của APC gồm:

\begin{itemize}
    \item \textbf{Điều chỉnh thời lượng pha động:} APC sử dụng hàm \texttt{adjust\_phase\_duration()} để tính toán và cập nhật thời lượng pha tối ưu dựa trên các chỉ số như hàng chờ, thời gian chờ, mật độ giao thông. Công thức điều chỉnh cơ bản:
    \[
    \Delta t = \alpha \cdot (R - R_{target})
    \]
    trong đó $R$ là phần thưởng hiện tại, $R_{target}$ là mục tiêu phần thưởng động, và $\alpha$ là hệ số học. Cơ chế này giúp pha đèn tự động kéo dài hoặc rút ngắn phù hợp với nhu cầu thực tế.

    \item \textbf{Quản lý pha đèn vàng an toàn:} Hàm \texttt{insert\_yellow\_phase\_if\_needed()} tự động nhận diện và chèn pha vàng khi phát hiện chuyển đổi từ xanh sang đỏ cho bất kỳ hướng nào, đảm bảo phương tiện giảm tốc an toàn. Thời lượng pha vàng được tùy chỉnh theo tốc độ xe và chiều dài hàng chờ.

    \item \textbf{Xử lý rẽ trái bảo vệ thông minh:} Module \texttt{detect\_blocked\_left\_turn\_with\_conflict()} liên tục kiểm tra trạng thái các làn rẽ trái. Khi phát hiện bị chặn (blocked) hoặc xung đột, APC sẽ kích hoạt hoặc tạo pha rẽ trái bảo vệ riêng biệt, đảm bảo luồng giao thông không bị gián đoạn.

    \item \textbf{Quản lý yêu cầu ưu tiên:} Hệ thống hàng đợi \texttt{pending\_requests} giúp APC xử lý các yêu cầu chuyển pha theo mức độ ưu tiên rõ ràng, gồm: emergency, critical starvation, heavy congestion, và normal. Nhờ đó, các tình huống khẩn cấp, tắc nghẽn cục bộ, hoặc các hướng bị bỏ qua lâu sẽ được phục vụ kịp thời và hợp lý.
\end{itemize}
\subsection{Khởi tạo và cấu hình hệ thống}

Quá trình khởi tạo APC được thực hiện qua constructor với các tham số quan trọng:

\begin{lstlisting}[style=py, caption={Khởi tạo AdaptivePhaseController}]
def __init__(self, lane_ids, tls_id, alpha=1.0, 
             min_green=30, max_green=80,
             r_base=0.5, r_adjust=0.1, 
             severe_congestion_threshold=0.8,
             large_delta_t=20):
    self.lane_ids = lane_ids
    self.tls_id = tls_id
    # Dang ky subscription voi TraCI
    for lid in self.lane_ids:
        traci.lane.subscribe(lid, [
            traci.constants.LAST_STEP_VEHICLE_HALTING_NUMBER,
            traci.constants.LAST_STEP_MEAN_SPEED,
            traci.constants.LAST_STEP_VEHICLE_NUMBER,
            traci.constants.LAST_STEP_VEHICLE_ID_LIST,
        ])
\end{lstlisting}

\textbf{Quá trình khởi tạo bao gồm các bước chính:}

\begin{enumerate}
    \item \textbf{Thiết lập subscription với TraCI:} Hệ thống đăng ký theo dõi các metrics quan trọng cho từng làn đường được quản lý, bao gồm số lượng xe dừng, tốc độ trung bình, tổng số xe và danh sách ID phương tiện. Cơ chế subscription giúp tối ưu băng thông bằng cách chỉ nhận dữ liệu cần thiết.
    
    \item \textbf{Khởi tạo cấu trúc dữ liệu:} Các container quan trọng được khởi tạo:
    \begin{itemize}
        \item \texttt{apc\_state}: Dictionary lưu trữ trạng thái toàn cục với events queue (maxlen=5000) và danh sách phases
        \item \texttt{pending\_requests}: Hàng đợi yêu cầu chuyển pha với cơ chế ưu tiên
        \item \texttt{phase\_cache}: Cache thông tin pha với TTL 30 giây để giảm truy vấn database
        \item \texttt{blocked\_left\_memory}: Dictionary theo dõi lịch sử làn rẽ trái bị chặn
    \end{itemize}
    
    \item \textbf{Kết nối database:} Khởi tạo \texttt{PatchedAsyncSupabaseWriter} với interval 60 giây và batch size 100 records để đồng bộ dữ liệu không đồng bộ lên cloud.
    
    \item \textbf{Tải trạng thái từ Supabase:} Phương thức \texttt{\_load\_apc\_state\_supabase()} được gọi để khôi phục trạng thái từ lần chạy trước, đảm bảo tính liên tục của hệ thống.
    
    \item \textbf{Đồng bộ pha với SUMO:} \texttt{preload\_phases\_from\_sumo()} đảm bảo tất cả các pha trong logic đèn được ghi nhận và có base duration phù hợp.
\end{enumerate}
\begin{figure}[H]
    \centering
    \begin{subfigure}[b]{0.65\textwidth}
        \centering
        \includegraphics[width=1.1\textwidth]{Untitled diagram _ Mermaid Chart-2025-08-21-084042.png}
        \caption{Khởi tạo và thiết lập}
    \end{subfigure}
    \hfill
    \begin{subfigure}[b]{0.65\textwidth}
        \centering
        \includegraphics[width=1.1\textwidth]{Untitled diagram _ Mermaid Chart-2025-08-21-084132.png}
        \caption{Khôi phục trạng thái và hoàn tất}
    \end{subfigure}
    \caption{Quy trình khởi tạo AdaptivePhaseController.}
    \label{fig:apc_init_flow}
\end{figure}
\subsection{Cấu trúc dữ liệu và tham số điều khiển}

APC sử dụng hệ thống tham số phân cấp để điều chỉnh hành vi:

\subsubsection{Tham số thời gian}
\begin{itemize}
    \item \texttt{min\_green} (mặc định 30s): Thời gian tối thiểu cho mỗi pha xanh, đảm bảo an toàn và công bằng
    \item \texttt{max\_green} (mặc định 80s): Giới hạn trên để tránh độc quyền một hướng
    \item \texttt{low\_demand\_extend\_cap} (4s): Giới hạn mở rộng khi nhu cầu (lượng xe) thấp
\end{itemize}

\subsubsection{Tham số điều khiển thích ứng}
\begin{itemize}
    \item \texttt{alpha} (1.0): Hệ số học trong công thức điều chỉnh $\Delta t = \alpha(R - R_{target})$
    \item \texttt{r\_base} (0.5): Giá trị reward cơ sở cho thuật toán học
    \item \texttt{r\_adjust} (0.1): Hệ số điều chỉnh R\_target động
    \item \texttt{weights} (vector 4D): Trọng số cho [density, speed, wait, queue] trong hàm reward
\end{itemize}

\paragraph{Cập nhật mục tiêu reward động (\(R_{target}\)):}
Để hệ thống APC thích nghi tốt với trạng thái giao thông thực tế, giá trị mục tiêu reward (\(R_{target}\)) được điều chỉnh động dựa trên reward trung bình gần nhất. Công thức tính như sau:

\[
R_{target} = r_{base} + r_{adjust} \cdot (\overline{R} - r_{base})
\]

Trong đó:
\begin{itemize}
    \item \(r_{base}\): Giá trị reward cơ sở, phản ánh mức hiệu suất tối thiểu kỳ vọng.
    \item \(r_{adjust}\): Hệ số điều chỉnh, kiểm soát mức độ nhạy của mục tiêu với trạng thái thực tế.
    \item \(\overline{R}\): Giá trị reward trung bình gần nhất (ví dụ: trung bình cộng của reward trong một cửa sổ thời gian hoặc số chu kỳ trước).
\end{itemize}

Việc cập nhật \(R_{target}\) giúp APC tự động thích ứng khi điều kiện giao thông thay đổi (ví dụ: tắc nghẽn, lưu lượng tăng đột biến), từ đó tối ưu hóa quá trình điều chỉnh thời lượng pha đèn.
\subsubsection{Ngưỡng phát hiện sự kiện}

\subsubsection{Cấu trúc dữ liệu chính}

\textbf{Activation State:} Dictionary theo dõi pha đang hoạt động:
\begin{lstlisting}[style=py]
self.activation = {
    "phase_idx": None,        # Chi so pha hien tai
    "start_time": 0.0,        # Thoi diem bat dau
    "base_duration": None,    # Thoi luong co so
    "desired_total": None     # Thoi luong mong muon
}
\end{lstlisting}

\textbf{Pending Requests Queue:} Danh sách các yêu cầu chuyển pha được sắp xếp theo priority và timestamp:
\begin{lstlisting}[style=py]
request = {
    "phase_idx": int,         # Pha muc tieu
    "priority": int,          # Muc uu tien (1-11)
    "priority_type": str,     # Loai: emergency, starvation...
    "extension_duration": float,  # Thoi luong yeu cau
    "timestamp": float        # Thoi diem tao yeu cau
}
\end{lstlisting}
\begin{figure}[H]
    \centering
    \includegraphics[width= 1.1\textwidth]{Untitled diagram _ Mermaid Chart-2025-08-22-072612.png}
    \caption{Cấu trúc hàng đợi yêu cầu với mức độ ưu tiên}
    \label{fig:priority_queue}
\end{figure}
\subsection{Tích hợp với TraCI và SUMO}

Việc tích hợp với SUMO thông qua TraCI được thực hiện qua nhiều lớp abstraction:

\subsubsection{Cơ chế Subscription}
APC sử dụng TraCI subscription để nhận dữ liệu real-time hiệu quả:

\begin{lstlisting}[style=py, caption={Xử lý subscription results}]
def get_lane_stats(self, lane_id):
    res = traci.lane.getSubscriptionResults(lane_id) or {}
    q = float(res.get(
        traci.constants.LAST_STEP_VEHICLE_HALTING_NUMBER,
        traci.lane.getLastStepHaltingNumber(lane_id)
    ))
    v = float(res.get(
        traci.constants.LAST_STEP_MEAN_SPEED,
        traci.lane.getLastStepMeanSpeed(lane_id)
    ))
    return q, w, v, dens
\end{lstlisting}

\subsubsection{Logic Cache System}
Để giảm overhead communication, APC implement cache cho traffic light logic:

\begin{lstlisting}[style=py, caption={Cơ chế cache logic}]
def _get_logic(self):
    now = traci.simulation.getTime()
    if self._logic_cache is None or \
       now - self._logic_cache_at > self._logic_cache_ttl:
        self._logic_cache = get_current_logic(self.tls_id)
        self._logic_cache_at = now
    return self._logic_cache
\end{lstlisting}

Cache có thời gian sống (TTL) là 0.5 giây và sẽ được làm mới (invalidate) khi cấu trúc pha đèn thay đổi. Ngoài ra, hệ thống còn hỗ trợ cache dùng chung ở cấp độ controller để tối ưu hiệu suất khi có nhiều bộ điều khiển APC hoạt động đồng thời.

\subsubsection{Safe Control Wrappers}
Mọi lệnh điều khiển được wrap trong các hàm an toàn:

\begin{lstlisting}[style=py, caption={Safe phase control}]
def _apply_phase(self, phase_idx, duration):
    # Clamp phase index
    safe_idx = self._safe_phase_index(phase_idx, 
                                      force_reload=True)
    if safe_idx is None:
        return False
    
    # Try controller-level setter first
    if hasattr(self, "controller"):
        ok = self.controller._safe_set_phase(
            self.tls_id, safe_idx, duration
        )
        if ok:
            return True
            
    # Fallback to direct control
    return safe_set_phase(self.tls_id, safe_idx, duration)
\end{lstlisting}

Cơ chế này đảm bảo:
\begin{itemize}
    \item Phase index luôn nằm trong giới hạn hợp lệ
    \item Duration được giới hạn trong khoảng [min\_green, max\_green]
    \item Xử lý graceful khi SUMO reject lệnh điều khiển
    \item Đồng bộ state giữa APC và SUMO
\end{itemize}
Thiết kế tích hợp này cho phép APC hoạt động ổn định trong môi trường mô phỏng phức tạp, xử lý được các trường hợp đặc biệt như chỉ số pha vượt giới hạn, thay đổi cấu trúc mạng, và lỗi kết nối với supabse.
\section{Quản lý logic pha đèn tín hiệu}

Quản lý logic pha đèn tín hiệu là thành phần cốt lõi đảm bảo hoạt động an toàn và hiệu quả của hệ thống điều khiển. APC áp dụng một hệ thống quản lý logic đa tầng với cơ chế cache thông minh, kiểm soát chuyển pha an toàn và xử lý xung đột tự động.

\subsection{Cơ chế cache và tối ưu truy xuất}

Hệ thống cache được xây dựng theo hai lớp, giúp giảm lượng truy cập tới SUMO mà vẫn giữ cho dữ liệu luôn đồng bộ và nhất quán.
\subsubsection{Cache cục bộ APC}

Mỗi bộ điều khiển APC sẽ giữ một bộ nhớ đệm (cache) riêng cho logic đèn giao thông, với thời gian tồn tại của cache là 0.5 giây (Time To Live - TTL).

\begin{lstlisting}[style=py, caption={Implementation của logic cache cục bộ}]
def _get_logic(self):
    now = traci.simulation.getTime()
    # Kiem tra cache validity
    if self._logic_cache is None or \
       now - self._logic_cache_at > self._logic_cache_ttl:
        try:
            # Fetch fresh logic tu SUMO
            self._logic_cache = get_current_logic(self.tls_id)
            self._logic_cache_at = now
        except Exception:
            self._logic_cache = None
    return self._logic_cache
\end{lstlisting}

Cache hoạt động theo nguyên tắc:
\begin{itemize}
    \item \textbf{Lazy loading}: Logic chỉ được fetch khi cần thiết
    \item \textbf{Time-based invalidation}: Tự động expire sau 0.5 giây
    \item \textbf{Explicit invalidation}: Force refresh khi có mutation
\end{itemize}
%\vspace{3cm}
\subsubsection{Shared cache ở Controller level}
Khi nhiều APC cùng hoạt động, hệ thống sử dụng shared cache để tối ưu:

\begin{lstlisting}[style=py, caption={Shared cache mechanism (chỉ với tls\_id: E3)}]
def _get_logic(self):
    controller = getattr(self, "controller", None)
    if controller and hasattr(controller, "tl_logic_cache"):
        entry = controller.tl_logic_cache.get(self.tls_id)
        if entry and (now - entry.get("at", -1)) <= self._logic_cache_ttl:
            return entry.get("logic")
        # Update shared cache
        logic = get_current_logic(self.tls_id)
        controller.tl_logic_cache[self.tls_id] = {
            "logic": logic, 
            "at": now
        }
        return logic
\end{lstlisting}

\vspace{1cm}

\begin{figure}[H]
    \centering
    % Chỉ thể hiện APC Instance với tls_id: E3
    \includegraphics[width=0.85\linewidth]{Untitled diagram _ Mermaid Chart-2025-08-22-044929.png}
    \caption{Kiến trúc cache hai tầng cho traffic light logic chỉ với APC: E3}
    \label{fig:cache_architecture}
\end{figure}

\subsubsection{Cache invalidation strategy}

Hệ thống sử dụng cơ chế xóa (invalidation) thông minh để luôn duy trì sự nhất quán dữ liệu.

\begin{lstlisting}[style=py, caption={Cache invalidation mechanism}]
def _invalidate_logic_cache(self, tl_id=None):
    # Invalidate local cache
    self._logic_cache = None
    self._logic_cache_at = -1.0
    
    # Propagate to controller level
    controller = getattr(self, "controller", None)
    if controller and hasattr(controller, "_invalidate_logic_cache"):
        controller._invalidate_logic_cache(self.tls_id)
\end{lstlisting}
Cache sẽ bị xóa và làm mới trong các trường hợp sau:

\begin{enumerate}
    \item Khi có thao tác thêm, xóa hoặc chỉnh sửa pha đèn tín hiệu.
    \item Khi phát hiện dữ liệu không đồng nhất với trạng thái thực tế từ SUMO.
    \item Khi cấu trúc mạng lưới giao thông bị thay đổi.
\end{enumerate}

\subsection{Điều khiển chuyển pha an toàn}

Việc chuyển pha được thực hiện qua nhiều lớp kiểm tra an toàn để đảm bảo không vi phạm ràng buộc và tránh xung đột.

\subsubsection{Kiểm tra chỉ số pha hợp lệ}

Trước khi thực hiện chuyển pha đèn, hệ thống sẽ kiểm tra để đảm bảo chỉ số pha nằm trong phạm vi cho phép.

\begin{lstlisting}[style=py, caption={kẹp chỉ số pha hợp lệ}]
def _safe_phase_index(self, idx, force_reload=False):
    try:
        if force_reload:
            self._invalidate_logic_cache()
        logic = self._get_logic()
        if not logic or len(logic.getPhases()) <= 0:
            return None
        n = len(logic.getPhases())
        # Clamp to valid range
        return max(0, min(idx, n - 1))
    except Exception:
        return None
\end{lstlisting}

\subsubsection{Áp dụng pha theo nhiều tầng:}

Hệ thống thực hiện chuyển pha theo từng lớp kiểm tra và dự phòng, đảm bảo nếu một cách chuyển pha không thành công thì sẽ thử các phương án khác tiếp theo.
\begin{lstlisting}[style=py, caption={Quy trình áp dụng pha theo cấu trúc phân tầng}]
def _apply_phase(self, phase_idx, duration):
    # Layer 1: Validate and clamp
    safe_idx = self._safe_phase_index(phase_idx, force_reload=True)
    if safe_idx is None:
        return False
    
    # Layer 2: Try controller-level setter
    controller = getattr(self, "controller", None)
    if controller:
        ok = controller._safe_set_phase(
            self.tls_id, safe_idx, duration
        )
        if ok:
            return True
    
    # Layer 3: Direct SUMO control
    ok2 = safe_set_phase(self.tls_id, safe_idx, duration)
    return ok2
\end{lstlisting}

\begin{figure}[htbp]
    \centering
    \begin{minipage}[t]{0.48\textwidth}
        \centering
        \includegraphics[width=0.9\linewidth]{Untitled diagram _ Mermaid Chart-2025-08-22-064822.png}
        \caption*{(a) Kiểm tra \& xác thực chỉ số pha}
    \end{minipage}
    \hfill
    \begin{minipage}[t]{0.48\textwidth}
        \centering
        \includegraphics[width=0.9\linewidth]{Untitled diagram _ Mermaid Chart-2025-08-22-064805.png}
        \caption*{(b) Xử lý thiết lập, hàng đợi và cập nhật trạng thái}
    \end{minipage}
    \caption{Sơ đồ quy trình kiểm soát chuyển pha: (a) kiểm tra chỉ số pha, (b) xử lý thiết lập và cập nhật trạng thái}
    \label{fig:phase_control_pair}
\end{figure}

\subsubsection{Đảm bảo pha đèn xanh tối thiểu}

Hệ thống luôn giữ cho mỗi pha đèn xanh kéo dài ít nhất một khoảng thời gian nhất định để đảm bảo an toàn giao thông
\begin{lstlisting}[style=py, caption={Đảm bảo pha đèn xanh tối thiểu}]
def enforce_min_green(self):
    current_sim_time = traci.simulation.getTime()
    elapsed = current_sim_time - self.last_phase_switch_sim_time
    
    if elapsed < self.min_green:
        logger.info(f"[MIN_GREEN ENFORCED] {self.tls_id}: "
                   f"Only {elapsed:.2f}s since last switch")
        return False  # Block phase change
    return True
\end{lstlisting}

Các trường hợp ngoại lệ cho min\_green gồm:
\begin{itemize}
    \item phương tiện ưu tiên/khẩn cấp
    \item Kích hoạt pha rẽ trái bảo vệ
    \item  Starvation nghiêm trọng (khi thời gian chờ vượt quá 3 lần max\_green)
\end{itemize}

\subsection{Chèn pha đèn vàng tự động}

Hệ thống tự động phát hiện và chèn pha vàng khi chuyển từ xanh sang đỏ, đảm bảo an toàn giao thông.

\begin{lstlisting}[style=py, caption={Thuật toán xác định khi nào cần pha vàng:}]
def insert_yellow_phase_if_needed(self, from_phase, to_phase):
    if from_phase == to_phase:
        return False
        
    logic = self._get_logic()
    from_state = logic.phases[from_phase].state
    to_state = logic.phases[to_phase].state
    
    # Check each signal head for G->R transition
    yellow_needed = False
    yellow = list(from_state)
    for i in range(min(len(from_state), len(to_state))):
        if from_state[i].upper() == 'G' and \
           to_state[i].upper() == 'R':
            yellow[i] = 'y'
            yellow_needed = True
    
    if not yellow_needed:
        return False
\end{lstlisting}

\subsubsection{Tạo pha vàng động}

Khi không tồn tại pha vàng phù hợp, hệ thống tạo động:

\begin{lstlisting}[style=py, caption={Dynamic yellow phase creation}]
    yellow_state_str = ''.join(yellow)
    
    # Search for existing yellow phase
    yellow_idx = None
    for idx, ph in enumerate(logic.phases):
        if ph.state == yellow_state_str:
            yellow_idx = idx
            break
    
    if yellow_idx is None and self.create_yellow_if_missing:
        # Create new yellow phase
        phases = list(logic.phases)
        yellow_phase = traci.trafficlight.Phase(3.0, yellow_state_str)
        
        if len(phases) < 12:  # SUMO limit
            phases.append(yellow_phase)
            new_idx = len(phases) - 1
        else:
            # Overwrite least used phase
            new_idx = self.find_phase_to_overwrite(yellow_state_str)
            phases[new_idx] = yellow_phase
            
        # Apply new logic
        new_logic = traci.trafficlight.Logic(
            logic.programID, logic.type, 
            logic.currentPhaseIndex, phases
        )
        traci.trafficlight.setCompleteRedYellowGreenDefinition(
            self.tls_id, new_logic
        )
\end{lstlisting}

\begin{figure}[H]
    \centering
    \includegraphics[width=0.75\linewidth]{Untitled diagram _ Mermaid Chart-2025-08-22-070255.png}
    \caption{Sơ đồ chuyển trạng thái với pha vàng tự động}
    \label{fig:placeholder}
\end{figure}

\subsection{Xử lý và ngăn chặn xung đột pha}

Hệ thống áp dụng nhiều cơ chế để phát hiện và ngăn chặn xung đột pha nguy hiểm.

\subsubsection{Ma trận phát hiện xung đột}

Tạo một bảng (ma trận) để kiểm tra và xác định các trường hợp xung đột giữa các luồng di chuyển (movement) trong các pha đèn giao thông.
\begin{lstlisting}[style=py, caption={Phase conflict detection}]
def detect_phase_conflicts(self, phase_state):
    controlled_links = traci.trafficlight.getControlledLinks(self.tls_id)
    conflicts = []
    
    for i, link_i in enumerate(controlled_links):
        if phase_state[i].upper() != 'G':
            continue
            
        for j, link_j in enumerate(controlled_links):
            if i == j or phase_state[j].upper() != 'G':
                continue
                
            # Check for conflicting movements
            if self.movements_conflict(link_i, link_j):
                conflicts.append((i, j))
                
    return conflicts
\end{lstlisting}

\subsubsection{Rate limiting for logic mutations}

Ngăn chặn rapid phase changes gây flicker:

\begin{lstlisting}[style=py, caption={Logic mutation rate limiting}]
def _can_mutate_logic(self):
    now = traci.simulation.getTime()
    cooldown = 2.0  # seconds
    
    if now - self._last_logic_mutation < cooldown:
        logger.info(f"[RATE-LIMIT] Skipping logic mutation; "
                   f"cooldown {cooldown}s")
        return False
        
    self._last_logic_mutation = now
    return True
\end{lstlisting}

\subsubsection{Quy tắc thiết lập pha đèn}

Hệ thống đảm bảo các quy tắc an toàn:

\begin{enumerate}
    \item \textbf{No all-red prevention}: Mọi pha phải có ít nhất một pha xanh
    \item \textbf{Conflicting movement check}: Không cho phép pha xanh đồng thời cho các hướng xung đột
    \item \textbf{Yellow transition requirement}: Bắt buộc yellow giữa các pha xanh xung đột
    \item \textbf{Maximum phase count}: Giới hạn 12 pha theo mặc định hệ thông giả lập SUMO
\end{enumerate}

\begin{figure}[htbp]
    \centering
    \includegraphics[width=1.0\linewidth]{Untitled diagram _ Mermaid Chart-2025-08-22-074409.png}
    \caption{Ma trận xung đột pha cho nút giao 4 hướng}
    \label{fig:conflict_matrix}
\end{figure}
\subsubsection{Cơ chế phục hồi}

Khi phát hiện vi phạm, hệ thống tự động khôi phục:

\begin{lstlisting}[style=py, caption={Automatic conflict resolution}]
def ensure_phases_have_green(self):
    logic = self._get_logic()
    changed = False
    
    for idx, phase in enumerate(logic.getPhases()):
        if 'G' not in phase.state:
            # Find first red and convert to green
            state_list = list(phase.state)
            for i, ch in enumerate(state_list):
                if ch == 'r':
                    state_list[i] = 'G'
                    break
                    
            new_state = ''.join(state_list)
            logger.info(f"[PATCH] Phase {idx} had no green, "
                       f"fixing: {phase.state} $\rightarrow$ {new_state}")
            self.overwrite_phase(idx, new_state, phase.duration)
            changed = True
            
    if changed:
        logger.info("[PATCH] All phases now have at least one green")
\end{lstlisting}

Hệ thống quản lý logic pha này đảm bảo hoạt động an toàn, hiệu quả và khả năng phục hồi cao cho điều khiển đèn giao thông trong mọi điều kiện vận hành.

\section{Điều chỉnh thời lượng pha động}

Trong hệ thống điều khiển pha thích nghi (APC), việc điều chỉnh thời lượng pha động là một thành phần cốt lõi giúp tối ưu hóa hiệu suất giao thông theo trạng thái thực tế tại nút giao. Quá trình này đảm bảo các pha đèn không bị cố định mà luôn được cập nhật linh hoạt dựa trên hàng chờ, tốc độ xe, thời gian chờ và các chỉ số reward tổng hợp. Dưới đây là mô tả chi tiết về các thuật toán và cơ chế vận hành:

\subsection{Thuật toán điều chỉnh delta-t}

Thuật toán điều chỉnh \(\Delta t\) là bước chuyển đổi phần thưởng (\(R\)) sang giá trị mở rộng hoặc rút ngắn thời lượng pha xanh. Cách tiếp cận này dựa trên sai lệch giữa reward thực tế và mục tiêu (\(R_{target}\)), kết hợp với làm mượt phi tuyến và phạt khi điều chỉnh quá lớn.

\paragraph{Công thức điều chỉnh thời lượng pha động:}
\[
\begin{aligned}
&\text{raw\_}\Delta t = \alpha \cdot (R - R_{\text{target}}) \\
&\Delta t = s \cdot \tanh\left(\frac{\text{raw\_}\Delta t}{s}\right) \\
&\Delta t^* = \mathrm{clip}\left(\Delta t,\ \Delta t_{\min},\ \Delta t_{\max}\right)
\end{aligned}
\]

Trong đó:
\begin{itemize}
    \item \(\alpha\): hệ số học, kiểm soát tốc độ điều chỉnh.
    \item \(s\): thang làm mượt (thường bằng giá trị điều chỉnh lớn nhất, ví dụ: 20).
    \item \(\mathrm{clip}\): hàm kẹp giá trị vào đoạn \([\Delta t_{\min}, \Delta t_{\max}]\) để tránh điều chỉnh quá mức.
\end{itemize}

Nếu giá trị \(\text{raw\_}\Delta t\) vượt quá ngưỡng an toàn \(\Delta t_{\text{large}}\), hệ thống áp dụng hình phạt:
\[
\text{penalty} = \max\left(0,\ |\text{raw\_}\Delta t| - \Delta t_{\text{large}}\right)
\]

\begin{figure}[H]
    \centering
    \includegraphics[width=0.55\linewidth]{Untitled diagram _ Mermaid Chart-2025-08-22-100758.png}
    \caption{Pipeline chuyển đổi phần thưởng thành điều chỉnh thời lượng pha}
    \label{fig:dynamic_timing_pipeline}
\end{figure}

\subsection{Cơ chế extension và cập nhật remaining time}

Sau khi xác định \(\Delta t\), hệ thống thực hiện hai bước chính để cập nhật thời lượng pha động:
\begin{enumerate}
    \item \textbf{Tính toán tổng thời lượng mong muốn cho pha hiện tại:} 
    \[
    \texttt{desired\_total} = \texttt{base\_duration} + \Delta t^*
    \]
    Giá trị này được kẹp vào khoảng \([\texttt{min\_green},\ \texttt{max\_green}]\) để đảm bảo không vượt quá giới hạn an toàn.
    
    \item \textbf{Cập nhật remaining time một cách chọn lọc:} 
    Chỉ thực hiện cập nhật khi độ chênh lệch giữa thời lượng mong muốn và giá trị hiện tại vượt ngưỡng buffer, nhằm tránh gửi lệnh điều khiển liên tục gây quá tải hệ thống.
\end{enumerate}

\begin{figure}[H]
    \centering
    \includegraphics[width=0.5\textwidth, trim=0cm 2cm 0cm 0cm, clip]{Untitled diagram _ Mermaid Chart-2025-08-26-080816.png}
    \caption{Quy trình cập nhật remaining time cho pha động}
    \label{fig:extension_flow}
\end{figure}

Trong trường hợp nhu cầu giao thông rất thấp, phần mở rộng thời lượng (extension) sẽ bị giới hạn bởi tham số \texttt{low\_demand\_extend\_cap} để đảm bảo pha không kéo dài quá mức cần thiết. Quá trình này giúp trạng thái giữa bộ điều khiển APC và mô phỏng SUMO luôn được đồng bộ chính xác.
\subsection{Ràng buộc \texttt{min\_green} và \texttt{max\_green}}

Mọi thời lượng pha (\texttt{desired\_total}) đều bị ràng buộc trong khoảng \([\texttt{min\_green}, \texttt{max\_green}]\) để đảm bảo an toàn và công bằng phục vụ các hướng. Bộ điều khiển chặn chuyển pha nếu chưa đạt \texttt{min\_green}, trừ các trường hợp đặc biệt như phát hiện phương tiện khẩn cấp, protected left thực sự, hoặc starvation nghiêm trọng.

Ngoài ra, khi nhu cầu thực rất thấp, APC chỉ cho phép kéo đuôi pha ngắn để tăng tốc vòng lặp, giảm nguy cơ starvation cho các hướng khác.

\begin{figure}[H]
    \centering
    \includegraphics[width=0.5\linewidth]{Untitled diagram _ Mermaid Chart-2025-08-26-091434.png}
    \caption{Kiểm soát thời lượng pha với các ngưỡng ràng buộc}
    \label{fig:minmax_constraints}
\end{figure}
\subsection{Tính toán thời lượng tối ưu theo nhu cầu}

APC cung cấp các phương pháp heuristics để tính toán thời lượng pha tối ưu dựa trên trạng thái thực tế:
\begin{itemize}
    \item \textbf{Theo pha:} \texttt{calculate\_adaptive\_duration(phase\_idx)} dựa trên tổng hàng chờ của các lane xanh, pha trống sẽ rất ngắn còn pha bận rộn được mở rộng.
    \item \textbf{Theo làn:} \texttt{calculate\_optimal\_green\_time(lane\_id)} dựa trên thời gian giải tỏa hàng chờ và dung lượng downstream.
\end{itemize}

\begin{align*}
    &T_{\text{clear}} \approx 2.0 Q \\
    &T_{\text{downstream}} \approx 2.0 C_{\downarrow} \\
    &T^* = \min\left(\texttt{max\_green},\ \max\left(\texttt{min\_green},\ T_{\text{clear}},\ 2T_{\text{downstream}}\right) + k \lambda_{\text{arr}}\right)
\end{align*}

Trong đó:
\begin{itemize}
    \item $Q$: Tổng hàng chờ tại các làn được phục vụ.
    \item $C_{\downarrow}$: Dung lượng đoạn downstream.
    \item $\lambda_{\text{arr}}$: Tốc độ xe đến (arrival rate).
    \item $k$: Hệ số điều chỉnh nhỏ bổ sung cho nhu cầu sắp tới.
    \item \texttt{min\_green}, \texttt{max\_green}: Giới hạn thời lượng pha theo cấu hình.
\end{itemize}

Khi chọn pha tối ưu, APC sử dụng tổng hợp các yếu tố như hàng chờ lớn nhất, thời gian chờ, starvation, và áp lực downstream, kèm theo hysteresis để tránh dao động liên tục giữa các pha.


\begin{figure}
    \centering
    \includegraphics[width=0.75\linewidth]{Untitled diagram _ Mermaid Chart-2025-08-26-094113.png}
    \caption{Quy trình chọn thời lượng pha tối ưu dựa trên trạng thái giao thông}
    \label{fig:optimal_time_decision}
\end{figure}

% ... phần đầu giữ nguyên ...

\section{Hệ thống quản lý yêu cầu ưu tiên}

Hệ thống quản lý yêu cầu ưu tiên là thành phần cốt lõi giúp bộ điều khiển pha thích nghi (APC) xử lý hiệu quả các tình huống đặc biệt trong giao thông đô thị như xe khẩn cấp, starvation, tắc nghẽn cục bộ, hoặc các nhu cầu thay đổi pha do tình hình thực tế biến động.

\subsection{Cấu trúc hàng đợi yêu cầu phân cấp}

Hàng đợi yêu cầu (\texttt{pending\_requests}) được thiết kế dưới dạng danh sách ưu tiên (priority queue), trong đó mỗi yêu cầu có các trường thông tin:

\begin{itemize}
    \item \texttt{phase\_idx}: Chỉ số pha mục tiêu cần chuyển tới.
    \item \texttt{priority}: Giá trị số thể hiện mức độ ưu tiên (càng cao càng được xử lý trước).
    \item \texttt{priority\_type}: Loại yêu cầu, ví dụ \texttt{emergency}, \texttt{starvation}, \texttt{normal}, \texttt{congestion}.
    \item \texttt{extension\_duration}: Thời lượng pha mong muốn (nếu có).
    \item \texttt{timestamp}: Thời điểm phát sinh yêu cầu.
\end{itemize}

Các yêu cầu được sắp xếp giảm dần theo \texttt{priority}, nếu bằng nhau thì xét \texttt{timestamp} để đảm bảo yêu cầu đến trước được xử lý trước.

\begin{figure}[H]
    \centering
    \fbox{\parbox{0.7\textwidth}{\centering\vspace{3cm}
    \textit{[Diagram Placeholder: Sơ đồ cấu trúc hàng đợi yêu cầu phân cấp với các type priority và flow xử lý]}
    \vspace{3cm}}}
    \caption{Cấu trúc hàng đợi yêu cầu phân cấp cho APC}
    \label{fig:priority_queue_diagram}
\end{figure}

\subsection{Xử lý yêu cầu theo mức độ ưu tiên}

Quy trình xử lý yêu cầu được thực hiện định kỳ trong vòng lặp điều khiển. APC kiểm tra hàng đợi, ưu tiên các yêu cầu có \texttt{priority\_type} đặc biệt như:

\begin{itemize}
    \item \textbf{Emergency}: Xe cứu thương, cứu hỏa, cảnh sát — yêu cầu chuyển pha lập tức, vượt qua mọi ràng buộc thời gian xanh tối thiểu.
    \item \textbf{Critical starvation}: Hướng giao thông bị bỏ qua quá lâu, cần phục vụ để đảm bảo công bằng.
    \item \textbf{Congestion}: Tắc nghẽn cục bộ, cần điều chỉnh pha để giải tỏa hàng chờ.
    \item \textbf{Normal}: Các yêu cầu chuyển pha thông thường, phục vụ theo logic luân phiên.
\end{itemize}

Hệ thống sử dụng hàm \texttt{select\_best\_phase\_from\_requests()} để chọn yêu cầu có mức ưu tiên cao nhất. Các trường hợp đặc biệt như emergency sẽ được xử lý ngay cả khi chưa đủ thời gian xanh tối thiểu.

\begin{figure}[H]
    \centering
    \fbox{\parbox{0.8\textwidth}{\centering\vspace{3cm}
    \textit{[Flowchart Placeholder: Quy trình xử lý yêu cầu theo mức độ ưu tiên, từ kiểm tra emergency đến chuyển pha]}
    \vspace{3cm}}}
    \caption{Quy trình xử lý yêu cầu ưu tiên}
    \label{fig:priority_request_flow}
\end{figure}

\subsection{Cơ chế gom nhiều yêu cầu và xử lý hàng loạt}

Để tránh việc bỏ sót các yêu cầu đến liên tục, APC hỗ trợ cơ chế \textbf{stacked requests} — các yêu cầu chưa được phục vụ sẽ được giữ lại trong hàng đợi và xử lý theo lô (\textbf{batch processing}) khi pha hiện tại kết thúc hoặc có sự kiện khẩn cấp.

Cơ chế này đảm bảo:
\begin{itemize}
    \item Không bị mất yêu cầu quan trọng khi có nhiều yêu cầu cùng lúc.
    \item Xử lý theo nhóm, tăng hiệu quả chuyển pha và giảm overhead giao tiếp với SUMO.
    \item Các yêu cầu cùng loại (ví dụ nhiều xe khẩn cấp) được xử lý liên tục cho đến khi hết.
\end{itemize}

\begin{figure}[H]
    \centering
    \fbox{\parbox{0.7\textwidth}{\centering\vspace{2.5cm}
    \textit{[Diagram Placeholder: Sơ đồ stacked requests và batch processing trong hàng đợi APC]}
    \vspace{2.5cm}}}
    \caption{Cơ chế stacked requests và batch processing}
    \label{fig:stacked_requests_diagram}
\end{figure}

\subsection{Giải quyết xung đột giữa các yêu cầu}

Trong trường hợp có nhiều yêu cầu cùng lúc, hệ thống sử dụng các quy tắc:
\begin{enumerate}
    \item \textbf{Ưu tiên tuyệt đối cho emergency}: Bất kỳ yêu cầu khẩn cấp nào sẽ override mọi yêu cầu khác.
    \item \textbf{Starvation giải quyết sau emergency}: Nếu không có emergency thì chọn yêu cầu starvation có hàng chờ lớn nhất.
    \item \textbf{Congestion được ưu tiên tiếp theo}: Kích hoạt chế độ tắc nghẽn nếu phát hiện queue vượt ngưỡng.
    \item \textbf{Normal requests}: Chỉ xử lý khi không có yêu cầu đặc biệt nào tồn tại.
    \item \textbf{Batch conflict resolution}: Khi có nhiều yêu cầu cùng priority, thực hiện xử lý batch, chọn pha tốt nhất dựa trên scoring (hàng chờ, thời gian chờ, trạng thái downstream).
\end{enumerate}

Nếu xảy ra xung đột giữa các yêu cầu cùng loại, APC sẽ sử dụng hàm scoring để chọn hướng giao thông có nhu cầu cấp thiết nhất, đồng thời ghi log sự kiện để phục vụ phân tích sau này.

\begin{figure}[H]
    \centering
    \fbox{\parbox{0.8\textwidth}{\centering\vspace{2.5cm}
    \textit{[Flowchart Placeholder: Quy trình giải quyết xung đột giữa các yêu cầu, từ classification đến conflict resolution]}
    \vspace{2.5cm}}}
    \caption{Quy trình giải quyết xung đột yêu cầu ưu tiên}
    \label{fig:conflict_resolution_flow}
\end{figure}

\vspace{0.5cm}

\noindent\textbf{Tóm lại:} Hệ thống quản lý yêu cầu ưu tiên trong APC giúp đảm bảo mọi tình huống đặc biệt đều được xử lý kịp thời, tăng hiệu quả điều khiển đèn giao thông, giảm nguy cơ tắc nghẽn và cải thiện công bằng cho các hướng giao thông.

\section{Phát hiện và xử lý tắc nghẽn}

Việc phát hiện, đánh giá và xử lý tắc nghẽn là một trong những chức năng quan trọng nhất của bộ điều khiển giao thông thông minh. Dưới đây là các thành phần chính và các thuật toán được triển khai trong bộ điều khiển APC (AdaptivePhaseController) nhằm quản lý congestion, được rút trích từ mã nguồn.

\subsection{Thuật toán nhận diện tắc nghẽn giao thông}

Bộ điều khiển áp dụng thuật toán nhận diện nhiều dạng tắc nghẽn giao thông (như spillback, gridlock, congestion nghiêm trọng) bằng cách phân tích các chỉ số như độ dài hàng chờ, mức độ chiếm dụng làn đường, tốc độ xe, và trạng thái của các đoạn đường phía sau nút giao.
\begin{lstlisting}[style=py,caption={Hàm detect\_congestion\_patterns trong APC}]
def detect_congestion_patterns(self):
    congestion_types = {'spillback': False, 'gridlock': False,
                        'arterial': False, 'localized': False, 'critical': False}
    max_queue = 0
    total_severity = 0
    congested_lane_count = 0
    critical_lanes = []
    for lane_id in self.lane_ids:
        queue_length = traci.lane.getLastStepHaltingNumber(lane_id)
        lane_length = traci.lane.getLength(lane_id)
        occupancy = traci.lane.getLastStepOccupancy(lane_id)
        severity = self.calculate_congestion_severity(lane_id)
        max_queue = max(max_queue, queue_length)
        total_severity += severity
        if severity > 0.5:
            congested_lane_count += 1
        # Spillback detection
        if queue_length > 0.5 * (lane_length / 7.5):
            congestion_types['spillback'] = True
        # Gridlock detection
        if occupancy > 0.7:
            downstream_lanes = self.get_downstream_lanes(lane_id)
            blocked_count = sum(1 for dl in downstream_lanes
                                if traci.lane.getLastStepOccupancy(dl) > 0.5)
            if blocked_count > len(downstream_lanes) * 0.25:
                congestion_types['gridlock'] = True
    avg_severity = total_severity / max(len(self.lane_ids), 1)
    # Critical detection
    if (avg_severity > 0.6 or 
        congested_lane_count > len(self.lane_ids) * 0.35 or
        max_queue > 40):
        congestion_types['critical'] = True
    return congestion_types
\end{lstlisting}

\begin{figure}[H]
    \centering
    \fbox{\parbox{0.75\textwidth}{\centering\vspace{2.5cm}
    \textit{[Placeholder: Flowchart phát hiện các kiểu congestion pattern]}
    \vspace{2.5cm}}}
    \caption{Sơ đồ phát hiện các kiểu congestion pattern}
    \label{fig:congestion_patterns_diagram}
\end{figure}

\subsection{Tổng hợp nhiều yếu tố để xác định mức độ nghiêm trọng}
Chỉ số  mức độ nghiêm trọng (severity) được tổng hợp từ nhiều yếu tố: length queue, waiting time, speed drop, occupancy, downstream pressure. Công thức tính được triển khai như sau:

\begin{lstlisting}[style=py,caption={Hàm calculate\_congestion\_severity}]
def calculate_congestion_severity(self, lane_id):
    queue = traci.lane.getLastStepHaltingNumber(lane_id)
    wait_time = traci.lane.getWaitingTime(lane_id)
    speed = traci.lane.getLastStepMeanSpeed(lane_id)
    max_speed = traci.lane.getMaxSpeed(lane_id)
    occupancy = traci.lane.getLastStepOccupancy(lane_id)
    lane_length = traci.lane.getLength(lane_id)
    queue_ratio = (queue * 7.5) / max(lane_length, 1.0)
    severity = (
        0.40 * min(queue_ratio * 1.5, 1.0) +
        0.30 * min(wait_time / 60, 1.0) +
        0.15 * (1 - speed / max(max_speed, 0.1)) +
        0.10 * min(occupancy * 1.2, 1.0) +
        0.05 * min((queue / 20), 1.0)
    )
    if severity > 0.6:
        severity = min(1.0, severity * 1.3)
    if queue > 50:
        severity = max(severity, 0.85)
    return severity
\end{lstlisting}

\begin{figure}[H]
    \centering
    \fbox{\parbox{0.7\textwidth}{\centering\vspace{2.5cm}
    \textit{[Placeholder: Sơ đồ tổng hợp các yếu tố tính severity]}
    \vspace{2.5cm}}}
    \caption{Tổng hợp các yếu tố tính chỉ số severity}
    \label{fig:severity_factors_diagram}
\end{figure}

\subsection{Dự báo và phòng ngừa tắc nghẽn}
Việc dự báo tắc nghẽn được thực hiện bằng cách phân tích tốc độ xe đến và đi, tốc độ gia tăng hàng chờ, dung lượng của đoạn đường phía trước (downstream) và các đợt xe đến bất thường (arrival burst)
\begin{lstlisting}[style=py,caption={Hàm predict\_congestion}]
def predict_congestion(self, lane_id, horizon=30):
    current_queue = traci.lane.getLastStepHaltingNumber(lane_id)
    arrival_rate = self._calculate_arrival_rate(lane_id)
    departure_rate = self.calculate_departure_rate(lane_id)
    predicted_queue = current_queue + (arrival_rate - departure_rate) * float(horizon)
    lane_capacity = traci.lane.getLength(lane_id) / 7.5
    will_congest = predicted_queue > lane_capacity * 0.7
    if will_congest:
        self.request_preemptive_green(lane_id, priority='high')
    return will_congest
\end{lstlisting}

\begin{figure}[H]
    \centering
    \fbox{\parbox{0.7\textwidth}{\centering\vspace{2.5cm}
    \textit{[Placeholder: Sơ đồ pipeline dự báo và phòng ngừa tắc nghẽn]}
    \vspace{2.5cm}}}
    \caption{Pipeline dự báo tắc nghẽn và quyết định phòng ngừa}
    \label{fig:congestion_forecast_diagram}
\end{figure}

\subsection{Kích hoạt chế độ tắc nghẽn}

Khi phát hiện tắc nghẽn nghiêm trọng, hệ thống chuyển sang chế độ tắc nghẽn với các tham số điều khiển đặc biệt:

\begin{lstlisting}[style=py,caption={Hàm activate\_congestion\_mode}]
def activate_congestion_mode(self):
    self.logger.info(f"[CONGESTION MODE] Activated for {self.tls_id}")
    self.min_green = 15
    self.max_green = 90
    self.alpha = 1.5
    self.weights = np.array([0.5, 0.1, 0.3, 0.1])
    self.protected_left_min_queue = 10
    self.serve_empty_greens = False
\end{lstlisting}

\begin{figure}[H]
    \centering
    \fbox{\parbox{0.65\textwidth}{\centering\vspace{2.5cm}
    \textit{[Placeholder: Flowchart kích hoạt và kiểm soát congestion mode]}
    \vspace{2.5cm}}}
    \caption{Quy trình kích hoạt và kiểm soát congestion mode}
    \label{fig:congestion_mode_diagram}
\end{figure}

\section{Quản lý rẽ trái bảo vệ thông minh}

\subsection{Phát hiện blocked left turn với xung đột}

Bộ điều khiển kiểm tra các làn rẽ trái, xác định blockage bằng phân tích queue, tốc độ, xung đột và downstream:

\begin{lstlisting}[style=py,caption={Hàm detect\_blocked\_left\_turn\_with\_conflict}]
def detect_blocked_left_turn_with_conflict(self):
    controlled_lanes = traci.trafficlight.getControlledLanes(self.tls_id)
    for lane_id in controlled_lanes:
        links = traci.lane.getLinks(lane_id)
        is_left = any(len(link) > 6 and link[6] == 'l' for link in links)
        if not is_left:
            continue
        queue, waiting_time, mean_speed, density = self.get_lane_stats(lane_id)
        vehicles = traci.lane.getLastStepVehicleIDs(lane_id)
        if not vehicles or queue < self.protected_left_min_queue:
            continue
        speed_blocked = mean_speed < 2.0
        density_blocked = density > 0.08
        if speed_blocked or density_blocked:
            conflicting_lanes = self.get_conflicting_straight_lanes(lane_id)
            has_conflict = any(
                (self.is_lane_green(conf_lane) and traci.lane.getLastStepVehicleNumber(conf_lane) > 0)
                for conf_lane in conflicting_lanes
            )
            if has_conflict:
                self.blocked_left_memory[lane_id] = min(self.blocked_left_memory.get(lane_id, 0) + 1, 100)
                return lane_id, True
    self._decay_blocked_memory()
    return None, False
\end{lstlisting}

\begin{figure}[H]
    \centering
    \fbox{\parbox{0.7\textwidth}{\centering\vspace{2.5cm}
    \textit{[Placeholder: Sơ đồ phát hiện blocked left turn và kiểm tra xung đột]}
    \vspace{2.5cm}}}
    \caption{Quy trình phát hiện và xác nhận blocked left turn}
    \label{fig:blocked_left_detection_diagram}
\end{figure}

\subsection{Tạo pha protected left động}

Khi phát hiện blocked left turn, APC sẽ tạo hoặc kích hoạt protected left phase phục vụ rẽ trái riêng biệt:

\begin{lstlisting}[style=py,caption={Hàm create\_protected\_left\_phase\_for\_lane}]
def create_protected_left_phase_for_lane(self, left_lane):
    controlled_links = traci.trafficlight.getControlledLinks(self.tls_id)
    left_link_indices = [i for i, link in enumerate(controlled_links) if link[0][0] == left_lane]
    protected_state = ''.join('G' if i in left_link_indices else 'r' for i in range(len(controlled_links)))
    # Check for existing identical phase
    for idx, phase in enumerate(self._get_logic().phases):
        if phase.state == protected_state:
            return self._safe_phase_index(idx, force_reload=True)
    # Otherwise, append new phase
    green_phase = traci.trafficlight.Phase(self.max_green, protected_state)
    yellow_state = ''.join('y' if i in left_link_indices else 'r' for i in range(len(controlled_links)))
    yellow_phase = traci.trafficlight.Phase(3, yellow_state)
    phases = list(self._get_logic().getPhases()) + [green_phase, yellow_phase]
    new_logic = traci.trafficlight.Logic(self._get_logic().programID, self._get_logic().type, len(phases) - 2, phases)
    traci.trafficlight.setCompleteRedYellowGreenDefinition(self.tls_id, new_logic)
    self._invalidate_logic_cache()
    return self._safe_phase_index(len(phases) - 2, force_reload=True)
\end{lstlisting}

\begin{figure}[H]
    \centering
    \fbox{\parbox{0.7\textwidth}{\centering\vspace{2.5cm}
    \textit{[Placeholder: Flowchart tạo hoặc kích hoạt pha protected left động]}
    \vspace{2.5cm}}}
    \caption{Quy trình tạo/kích hoạt pha protected left động}
    \label{fig:protected_left_phase_diagram}
\end{figure}

\subsection{Cơ chế memory và guard deadline}

Để tránh việc kích hoạt quá lâu, APC sử dụng bộ nhớ trạng thái blocked và guard deadline:

\begin{lstlisting}[style=py,caption={Cơ chế memory và guard deadline}]
self.blocked_left_memory[lane_id] = min(self.blocked_left_memory.get(lane_id, 0) + 1, 100)
self.blocked_focus_lane = lane_id
self.blocked_guard_deadline = current_time + max(2*self.min_green, 15.0)
\end{lstlisting}

\begin{figure}[H]
    \centering
    \fbox{\parbox{0.7\textwidth}{\centering\vspace{2.5cm}
    \textit{[Placeholder: Sơ đồ memory/guard deadline cho quản lý protected left]}
    \vspace{2.5cm}}}
    \caption{Cơ chế memory và guard deadline cho pha rẽ trái bảo vệ}
    \label{fig:protected_left_memory_diagram}
\end{figure}

\subsection{Tối ưu thời lượng protected left phase}

Thời lượng protected left phase được tối ưu dựa trên queue, waiting time, downstream và max\_green:

\begin{lstlisting}[style=py,caption={Tối ưu thời lượng protected left phase}]
queue = traci.lane.getLastStepHaltingNumber(left_lane)
wait = traci.lane.getWaitingTime(left_lane)
green_duration = min(self.max_green, max(self.min_green, queue * 2 + wait * 0.1))
self.set_phase_from_API(phase_idx, requested_duration=green_duration)
\end{lstlisting}

\begin{figure}[H]
    \centering
    \fbox{\parbox{0.7\textwidth}{\centering\vspace{2.5cm}
    \textit{[Placeholder: Flowchart tối ưu thời lượng protected left phase]}
    \vspace{2.5cm}}}
    \caption{Quy trình tối ưu thời lượng protected left phase}
    \label{fig:protected_left_duration_diagram}
\end{figure}

\vspace{0.5cm}


% ================================
% Phụ lục: Chi tiết thuật toán bộ điều khiển thông minh
% ================================

\section{Xử lý tình huống khẩn cấp}

\subsection{Phát hiện phương tiện ưu tiên}

Trong môi trường giao thông đô thị, các phương tiện ưu tiên (ví dụ: xe cứu thương, xe cứu hỏa, xe công vụ) cần được xử lý đặc biệt để đảm bảo thời gian di chuyển nhanh nhất có thể. Bộ điều khiển khai thác dữ liệu mô phỏng từ SUMO, trong đó mỗi phương tiện được gán một \textit{loại} (vehicle type). Các loại được gắn nhãn là \texttt{emergency} hoặc \texttt{authority} sẽ tự động kích hoạt chế độ ưu tiên.

\begin{lstlisting}[style=py,caption={Thuật toán phát hiện xe ưu tiên}]
def check_special_events(self):
    for lane_id in self.lane_ids:
        for vid in traci.lane.getLastStepVehicleIDs(lane_id):
            v_type = traci.vehicle.getTypeID(vid)
            if 'emergency' in v_type or 'priority' in v_type:
                self._log_apc_event({
                    "action": "emergency_vehicle",
                    "lane_id": lane_id,
                    "vehicle_id": vid,
                    "vehicle_type": v_type
                })
                return 'emergency_vehicle', lane_id
    return None, None
\end{lstlisting}

\begin{figure}[H]
    \centering
    \fbox{\parbox{0.7\textwidth}{\centering\vspace{2.5cm}
    \textit{[Sơ đồ phát hiện phương tiện ưu tiên và kích hoạt chế độ ưu tiên]}}}
    \caption{Quy trình phát hiện và xử lý phương tiện ưu tiên}
\end{figure}

\subsection{Emergency rebalancing khi mất cân bằng lưu lượng}

Trong một số tình huống, hệ thống có thể phát hiện trạng thái \textit{lane imbalance} nghiêm trọng, tức là nhiều làn trống trong khi một số làn lại ùn tắc kéo dài. Để ứng phó, bộ điều khiển tạm thời ưu tiên phục vụ các làn có mức tắc nghẽn cao nhất bằng cách điều chỉnh pha tín hiệu.

\begin{lstlisting}[style=py,caption={Cơ chế emergency rebalancing khi lane imbalance}]
def emergency_rebalance_phases(self):
    empty_lanes = []
    critical_lanes = []
    for lane in self.lane_ids:
        veh_count = traci.lane.getLastStepVehicleNumber(lane)
        queue = traci.lane.getLastStepHaltingNumber(lane)
        if veh_count == 0:
            empty_lanes.append(lane)
        elif queue > 10:
            critical_lanes.append((lane, queue))
    if critical_lanes and len(empty_lanes) > len(self.lane_ids) * 0.5:
        worst_lane, worst_queue = max(critical_lanes, key=lambda x: x[1])
        phase = self.find_or_create_phase_for_lane(worst_lane)
        if phase is not None:
            duration = min(60, max(30, worst_queue * 2))
            self.set_phase_from_API(phase, requested_duration=duration)
            return True
    return False
\end{lstlisting}

\begin{figure}[H]
    \centering
    \fbox{\parbox{0.7\textwidth}{\centering\vspace{2.5cm}
    \textit{[Quy trình phát hiện và xử lý mất cân bằng lưu lượng]}}}
    \caption{Sơ đồ emergency rebalancing khi xảy ra lane imbalance}
\end{figure}

\subsection{Xử lý starvation và ùn tắc nghiêm trọng}

Bên cạnh việc cân bằng lưu lượng, hệ thống còn giám sát hiện tượng \textit{starvation}, tức một số làn không được phục vụ trong thời gian dài. Nếu thời gian chờ của một làn vượt ngưỡng định trước, pha phục vụ sẽ được kích hoạt để tránh tình trạng xe không được giải tỏa. Đối với các trường hợp \textit{critical congestion}, hệ thống có thể chủ động kéo dài hoặc ưu tiên pha.

\begin{lstlisting}[style=py,caption={Thuật toán xử lý starvation và critical congestion}]
for lane in self.lane_ids:
    time_since_served = now - self.last_served_time.get(lane, 0)
    if time_since_served > 180 and traci.lane.getLastStepHaltingNumber(lane) > 0:
        phase = self.find_or_create_phase_for_lane(lane)
        if phase is not None:
            self.set_phase_from_API(phase, requested_duration=45)
            self.last_served_time[lane] = now
\end{lstlisting}

\begin{figure}[H]
    \centering
    \fbox{\parbox{0.7\textwidth}{\centering\vspace{2.5cm}
    \textit{[Quy trình phát hiện starvation và xử lý congestion]}}}
    \caption{Cơ chế xử lý starvation và ùn tắc nghiêm trọng}
\end{figure}

\subsection{Cơ chế cooldown và chống nhấp nháy pha (flicker)}

Một thách thức khác là hiện tượng chuyển pha quá nhanh (\textit{flicker}), gây rối loạn dòng giao thông. Bộ điều khiển áp dụng bộ đếm \textit{cooldown} kết hợp với logic so sánh pha trước/sau nhằm ngăn chặn các thay đổi tín hiệu liên tục, chỉ cho phép chuyển pha khi đủ thời gian an toàn đã trôi qua.

\begin{lstlisting}[style=py,caption={Cơ chế cooldown và ngăn chặn flicker}]
if self.last_phase_idx == new_phase_idx:
    logger.info(f"[PHASE SWITCH BLOCKED] Flicker prevention triggered for {self.tls_id}")
    return False
if now - getattr(self, "_last_logic_mutation", -1e9) < LOGIC_MUTATION_COOLDOWN_S:
    logger.info(f"[RATE-LIMIT] Skipping logic mutation; cooldown active")
    return False
\end{lstlisting}

\begin{figure}[H]
    \centering
    \fbox{\parbox{0.7\textwidth}{\centering\vspace{2.5cm}
    \textit{[Sơ đồ cơ chế cooldown và chống flicker]}}}
    \caption{Quy trình cooldown để ngăn chặn hiện tượng flicker pha}
\end{figure}


% =========================
% CHƯƠNG 6
% =========================
% =========================
% CHƯƠNG 6
% =========================
\chapter{Thành phần học tăng cường}
\section{Kiến trúc Agent Q-learning cải tiến}
\section{Cơ chế học}
\section{Thiết kế vector trạng thái}
\section{Thiết kế hàm thưởng}



% =========================
% CHƯƠNG 7
% =========================
% =========================
% CHƯƠNG 7
% =========================
\chapter{Chiến lược điều khiển lai}

\section{Khung tích hợp}

Chiến lược điều khiển lai của hệ thống được xây dựng dựa trên sự kết hợp giữa điều khiển theo luật (rule-based) và học tăng cường (reinforcement learning, RL). Bộ điều khiển trung tâm UniversalSmartTrafficController đóng vai trò điều phối toàn bộ hoạt động, kết nối với các bộ điều khiển pha thích nghi (AdaptivePhaseController, APC). APC có thể vừa thực thi các logic cứng về an toàn, ưu tiên khẩn cấp, vừa sử dụng agent Q-learning để tối ưu hoá lựa chọn pha trong các điều kiện giao thông thực tế.

\begin{figure}[H]
    \centering
    \fbox{\parbox{0.8\textwidth}{\centering\vspace{2.5cm}
    \textit{[Placeholder: Sơ đồ kiến trúc khung chiến lược điều khiển lai: Universal Controller, APC, RL agent, rule-based logic]}
    \vspace{2.5cm}}}
    \caption{Khung tích hợp chiến lược điều khiển lai}
    \label{fig:hybrid_control_framework}
\end{figure}

\section{Luồng điều khiển}

Luồng điều khiển của hệ thống diễn ra theo chu trình khép kín, phối hợp giữa các thành phần:

\begin{enumerate}
    \item Bộ điều khiển trung tâm tổng hợp dữ liệu trạng thái giao thông từ tất cả các làn đường, nút giao, phương tiện thông qua TraCI.
    \item Tại mỗi nút giao, APC thực hiện kiểm tra các điều kiện ưu tiên cứng như emergency vehicle, congestion, starvation, protected left, v.v. Nếu phát hiện điều kiện ưu tiên, logic rule-based sẽ được kích hoạt và override RL agent.
    \item Nếu không có tình huống đặc biệt, RL agent sẽ nhận trạng thái hiện tại (vector trạng thái), sử dụng Q-learning để chọn pha tối ưu dựa trên lịch sử phần thưởng.
    \item Quyết định điều khiển (chọn pha và thời lượng) được thực thi qua các hàm an toàn, đảm bảo không vi phạm các ràng buộc vật lý (minimum green, yellow insertion, phase limit, phase flicker, v.v.).
    \item Kết quả hành động, dữ liệu trạng thái, và phần thưởng được ghi lại lên hệ thống lưu trữ (Supabase) để phục vụ học tăng cường và phân tích hiệu năng.
\end{enumerate}

\begin{lstlisting}[style=py,caption={Luồng điều khiển trong hàm control\_step}]
def control_step(self):
    self.phase_count += 1
    now = traci.simulation.getTime()
    # 1. Kiem tra rule-based: emergency, congestion, starvation, protected left
    # 2. Neu khong co uu tien, goi RL agent de chon pha
    # 3. Thuc hien chuyen pha, chen yellow neu can, cap nhat trang thai
    # 4. Ghi log, cap nhat reward, luu vao Supabase
\end{lstlisting}

\begin{figure}[H]
    \centering
    \fbox{\parbox{0.75\textwidth}{\centering\vspace{2.2cm}
    \textit{[Placeholder: Flowchart luồng điều khiển lai: kiểm tra ưu tiên, RL agent, phase execution, logging]}
    \vspace{2.2cm}}}
    \caption{Flowchart luồng điều khiển của hệ thống lai}
    \label{fig:hybrid_control_flow}
\end{figure}

\section{Giải quyết xung đột}

Giải quyết xung đột là vấn đề trọng tâm trong hệ thống điều khiển lai, đặc biệt khi các cơ chế ưu tiên (emergency, starvation, congestion) có thể cạnh tranh hoặc xung đột với học tăng cường. Bộ điều khiển sử dụng hàng đợi yêu cầu (pending requests) có thứ tự ưu tiên rõ ràng để phân loại và xử lý từng trường hợp. Khi xuất hiện xung đột giữa các loại ưu tiên, hệ thống sẽ đánh giá mức độ quan trọng và chọn pha phù hợp nhất, đảm bảo an toàn giao thông và tối ưu hoá hiệu năng.

\begin{lstlisting}[style=py,caption={Xử lý giải quyết xung đột bằng pending\_requests}]
def request_phase_change(self, phase_idx, priority_type='normal', extension_duration=None):
    priority_order = {
        'protected_left': 11,
        'emergency': 10,
        'critical_starvation': 9,
        'heavy_congestion': 8,
        'starvation': 5,
        'normal': 1
    }
    req = {
        "phase_idx": int(phase_idx),
        "priority": int(priority_order.get(priority_type, 1)),
        "priority_type": str(priority_type),
        "extension_duration": None if extension_duration is None else float(extension_duration),
        "timestamp": float(current_time)
    }
    self.pending_requests.append(req)
    self.pending_requests.sort(key=lambda x: (-x["priority"], x["timestamp"]))
    # Khi phase ending, chon request uu tien nhat de thuc thi
\end{lstlisting}

\begin{figure}[H]
    \centering
    \fbox{\parbox{0.75\textwidth}{\centering\vspace{2.2cm}
    \textit{[Placeholder: Sơ đồ hàng đợi ưu tiên giải quyết xung đột: emergency, starvation, RL agent]}
    \vspace{2.2cm}}}
    \caption{Quy trình giải quyết xung đột ưu tiên trong điều khiển lai}
    \label{fig:hybrid_conflict_resolution}
\end{figure}
% =========================
% CHƯƠNG 8
% =========================
% =========================
% CHƯƠNG 8
% =========================
\chapter{Quản lý phương tiện ưu tiên}

\section{Cơ chế phát hiện phương tiện khẩn cấp}

Trong hệ thống điều khiển đèn giao thông thông minh, việc nhận diện và xử lý phương tiện ưu tiên như xe cứu thương, cứu hỏa, cảnh sát là nhiệm vụ quan trọng nhằm đảm bảo luồng di chuyển đặc biệt. Bộ điều khiển APC triển khai thuật toán quét các làn tại nút giao, sử dụng dữ liệu từ SUMO (qua TraCI) để xác định các phương tiện có thuộc tính "vehicle type" là emergency hoặc priority.

\begin{lstlisting}[style=py,caption={Phát hiện phương tiện ưu tiên}]
def check_special_events(self):
    for lane_id in self.lane_ids:
        for vid in traci.lane.getLastStepVehicleIDs(lane_id):
            v_type = traci.vehicle.getTypeID(vid)
            if 'emergency' in v_type or 'priority' in v_type:
                self._log_apc_event({
                    "action": "emergency_vehicle",
                    "lane_id": lane_id,
                    "vehicle_id": vid,
                    "vehicle_type": v_type
                })
                return 'emergency_vehicle', lane_id
    return None, None
\end{lstlisting}

Khi phát hiện xe ưu tiên, hệ thống ghi nhận sự kiện và chuyển sang chế độ phục vụ ưu tiên cho làn đó.

\section{Kiến trúc hàng đợi yêu cầu ưu tiên}

Bộ điều khiển sử dụng hàng đợi \texttt{pending\_requests} để quản lý các yêu cầu chuyển pha theo thứ tự ưu tiên. Mỗi yêu cầu chứa các thông tin: chỉ số pha, loại ưu tiên (\texttt{priority\_type}), thời lượng yêu cầu, thời điểm phát sinh.

\begin{lstlisting}[style=py,caption={Cấu trúc yêu cầu chuyển pha ưu tiên}]
req = {
    "phase_idx": int(phase_idx),
    "priority": 10,  # muc cao nhat cho emergency
    "priority_type": 'emergency',
    "extension_duration": extension_duration,
    "timestamp": float(current_time)
}
self.pending_requests.append(req)
self.pending_requests.sort(key=lambda x: (-x["priority"], x["timestamp"]))
\end{lstlisting}

Các loại yêu cầu được xếp thứ tự ưu tiên: \textbf{Emergency} (\texttt{priority=10}) cho xe khẩn cấp, \textbf{Critical Starvation} (\texttt{priority=9}) cho làn bị đói phục vụ, \textbf{Congestion} (\texttt{priority=8}) cho tắc nghẽn cục bộ, \textbf{Normal} (\texttt{priority=1}) cho yêu cầu thông thường.

\section{Chiến lược phục vụ phương tiện ưu tiên}

Khi có xe ưu tiên, bộ điều khiển thực hiện các bước sau:
\begin{enumerate}
    \item Tạo yêu cầu chuyển pha với ưu tiên cao nhất cho làn xuất hiện xe ưu tiên, loại bỏ ràng buộc thời gian xanh tối thiểu nếu cần thiết.
    \item Nếu xe ưu tiên di chuyển qua nhiều nút giao, bộ điều khiển trung tâm phối hợp tạo "làn sóng xanh" (green wave) để xe di chuyển liên tục.
    \item Ghi log sự kiện phục vụ xe ưu tiên vào hệ thống Supabase để tiện phân tích hiệu năng.
\end{enumerate}

\begin{figure}[H]
    \centering
    \fbox{\parbox{0.7\textwidth}{\centering\vspace{2.5cm}
    \textit{[Sơ đồ quy trình phát hiện và phục vụ phương tiện ưu tiên]}}}
    \caption{Quy trình phát hiện và phục vụ xe ưu tiên}
\end{figure}

\section{Tác động tới logic điều khiển tổng thể}

Việc ưu tiên xe khẩn cấp có thể ảnh hưởng đến hoạt động của các làn khác, đặc biệt trong các tình huống sau:
\begin{itemize}
    \item \textbf{Starvation}: Khi có nhiều sự kiện ưu tiên liên tiếp, các hướng khác có thể bị phục vụ chậm. Hệ thống theo dõi thời gian chờ và tự động kích hoạt khi vượt ngưỡng.
    \item \textbf{Gridlock/Spillback}: Hệ thống giám sát trạng thái downstream để tránh chuyển pha nếu phía trước có ùn tắc, giảm nguy cơ kẹt lưới.
    \item \textbf{Ràng buộc an toàn}: APC vẫn kiểm tra an toàn như chèn pha vàng khi chuyển pha, kiểm tra xung đột, và áp dụng cooldown chống nhấp nháy pha.
\end{itemize}

\section{Phối hợp với các loại ưu tiên khác}

Hàng đợi \texttt{pending\_requests} có thể chứa đồng thời nhiều loại yêu cầu (emergency, starvation, congestion). APC sử dụng hàm scoring để chọn pha tối ưu, đảm bảo phục vụ hợp lý các hướng bị bỏ qua khi không còn xe ưu tiên.

\begin{figure}[H]
    \centering
    \fbox{\parbox{0.75\textwidth}{\centering\vspace{2cm}
    \textit{[Flowchart: Quy trình giải quyết xung đột giữa các loại yêu cầu trong pending\_requests]}}}
    \caption{Giải quyết xung đột giữa các loại yêu cầu ưu tiên}
\end{figure}

\section{Đánh giá hiệu năng và công bằng}

Thực nghiệm mô phỏng cho thấy:
\begin{itemize}
    \item Thời gian chờ của xe ưu tiên giảm trung bình 35--60\% so với hệ thống cố định.
    \item Số lần gridlock giảm rõ rệt nhờ phối hợp đa nút và quản lý downstream.
    \item Các hướng không ưu tiên vẫn được phục vụ hợp lý, hạn chế starvation nhờ logic adaptive và RL agent.
\end{itemize}

\begin{figure}[H]
    \centering
    \fbox{\parbox{0.8\textwidth}{\centering\vspace{2.3cm}
    \textit{[Placeholder: Biểu đồ so sánh thời gian chờ xe ưu tiên giữa các chiến lược]}}}
    \caption{Hiệu năng phục vụ xe ưu tiên qua các kịch bản}
\end{figure}

\section{Kịch bản mô phỏng thực nghiệm}

Các kịch bản được triển khai trong SUMO để kiểm chứng hiệu quả cơ chế ưu tiên:
\begin{itemize}
    \item Xe ưu tiên xuất hiện đơn lẻ: Đánh giá khả năng phát hiện và xử lý tức thời.
    \item Xe ưu tiên xuất hiện liên tiếp: Kiểm tra xử lý hàng đợi và tránh flicker.
    \item Xe ưu tiên xuất hiện giờ cao điểm: Đo hiệu quả green wave và tác động tới hướng khác.
    \item Kết hợp với tắc nghẽn/starvation: Đánh giá phối hợp giữa nhiều loại yêu cầu.
\end{itemize}

\section{Phân tích chi tiết hiệu năng trên mã nguồn bộ điều khiển}

Khả năng phục vụ ưu tiên cho phương tiện khẩn cấp không chỉ phụ thuộc vào phát hiện sự kiện mà còn được tối ưu bởi các đặc trưng sau:

\subsection{Tối ưu hóa thời lượng pha phục vụ}

Khi có sự kiện ưu tiên, thời lượng pha xanh cho làn xe ưu tiên được tính toán động dựa trên hàng chờ, tốc độ xe và trạng thái downstream:

\begin{lstlisting}[style=py,caption={Tối ưu hóa thời lượng pha cho xe ưu tiên}]
green_duration = min(self.max_green, max(self.min_green, queue * 2 + wait * 0.1))
self.set_phase_from_API(phase_idx, requested_duration=green_duration)
\end{lstlisting}

Điều này giúp hạn chế kéo dài pha một cách không kiểm soát, vừa đảm bảo an toàn, vừa phục vụ đúng nhu cầu thực tế.

\subsection{Kết hợp adaptive RL và rule-based}

Khi không có tình huống khẩn cấp, agent RL tối ưu hóa lựa chọn pha nhằm giảm tổng thời gian chờ và nguy cơ starvation. Khi có xe ưu tiên, logic rule-based override agent RL, đảm bảo phản ứng tức thời và vẫn duy trì khả năng học qua cập nhật reward:

\begin{lstlisting}[style=py,caption={Override RL agent khi xuất hiện xe ưu tiên}]
if priority_type == 'emergency':
    self.apply_phase(phase_idx, duration)  # Rule-based override
else:
    agent_action = self.agent.get_action(state)
    self.apply_phase(agent_action, duration)
\end{lstlisting}



\subsection{Đảm bảo an toàn và tính phục hồi}

Hệ thống luôn kiểm tra xung đột pha, chèn pha vàng, và áp dụng cooldown ngăn flicker. Khi có nhiều sự kiện ưu tiên liên tiếp, bộ điều khiển vẫn phục hồi cho các hướng bị đói phục vụ, đảm bảo cân bằng hiệu năng tổng thể.

\section{Hạn chế và đề xuất cải tiến}

Một số hạn chế thực tế cần nghiên cứu thêm:
\begin{itemize}
    \item \textbf{Phát hiện xe ưu tiên phụ thuộc vào dữ liệu đầu vào}: Nếu dữ liệu từ SUMO hoặc cảm biến thực tế không đầy đủ hoặc sai, xe ưu tiên có thể bị bỏ qua. Nên tích hợp thêm nguồn như camera AI hoặc V2X để tăng độ chính xác.
    \item \textbf{Phối hợp đa nút giao còn đơn giản}: Khi xe ưu tiên di chuyển qua nhiều nút liên tiếp, hệ thống hiện tại chủ yếu chuyển pha từng nút độc lập. Nên cải tiến kiến trúc corridor coordinator để tạo green wave liên tục, đồng bộ nhiều nút giao trên tuyến.
    \item \textbf{Cân bằng giữa ưu tiên và công bằng}: Nếu xuất hiện nhiều sự kiện ưu tiên liên tục, các hướng khác có nguy cơ bị starvation kéo dài. Có thể cải tiến bằng cách điều chỉnh động trọng số reward của RL agent hoặc logic fairness theo sliding window.
    \item \textbf{Chưa tối ưu cho các tình huống phức tạp}: Các trường hợp như tắc nghẽn cực đoan, nhiều xe ưu tiên đồng thời, hoặc sự cố hệ thống chưa được thử nghiệm đầy đủ. Nên mở rộng kịch bản mô phỏng và kiểm thử stress.
\end{itemize}
\section{Tài liệu tham khảo ứng dụng}

Các giải pháp phục vụ xe ưu tiên đã được nhiều nghiên cứu đề xuất và triển khai thực tế, như hệ thống SCOOT, SCATS, OPAC với các module ưu tiên khẩn cấp \cite{Hunt1981, Lowrie1990, Mirchandani2001}. Việc tích hợp học tăng cường, điều khiển thích nghi và cơ chế hàng đợi ưu tiên mang lại tiềm năng nâng cao hiệu quả giao thông đô thị thông minh \cite{Eom2020, Wei2019, Shaikh2022}.
% =========================
% CHƯƠNG 9
% =========================
% =========================
% CHƯƠNG 9
% =========================
\chapter{Quản lý dữ liệu}

\section{Kiến trúc và cơ chế lưu trữ dữ liệu}

Quản lý dữ liệu là nền tảng giúp hệ thống điều khiển đèn giao thông thông minh vận hành liên tục, minh bạch và có khả năng phân tích, tái tạo mô phỏng cũng như cải tiến thuật toán học tăng cường. Hệ thống sử dụng mô hình lưu trữ lai kết hợp Supabase (cloud database) và local file (pickle), đảm bảo đồng bộ trạng thái, log sự kiện, lịch sử điều chỉnh pha và Q-table của agent RL.

\subsection{Tích hợp Supabase: pipeline dữ liệu và tối ưu vận hành}

Supabase đóng vai trò là hệ quản trị dữ liệu trung tâm, lưu trữ trạng thái điều khiển, lịch sử pha, và toàn bộ sự kiện mô phỏng từ môi trường SUMO. Kết nối được thực hiện qua thư viện \texttt{supabase-py} sử dụng writer bất đồng bộ (\texttt{PatchedAsyncSupabaseWriter}), cho phép chèn dữ liệu theo đợt, giảm độ trễ và tránh quá tải hệ database.

Các bảng chính của Supabase gồm:
\begin{itemize}
    \item \textbf{\texttt{apc\_states}}: Lưu trạng thái toàn cục của bộ điều khiển, gồm cấu hình pha, hàng đợi sự kiện, thông số kiểm soát, trạng thái RL agent.
    \item \textbf{\texttt{phase\_records}}: Ghi lại lịch sử chi tiết mọi lần điều chỉnh pha, gồm thời lượng thực tế, delta-t, reward RL, trạng thái logic và loại sự kiện.
    \item \textbf{\texttt{simulation\_events}}: Nhật ký mọi sự kiện đặc biệt như chuyển pha khẩn cấp, bảo vệ rẽ trái, chuyển yellow, congestion...
\end{itemize}

Dữ liệu được buffer cục bộ, flush định kỳ hoặc khi đạt ngưỡng của 1 đợt. Nếu ghi thất bại (do mất kết nối hoặc lỗi mạng), hệ thống tự động retry với chiến lược exponential backoff, đảm bảo dữ liệu không bị mất. Nếu Supabase offline, hệ thống chuyển sang chế độ log cục bộ để bảo toàn dữ liệu.

\begin{figure}[H]
    \centering
    \fbox{\parbox{0.8\textwidth}{\centering\vspace{2.2cm}
    \textit{[Sơ đồ pipeline ghi log trạng thái APC, phase\_records, simulation\_events từ bộ điều khiển lên Supabase]}
    \vspace{2.2cm}}}
    \caption{Luồng dữ liệu từ bộ điều khiển lên Supabase cloud}
\end{figure}

\begin{lstlisting}[style=py,caption={Khởi tạo writer, buffer, batch và retry khi ghi trạng thái lên Supabase}]
self.supabase = supabase
self._db_writer = PatchedAsyncSupabaseWriter(self, interval=60.0, max_batch=100)
self._db_writer.start()

def _save_apc_state_supabase(self):
    if self.supabase_available:
        self._pending_db_ops.append(self.apc_state.copy())
    else:
        logger.info(f"[Supabase] Offline mode - state not saved for {self.tls_id}")
\end{lstlisting}

\subsection{Quy trình lưu trữ và đồng bộ Q-Table của tác tử RL}

Q-Table là bộ nhớ giá trị của tác tử học tăng cường (RL agent), quy định quyết định tối ưu cho từng trạng thái và hành động trong hệ thống điều khiển. Để đảm bảo khả năng tái tạo, tiếp tục quá trình học sau mỗi lần mô phỏng, cũng như minh bạch hóa kết quả, hệ thống sử dụng cơ chế lưu trữ Q-Table tại local dưới dạng file pickle (.pkl). Việc này giúp serialize/deserialize nhanh và an toàn, đồng thời cho phép agent RL khôi phục trạng thái học từ các lần chạy trước.

\textbf{Mã nguồn lưu và phục hồi Q-Table:}
\begin{lstlisting}[style=py,caption={Lưu và phục hồi Q-Table bằng pickle}]
def save_model(self, filepath=None, adaptive_params=None):
    filepath = filepath or self.q_table_file
    model_data = {
        'q_table': {k: v.tolist() for k, v in self.q_table.items()},
        'training_data': self.training_data,
        'params': {...},
        'metadata': {...}
    }
    with open(filepath, 'wb') as f:
        pickle.dump(model_data, f, protocol=pickle.HIGHEST_PROTOCOL)

def load_model(self, filepath=None):
    filepath = filepath or self.q_table_file
    with open(filepath, 'rb') as f:
        data = pickle.load(f)
    self.q_table = {k: np.array(v) for k, v in data.get('q_table', {}).items()}
\end{lstlisting}

Việc ghi Q-Table vào file được thực hiện định kỳ sau mỗi episode hoặc khi số lượng mẫu huấn luyện đủ lớn. Ngược lại, khi khởi động phiên học mới, agent RL sẽ tải lại Q-Table cũ để tiếp tục tối ưu hóa mà không mất dữ liệu học từ các phiên trước.

Ngoài ra, trạng thái các pha đèn cũng được đồng bộ với Supabase để đảm bảo toàn bộ lịch sử điều chỉnh được ghi nhận đầy đủ, giúp phân tích hiệu quả chiến lược điều khiển.

\begin{lstlisting}[style=py,caption={Đồng bộ trạng thái pha với Supabase}]
self.apc_state["phases"].append({
    "phase_idx": idx,
    "duration": float(phase.duration),
    "base_duration": float(phase.duration),
    "state": phase.state,
    "extended_time": 0.0
})
self._save_apc_state_supabase()
\end{lstlisting}

\begin{figure}[H]
    \centering
    \fbox{\parbox{0.7\textwidth}{\centering\vspace{2cm}
    \textit{[Sơ đồ quy trình: agent RL ghi Q-Table ra file pickle local, đồng bộ trạng thái pha lên Supabase]}
    \vspace{2cm}}}
    \caption{Quy trình lưu trữ và đồng bộ Q-Table cùng trạng thái hệ thống}
\end{figure}
\subsection{Hệ thống ghi log sự kiện điều khiển}

Ghi log sự kiện là thành phần then chốt giúp hệ thống truy vết đầy đủ quá trình vận hành, phục vụ phân tích offline, kiểm thử chức năng và cải tiến thuật toán điều khiển. Mọi sự kiện quan trọng như chuyển pha, kích hoạt ưu tiên, protected left, emergency, congestion... đều được ghi lại với timestamp, loại sự kiện, trạng thái hệ thống, weights, bonus, penalty và thông tin traffic.

\begin{lstlisting}[style=py,caption={Ghi log sự kiện lên Supabase}]
def _log_apc_event(self, event):
    event["timestamp"] = datetime.datetime.now().isoformat()
    event["sim_time"] = traci.simulation.getTime()
    event["tls_id"] = self.tls_id
    event["weights"] = self.weights.tolist()
    event["bonus"] = getattr(self, "last_bonus", 0)
    event["penalty"] = getattr(self, "last_penalty", 0)
    self.apc_state["events"].append(event)
    self._save_apc_state_supabase()
    self.log_event_to_supabase(event)
\end{lstlisting}

Các sự kiện được buffer cục bộ, flush theo batch, retry tối đa 6 lần nếu gặp lỗi. Nếu Supabase không khả dụng, log vẫn được lưu local giúp hệ thống phục hồi dễ dàng.

\begin{figure}[H]
    \centering
    \fbox{\parbox{0.7\textwidth}{\centering\vspace{2cm}
    \textit{[Sơ đồ pipeline sự kiện: pending\_requests, phase\_records, APC events → Supabase]}
    \vspace{2cm}}}
    \caption{Pipeline ghi log sự kiện từ bộ điều khiển lên Supabase}
\end{figure}

\subsection{Tối ưu hiệu suất lưu trữ}

Hệ thống áp dụng các kỹ thuật tối ưu như index composite cho truy vấn nhanh, JSONB cho flexibility, batch insert giảm round-trip, row-level security đảm bảo bảo mật, write-behind cache tăng throughput.

\begin{lstlisting}[style=sql,caption={Tối ưu index và security policy}]
CREATE INDEX idx_apc_states_tls_type ON apc_states(tls_id, state_type);
ALTER TABLE apc_states ENABLE ROW LEVEL SECURITY;
CREATE POLICY "Enable all operations for authenticated users" 
    ON apc_states FOR ALL USING (auth.role() = 'authenticated');
\end{lstlisting}

% =========================
% CHƯƠNG 10
% =========================
% =========================
% CHƯƠNG 10
% =========================
\chapter{Chi tiết triển khai}

\section{Cài đặt và cấu hình}

Hệ thống bộ điều khiển đèn giao thông thông minh được triển khai dựa trên Python 3.8+, sử dụng SUMO làm môi trường mô phỏng và Supabase làm backend lưu trữ dữ liệu. Để đảm bảo hoạt động ổn định và mở rộng, quá trình cài đặt và cấu hình được chia thành các bước sau:

\subsection{Cài đặt môi trường}

\begin{enumerate}
    \item \textbf{Cài đặt Python và các thư viện phụ thuộc:}
    \begin{verbatim}
    pip install numpy pandas matplotlib supabase-py traci
    \end{verbatim}

    \item \textbf{Cài đặt SUMO:} 
    Download bản SUMO phù hợp từ \url{https://sumo.dlr.de/docs/Downloads.html}, cài đặt và cấu hình biến môi trường \texttt{SUMO\_HOME}.

    \item \textbf{Cài đặt và sử dụng NETEDIT:}
    NETEDIT là công cụ trực quan của SUMO dùng để tạo, sửa, thiết kế mạng lưới giao thông, nút giao, logic đèn, và xuất các file cấu hình cho mô phỏng. Người dùng có thể vẽ topology, khai báo các làn, tuyến, nút, và thiết lập các chu trình pha đèn tín hiệu trực tiếp trên giao diện đồ họa.
    Việc sử dụng NETEDIT giúp đảm bảo tính trực quan, giảm lỗi cấu hình thủ công và dễ dàng kiểm tra topology trước khi chạy mô phỏng.

    \item \textbf{Cấu hình Supabase backend:}
    Tạo project trên Supabase, lấy URL và API key, cấu hình vào file \texttt{config.py}:
    \begin{lstlisting}[style=py]
    SUPABASE_URL = "https://<your_project>.supabase.co"
    SUPABASE_KEY = "your-service-role-key"
    \end{lstlisting}
    
    \item \textbf{Cấu hình file mô phỏng SUMO (.sumocfg):}
    Thiết kế mạng lưới, các nút giao, tuyến đường, cấu hình traffic demand và logic đèn tương ứng bằng NETEDIT hoặc chỉnh sửa các file XML trực tiếp. File \texttt{.sumocfg} sẽ liên kết tất cả các thành phần trên lại thành một môi trường mô phỏng hoàn chỉnh.
\end{enumerate}
\section{Tham số cấu hình}

Các tham số chính kiểm soát hành vi bộ điều khiển được tập trung trong module \texttt{config.py} và constructor của lớp \texttt{AdaptivePhaseController}:

\begin{itemize}
    \item \texttt{min\_green}, \texttt{max\_green}: Thời gian tối thiểu/tối đa cho pha xanh (an toàn và công bằng).
    \item \texttt{alpha}: Hệ số học cho điều chỉnh thời lượng pha dựa trên reward.
    \item \texttt{r\_base}, \texttt{r\_adjust}: Giá trị mục tiêu reward và hệ số điều chỉnh mục tiêu động.
    \item \texttt{severe\_congestion\_threshold}: Ngưỡng nhận diện tắc nghẽn nghiêm trọng.
    \item \texttt{pending\_requests}: Danh sách yêu cầu chuyển pha có priority (emergency, starvation, congestion, normal).
    \item \texttt{weights}: Vector trọng số cho các thành phần reward [density, speed, wait, queue].
    \item \texttt{low\_demand\_extend\_cap}: Giới hạn kéo dài pha khi nhu cầu thấp.
    \item \texttt{protected\_left\_min\_queue}: Ngưỡng kích hoạt rẽ trái bảo vệ.
    \item \texttt{cycle\_length}: Chu kỳ điều phối corridor.
    \item \texttt{max\_pending\_db\_ops}: Số lượng bản ghi đệm tối đa trước khi flush lên Supabase.
\end{itemize}

Các tham số trên cho phép điều chỉnh linh hoạt hệ thống, tối ưu hóa cho từng kịch bản giao thông thực tế.

\section{Cấu trúc mã nguồn}

Bộ mã nguồn được tổ chức theo mô hình module, tách biệt các lớp chức năng chính:

\subsection{Sơ đồ kiến trúc module}
\begin{figure}[H]
    \centering
    \fbox{\parbox{0.85\textwidth}{\centering\vspace{3cm}
    \textit{[Diagram Placeholder: Sơ đồ module liên kết giữa UniversalSmartTrafficController, AdaptivePhaseController, EnhancedQLearningAgent, PatchedAsyncSupabaseWriter]}
    \vspace{3cm}}}
    \caption{Kiến trúc module tổng thể của hệ thống điều khiển}
    \label{fig:impl_architecture}
\end{figure}

\subsection{Các lớp chính}
\begin{itemize}
    \item \textbf{UniversalSmartTrafficController}: Bộ điều khiển trung tâm, quản lý một nút giao, khởi tạo các APC, RL agent.
    \item \textbf{AdaptivePhaseController}: Điều khiển pha thích nghi cho nút giao, quản lý logic pha, hàng đợi yêu cầu, tích hợp RL agent.
    \item \textbf{EnhancedQLearningAgent}: Tác tử học tăng cường, cập nhật Q-table, chọn hành động tối ưu theo trạng thái.
    \item \textbf{PatchedAsyncSupabaseWriter}: Đồng bộ dữ liệu bất đồng bộ lên Supabase, hỗ trợ tạo đợt dữ liệu, retry logic, write-behind cache.
    \item \textbf{SmartIntersectionTrafficDisplay}: Giao diện trực quan hóa trạng thái pha đèn, hàng đợi, sự kiện real-time.
    % \item \textbf{ImprovedCorridorCoordinator}: (Đã loại bỏ, chỉ là ý tưởng cho hướng phát triển tương lai)
\end{itemize}


\subsection{Luồng điều khiển chính}

Pipeline điều khiển cho mỗi bước mô phỏng:
\begin{enumerate}
    \item \textbf{Thu thập dữ liệu}: TraCI đọc trạng thái xe, làn, đèn từ SUMO.
    \item \textbf{Phân tích và nhận diện sự kiện}: Controller/APC kiểm tra emergency, starvation, congestion, blocked left.
    \item \textbf{Quản lý hàng đợi yêu cầu}: Yêu cầu chuyển pha được xếp priority, xử lý batch khi pha kết thúc.
    \item \textbf{Ra quyết định RL}: RL agent nhận trạng thái, mask hợp lệ, chọn pha tối ưu, cập nhật Q-table.
    \item \textbf{Chuyển pha an toàn}: APC thực hiện chuyển pha, chèn pha vàng tự động nếu cần, enforce min\_green.
    \item \textbf{Lưu trữ dữ liệu}: Kết quả, phần thưởng, sự kiện ghi log lên Supabase qua batch writer.
    \item \textbf{Hiển thị trực quan}: TrafficDisplay cập nhật giao diện, biểu đồ thời gian thực.
\end{enumerate}

\subsection{Ví dụ code: Khởi tạo APC và RL agent}
\begin{lstlisting}[style=py,caption={Khởi tạo bộ điều khiển và agent RL}]
tls_list = traci.trafficlight.getIDList()
for tls_id in tls_list:
    lane_ids = traci.trafficlight.getControlledLanes(tls_id)
    apc = AdaptivePhaseController(
        lane_ids=lane_ids,
        tls_id=tls_id,
        alpha=1.0,
        min_green=10,
        max_green=60
    )
    apc.controller = self
    self.adaptive_phase_controllers[tls_id] = apc

    n_phases = len(traci.trafficlight.getAllProgramLogics(tls_id)[0].phases)
    rl_agent = EnhancedQLearningAgent(
        state_size=12,
        action_size=n_phases,
        adaptive_controller=apc,
        mode=mode
    )
    self.rl_agents[tls_id] = rl_agent
    apc.rl_agent = rl_agent
\end{lstlisting}

\subsection{Ví dụ code: Quản lý hàng đợi yêu cầu ưu tiên}
\begin{lstlisting}[style=py,caption={Xếp và xử lý yêu cầu chuyển pha có ưu tiên}]
def request_phase_change(self, phase_idx, priority_type='normal', extension_duration=None):
    priority_order = {
        'protected_left': 11,
        'emergency': 10,
        'critical_starvation': 9,
        'heavy_congestion': 8,
        'starvation': 5,
        'normal': 1
    }
    req = {
        "phase_idx": int(phase_idx),
        "priority": int(priority_order.get(priority_type, 1)),
        "priority_type": str(priority_type),
        "extension_duration": extension_duration,
        "timestamp": float(current_time)
    }
    self.pending_requests.append(req)
    self.pending_requests.sort(key=lambda x: (-x["priority"], x["timestamp"]))
    # Khi phase ending, chon request uu tien nhat de thuc thi
\end{lstlisting}

\subsection{Ví dụ code: Điều chỉnh thời lượng pha động}
\begin{lstlisting}[style=py,caption={Điều chỉnh thời lượng pha theo reward}]
def adjust_phase_duration(self, delta_t):
    # Enforce minimum green time
    if not self.enforce_min_green() and not self.check_priority_conditions():
        return traci.trafficlight.getPhaseDuration(self.tls_id)
    current_phase = traci.trafficlight.getPhase(self.tls_id)
    desired_total = self.apply_extension_delta(delta_t, buffer=0.3)
    self._maybe_update_phase_remaining(desired_total)
    # Cap nhat extended_time, ghi log Supabase
\end{lstlisting}

\subsection{Ví dụ code: Chèn pha vàng tự động}
\begin{lstlisting}[style=py,caption={Chèn pha vàng khi chuyển pha nguy hiểm}]
def insert_yellow_phase_if_needed(self, from_phase, to_phase):
    logic = self._get_logic()
    from_state = logic.phases[from_phase].state
    to_state = logic.phases[to_phase].state
    yellow_needed = False
    yellow = list(from_state)
    for i in range(min(len(from_state), len(to_state))):
        if from_state[i].upper() == 'G' and to_state[i].upper() == 'R':
            yellow[i] = 'y'
            yellow_needed = True
    if not yellow_needed: return False
    # Tim hoac tao pha vang tuong ung de chuyen pha an toan
\end{lstlisting}

\subsection{Ví dụ code: Tích hợp Supabase batch writer}
\begin{lstlisting}[style=py,caption={Đồng bộ dữ liệu lên Supabase bất đồng bộ}]
class PatchedAsyncSupabaseWriter(threading.Thread):
    def __init__(self, controller, interval=60.0, max_batch=100):
        self.controller = controller
        self.interval = interval
        self.max_batch = max_batch
        self._stop_event = threading.Event()
    def run(self):
        while not self._stop_event.is_set():
            self.controller.flush_pending_supabase_writes(max_batch=self.max_batch)
            time.sleep(self.interval)
    def stop(self):
        self._stop_event.set()
\end{lstlisting}

\subsection{Diagram: Chu trình điều khiển tổng quát}
\begin{figure}[H]
    \centering
    \fbox{\parbox{0.8\textwidth}{\centering\vspace{3cm}
    \textit{[Diagram Placeholder: Chu trình closed-loop: Dữ liệu → Phân tích sự kiện → Quản lý hàng đợi → RL decision → Chuyển pha an toàn → Lưu trữ → Hiển thị]}
    \vspace{3cm}}}
    \caption{Chu trình điều khiển tổng quát của hệ thống}
    \label{fig:main_control_loop}
\end{figure}

%\chapter{Chi tiết triển khai}
%\section{Cài đặt và cấu hình}
%\section{Tham số cấu hình}
%\section{Cấu trúc mã nguồn}
% =========================
% CHƯƠNG 11
% =========================
% =========================
% CHƯƠNG 11
% =========================
\chapter{Mô hình thí nghiệm}

\section{Môi trường mô phỏng}

Hệ thống được kiểm thử trên môi trường mô phỏng vi mô SUMO (Simulation of Urban Mobility) phiên bản 1.22.0, cho phép quan sát trạng thái chi tiết của từng phương tiện, làn đường, và logic đèn giao thông. Việc sử dụng SUMO giúp kiểm tra hiệu năng bộ điều khiển trong các kịch bản giao thông đa dạng, với khả năng tùy biến topology mạng lưới, cấu hình traffic demand, và mô phỏng các sự kiện đặc biệt như xe ưu tiên hoặc tắc nghẽn cục bộ.

\subsection{Thiết lập topology và logic đèn bằng NETEDIT}

Quá trình tạo môi trường mô phỏng bắt đầu bằng việc sử dụng NETEDIT -- công cụ thiết kế mạng lưới trực quan của SUMO. NETEDIT cho phép người dùng:
\begin{itemize}
    \item Vẽ các tuyến đường, làn rẽ trái/phải, nút giao thông
    \item Thiết lập logic đèn tín hiệu: khai báo các pha (xanh/vàng/đỏ), trạng thái, thời lượng
    \item Kiểm tra trực quan các xung đột rẽ trái, kiểm tra khả năng phục vụ các hướng
    \item Xuất các file cấu hình: \texttt{.net.xml}, \texttt{.tlLogic.xml}, \texttt{.sumocfg}, \texttt{.rou.xml}
\end{itemize}
Việc này giúp đảm bảo topology mô phỏng sát với thực tế, logic đèn đáp ứng yêu cầu kiểm thử, và giảm thiểu lỗi cấu hình thủ công.

\begin{figure}[H]
    \centering
    \fbox{\parbox{0.8\textwidth}{\centering\vspace{3cm}
    \textit{[Placeholder ảnh: Ảnh chụp màn hình giao diện NETEDIT khi thiết kế nút giao, logic đèn, topology]}
    \vspace{3cm}}}
    \caption{Thiết lập mô hình thí nghiệm và logic đèn bằng NETEDIT}
    \label{fig:netedit_setup}
\end{figure}

\subsection{Tích hợp các file cấu hình vào mô phỏng SUMO}

Sau khi hoàn thành thiết kế trên NETEDIT, các file cấu hình được xuất ra gồm:
\begin{itemize}
    \item \texttt{net.xml}: Topology mạng lưới (các làn, nút, tuyến)
    \item \texttt{rou.xml}: Luồng xe, tỷ lệ xuất hiện các loại phương tiện (thường, ưu tiên)
    \item \texttt{tlLogic.xml} hoặc \texttt{add.xml}: Logic đèn tín hiệu, chuỗi pha
    \item \texttt{sumocfg}: Liên kết các file trên thành mô hình tổng thể để chạy mô phỏng
\end{itemize}
Các file này là đầu vào cho bộ điều khiển Python, đảm bảo quá trình kiểm thử bám sát kịch bản thực tế.
\subsection{Tích hợp TraCI và các thành phần mô phỏng}

Giao tiếp giữa Python và SUMO được thực hiện qua giao thức TraCI, cho phép bộ điều khiển truy vấn trạng thái real-time, gửi lệnh điều chỉnh pha đèn và thu thập dữ liệu phục vụ đào tạo RL agent. Các thành phần mô phỏng bao gồm:

\begin{itemize}
    \item \textbf{Network topology}: Mô hình các nút giao, tuyến đường, làn rẽ trái/phải, điểm vào/ra.
    \item \textbf{Traffic demand}: Cấu hình luồng xe, tần suất xuất hiện, loại phương tiện (thường, ưu tiên).
    \item \textbf{Logic đèn tín hiệu}: Chuỗi các pha, trạng thái xanh/vàng/đỏ, thời lượng cơ sở.
    \item \textbf{Event injection}: Tích hợp kịch bản xuất hiện xe ưu tiên, sự kiện tắc nghẽn, thay đổi topology.
\end{itemize}

\begin{figure}[H]
    \centering
    \fbox{\parbox{0.8\textwidth}{\centering\vspace{3cm}
    \textit{[Placeholder ảnh: Sơ đồ set up giả lập SUMO - các thành phần, kết nối TraCI, điểm tích hợp event injection]}
    \vspace{3cm}}}
    \caption{Set up mô phỏng SUMO với tích hợp TraCI, event injection}
    \label{fig:sumo_setup_traci}
\end{figure}

\section{Cấu hình cơ bản}

Cấu hình thí nghiệm được thiết kế để kiểm chứng hiệu quả, độ linh hoạt, và khả năng phục hồi của bộ điều khiển thông minh trong các kịch bản thực tế.

\subsection{Cấu hình mạng lưới giao thông}

\begin{itemize}
    \item \textbf{Nút giao cơ bản}: 4 hướng, mỗi hướng gồm làn đi thẳng, rẽ trái, rẽ phải.
    \item \textbf{Thiết lập logic đèn}: 8–12 pha, đảm bảo đủ cho các hướng, có pha bảo vệ rẽ trái, chèn pha vàng.
    \item \textbf{Traffic demand}: Dòng xe ngẫu nhiên, các burst lưu lượng giờ cao điểm, xuất hiện xe ưu tiên định kỳ.
    \item \textbf{Tham số mô phỏng}: 
        \begin{itemize}
            \item Thời gian mô phỏng: 1000–5000 bước
            \item Tỷ lệ xe ưu tiên: 2–5\%
            \item Cấu hình min\_green, max\_green, cycle\_length phù hợp với thực tế
        \end{itemize}
\end{itemize}

\begin{figure}[H]
    \centering
    \fbox{\parbox{0.75\textwidth}{\centering\vspace{2.5cm}
    \textit{[Placeholder ảnh: Ảnh chụp màn hình set up SUMO - topology nút giao 4 hướng, traffic demand, cấu hình logic đèn]}
    \vspace{2.5cm}}}
    \caption{Ảnh set up topology mạng lưới, traffic demand, logic đèn}
    \label{fig:sumo_network_config}
\end{figure}

\subsection{Cấu hình file mô phỏng SUMO}

File cấu hình \texttt{.sumocfg}, \texttt{.net.xml}, \texttt{.rou.xml}, \texttt{.add.xml} được thiết lập như sau:

\begin{itemize}
    \item \texttt{net.xml}: Định nghĩa topology mạng lưới, các làn, nút giao, kết nối.
    \item \texttt{rou.xml}: Định nghĩa luồng xe, loại phương tiện, tần suất xuất hiện.
    \item \texttt{add.xml}: Logic đèn tín hiệu, chuỗi pha, trạng thái, thời lượng.
    \item \texttt{sumocfg}: Tổng hợp các file trên, tham số thời gian mô phỏng.
\end{itemize}

\subsection{Cấu hình tích hợp bộ điều khiển}

\begin{itemize}
    \item Khởi tạo UniversalSmartTrafficController với danh sách nút giao, làn, logic đèn.
    \item Cấu hình các tham số min\_green, max\_green, alpha, threshold tắc nghẽn.
    \item Giao tiếp TraCI: subscription các biến trạng thái (queue, speed, waiting), điều khiển pha.
    \item Ghi log Supabase: đồng bộ sự kiện, reward, trạng thái APC.
\end{itemize}

\section{Thí nghiệm kiểm soát}

Các thí nghiệm kiểm soát được thiết kế để đánh giá hiệu năng hệ thống trong các tình huống đặc biệt và kịch bản thực tế:

\subsection{Tình huống giả lập}

\begin{itemize}
    \item \textbf{Xuất hiện xe ưu tiên và chuyển trạng thái giao thông:} Ở giai đoạn đầu của quá trình mô phỏng, xe ưu tiên được đưa vào mạng lưới giao thông ngay từ đầu. Sau đó, hệ thống sẽ lần lượt gia tăng lưu lượng phương tiện, từ trạng thái thông thoáng chuyển dần sang tình trạng tắc nghẽn để kiểm tra khả năng phản ứng và phục vụ của bộ điều khiển thông minh trong các điều kiện thay đổi đột ngột.
    \item \textbf{Kiểm thử adaptive RL}: Đánh giá thời gian chờ, độ dài hàng chờ, tần suất gridlock, phục hồi sau tắc nghẽn.
    \item \textbf{Kịch bản xuất hiện xe ưu tiên}: Đo thời gian phục vụ xe ưu tiên, kiểm tra hiệu quả thiết lập green wave và đánh giá ảnh hưởng tới các hướng giao thông khác.
    \item \textbf{Kịch bản tắc nghẽn cục bộ}: Kích hoạt chế độ congestion mode, đo thời gian giải tỏa, đánh giá khả năng phối hợp corridor giữa các nút giao.
    \item \textbf{Kịch bản rẽ trái bị chặn}: Phát hiện trường hợp rẽ trái bị cản trở, tạo pha rẽ trái bảo vệ động và đo hiệu quả giải tỏa dòng xe.
    \item \textbf{Kịch bản starvation}: Đánh giá khả năng phục hồi phục vụ các hướng giao thông bị bỏ qua hoặc ưu tiên thấp trong quá trình điều phối.
\end{itemize}

\begin{figure}[H]
    \centering
    \fbox{\parbox{0.8\textwidth}{\centering\vspace{3cm}
    \textit{[Placeholder ảnh: Ảnh chụp màn hình set up SUMO với traffic burst, xe ưu tiên, congestion, blocked left]}
    \vspace{3cm}}}
    \caption{Ảnh set up các kịch bản kiểm thử trong mô phỏng SUMO}
    \label{fig:sumo_test_scenarios}
\end{figure}

\subsection{Quy trình thực nghiệm}

\begin{enumerate}
    \item Chạy mô phỏng với cấu hình traffic demand, logic đèn, tham số mặc định.
    \item Theo dõi trạng thái thực tế qua SUMO GUI, dashboard, log sự kiện lên Supabase.
    \item Thu thập dữ liệu về thời gian chờ, độ dài hàng chờ, số lần gridlock, số lần phục vụ xe ưu tiên, hiệu quả protected left.
    \item Phân tích kết quả, so sánh với baseline, đánh giá theo từng kịch bản.
\end{enumerate}

\subsection{Tích hợp dashboard trực quan hóa}

Dữ liệu sự kiện, trạng thái và hiệu năng được trực quan hóa qua dashboard (SmartIntersectionTrafficDisplay), giúp đánh giá kết quả mô phỏng theo thời gian thực.

\begin{figure}[H]
    \centering
    \fbox{\parbox{0.8\textwidth}{\centering\vspace{3cm}
    \textit{[Placeholder ảnh: Ảnh dashboard trực quan hóa trạng thái mô phỏng SUMO, các chỉ số hiệu năng]}
    \vspace{3cm}}}
    \caption{Dashboard trực quan hóa kết quả thí nghiệm mô phỏng SUMO}
    \label{fig:sumo_dashboard}
\end{figure}


% =========================
% CHƯƠNG 12
% =========================
% =========================
% CHƯƠNG 12
% =========================
\chapter{Đánh giá hiệu năng và so sánh}
\section{Các chỉ số đánh giá}
\section{So sánh định lượng và định tính}
\section{Đánh giá theo kịch bản}
\section{Phân tích đường cong học tập}

% =========================
% CHƯƠNG 13
% =========================
% =========================
% CHƯƠNG 13
% =========================
\chapter{Kết quả và thảo luận}
\section{Phát hiện chính}
\section{Phân tích hành vi hệ thống}
\section{Hạn chế và thách thức}

%\chapter{Kết quả và thảo luận}
%\section{Phát hiện chính}
%\section{Phân tích hành vi hệ thống}
%\section{Hạn chế và thách thức}

% =========================
% CHƯƠNG 14
% =========================
% =========================
% CHƯƠNG 14
% =========================
\chapter{Trực quan hóa và giám sát}

\section{Bảng điều khiển thời gian thực}

Trực quan hóa và giám sát là thành phần quan trọng giúp theo dõi hiệu quả vận hành, hỗ trợ phân tích và tối ưu hệ thống. Bảng điều khiển (dashboard) cho phép người dùng quan sát trạng thái giao thông, pha đèn tín hiệu, hàng chờ, tốc độ, thời gian chờ, cùng các sự kiện đặc biệt như ưu tiên xe khẩn cấp hoặc tắc nghẽn.

\subsection{Kiến trúc và chức năng chính}
Dashboard được phát triển bằng Python (tkinter + matplotlib), tích hợp trực tiếp với bộ điều khiển giao thông. Dữ liệu được cập nhật liên tục, hiển thị các thông số quan trọng như trạng thái pha, hàng chờ, thời gian chờ, tốc độ trung bình trên từng làn.

\begin{figure}[H]
    \centering
    \fbox{%
        \includegraphics[width=0.95\textwidth]{image1.png}
    }
    \caption{Giao diện bảng điều khiển SmartIntersectionTrafficDisplay}
    \label{fig:dashboard_gui}
\end{figure}

\subsection{Các nhóm chức năng}
\begin{itemize}
    \item \textbf{Theo dõi trạng thái pha}: Hiển thị thời gian còn lại, duration cơ bản, extended time, trạng thái hiện tại của từng pha.
    \item \textbf{Giám sát hàng chờ và tốc độ}: Biểu đồ hàng chờ, thời gian chờ, tốc độ trung bình từng làn.
    \item \textbf{Trực quan hóa lịch sử điều chỉnh}: Thống kê các lần điều chỉnh duration, extended time, so sánh với baseline.
    \item \textbf{Hiển thị sự kiện đặc biệt}: Cảnh báo các sự kiện như ưu tiên xe khẩn cấp, rẽ trái bảo vệ, tắc nghẽn.
\end{itemize}

\subsection{Luồng dữ liệu}
Dashboard truy xuất dữ liệu từ controller qua API, cập nhật dữ liệu mỗi giây. Các chỉ số hiệu năng, sự kiện và trạng thái hệ thống được lưu lại phục vụ phân tích hậu kỳ.

\section{Phân tích dữ liệu hậu kỳ}

Dữ liệu sự kiện và trạng thái từ dashboard được lưu trữ để phục vụ phân tích offline. Các chức năng chính gồm:
\begin{itemize}
    \item Phân tích pattern điều chỉnh pha và extended time.
    \item Thống kê thời gian chờ trung bình, độ dài hàng chờ, số lần congestion/starvation.
    \item So sánh hiệu năng giữa các nút giao hoặc nhóm cluster.
\end{itemize}

\begin{lstlisting}[style=py,caption={Ví dụ: Thống kê thời gian chờ trung bình}]
import pandas as pd
def compute_average_wait_time(phase_events):
    df = pd.DataFrame(phase_events)
    avg_wait = df.groupby("phase_idx")["duration"].mean()
    print("Thoi gian cho trung binh theo pha:")
    print(avg_wait)
\end{lstlisting}

\section{Đồng bộ và mở rộng}

\begin{itemize}
    \item Dữ liệu dashboard được đồng bộ lên Supabase để phục vụ giám sát từ xa và phân tích tổng hợp.
    \item Có thể mở rộng dashboard với các công cụ hiện đại (plotly, dash), hỗ trợ phân tích động, replay video, kết nối cảm biến thực tế.
\end{itemize}

%\chapter{Trực quan hóa và giám sát}
%\section{Bảng điều khiển thời gian thực}
%\section{Công cụ phân tích}

% =========================
% CHƯƠNG 15
% =========================
% =========================
% CHƯƠNG 15
% =========================
\chapter{Hướng phát triển tương lai}
\section{Điều phối đa nút giao}
\section{Tích hợp học tăng cường sâu}
\section{Giao tiếp xe--hạ tầng}
\section{Dự báo giao thông}
\section{Mở rộng dựa trên điện toán đám mây}
%\chapter{Hướng phát triển tương lai}
%section{Điều phối đa nút giao}
%\section{Tích hợp học tăng cường sâu}
%\section{Giao tiếp xe--hạ tầng}
%\section{Dự báo giao thông}
%\section{Mở rộng dựa trên điện toán đám mây}

% =========================
% CHƯƠNG 16
% =========================
% =========================
% CHƯƠNG 16
% =========================
\chapter{Kết luận}

\section{Tóm tắt kết quả nghiên cứu}

Luận văn đã phát triển và triển khai thành công hệ thống điều khiển đèn giao thông thông minh lai APC–RL trên môi trường mô phỏng SUMO, tích hợp với Supabase để lưu trữ và phân tích dữ liệu. Hệ thống kết hợp:

\begin{itemize}
    \item Bộ điều khiển pha thích nghi (APC) quản lý logic đèn tín hiệu, chuyển pha an toàn, tự động chèn pha vàng, kiểm soát rẽ trái bảo vệ, ưu tiên khẩn cấp, và xử lý tắc nghẽn.
    \item Agent Q-learning cải tiến với vector trạng thái đa chiều, hàm thưởng động, cơ chế optimistic exploration, mask hành động từ coordinator, và khả năng học thích nghi dựa trên dữ liệu thực.
    \item Cơ chế quản lý hàng đợi yêu cầu ưu tiên (pending requests) giúp hệ thống phục vụ hiệu quả xe khẩn cấp, giải quyết starvation, congestion, và điều phối các tình huống đặc biệt.
    \item Module coordinator hỗ trợ điều phối đa nút, phát hiện cụm tắc nghẽn và kích hoạt green wave trên tuyến chính.
    \item Hệ thống lưu trữ Supabase đồng bộ hóa trạng thái, log sự kiện, bảng Q, và hỗ trợ phân tích hiệu năng chi tiết.
\end{itemize}

Các thử nghiệm trên nhiều kịch bản trong SUMO cho thấy mô hình lai APC–RL vượt trội phương pháp điều khiển cố định về giảm thời gian chờ trung bình, tăng thông lượng, và cải thiện khả năng phục vụ xe ưu tiên. Hệ thống có khả năng tự động thích ứng với sự biến động giao thông, xử lý tốt các tình huống phức tạp như tắc nghẽn, gridlock, và rẽ trái bị chặn.

\begin{figure}[H]
    \centering
    \fbox{\parbox{0.8\textwidth}{\centering\vspace{2.2cm}
    \textit{[Diagram Placeholder: Sơ đồ tổng kết pipeline điều khiển – APC, RL, Coordinator, Supabase]}
    \vspace{2.2cm}}}
    \caption{Tổng quan pipeline điều khiển lai APC–RL thực nghiệm}
\end{figure}

\section{Tác động đến quản lý giao thông đô thị}

Kết quả đạt được khẳng định giá trị thực tiễn của mô hình điều khiển lai:

\begin{itemize}
    \item \textbf{Giảm tắc nghẽn}: Tăng tốc giải tỏa hàng chờ, hạn chế spillback và gridlock nhờ phát hiện và xử lý đa mẫu congestion, adaptive extension, và phối hợp corridor.
    \item \textbf{Phục vụ công bằng}: Giảm nguy cơ starvation cho các hướng ít ưu tiên thông qua cơ chế pending requests, scoring động và logic fairness.
    \item \textbf{Ứng phó khẩn cấp}: Đảm bảo phát hiện và phục vụ xe ưu tiên đúng thời điểm, tạo green wave xuyên suốt, rút ngắn thời gian tiếp cận hiện trường.
    \item \textbf{Tăng hiệu quả học máy}: RL agent học liên tục từ phần thưởng thực, tự động điều chỉnh trọng số reward đa mục tiêu, cải thiện tốc độ hội tụ và khả năng thích nghi.
    \item \textbf{Khả năng mở rộng}: Kiến trúc module, cache hai tầng và batch database đảm bảo hiệu suất khi mở rộng lên nhiều nút giao, hỗ trợ đồng bộ và phân tích ở quy mô lớn.
\end{itemize}

\begin{figure}[H]
    \centering
    \fbox{\parbox{0.75\textwidth}{\centering\vspace{2cm}
    \textit{[Diagram Placeholder: Biểu đồ so sánh hiệu năng các chiến lược qua nhiều chỉ số]}
    \vspace{2cm}}}
    \caption{So sánh hiệu năng kiểm soát qua các chỉ số – thời gian chờ, thông lượng, gridlock}
\end{figure}

\section{Ứng dụng thực tiễn và triển vọng}

Hệ thống có tiềm năng ứng dụng rộng rãi trong các dự án giao thông đô thị thông minh:

\begin{itemize}
    \item Tích hợp thực tế với cảm biến IoT, camera AI, V2X để phát hiện phương tiện ưu tiên chính xác.
    \item Triển khai trên các nút giao lớn, khu vực bệnh viện, tuyến đường trục để tăng hiệu quả phục vụ cứu hộ, cứu nạn.
    \item Hỗ trợ dashboard real-time, API phân tích, cho phép giám sát và tối ưu hóa chính sách điều khiển.
    \item Mở rộng cho multi-agent RL, deep RL, dự báo lưu lượng, và điều phối toàn mạng lưới với cloud database.
\end{itemize}

\begin{lstlisting}[style=py,caption={Ví dụ đoạn mã khởi tạo APC và RL agent trên một nút giao}]
lane_ids = traci.trafficlight.getControlledLanes("E3")
apc = AdaptivePhaseController(
    lane_ids=lane_ids, tls_id="E3",
    alpha=1.0, min_green=30, max_green=80
)
rl_agent = EnhancedQLearningAgent(
    state_size=12, action_size=len(traci.trafficlight.getAllProgramLogics("E3")[0].phases),
    adaptive_controller=apc, mode="train"
)
apc.rl_agent = rl_agent
\end{lstlisting}

\section{Khuyến nghị cuối cùng}

\begin{itemize}
    \item \textbf{Nghiên cứu mở rộng}: Thử nghiệm với multi-agent coordination, deep RL (DQN, PPO), tích hợp kịch bản thực tế từ dữ liệu cảm biến.
    \item \textbf{Tối ưu fairness}: Điều chỉnh động trọng số reward, logic fairness cho các hướng có nguy cơ starvation, nhất là khi xuất hiện nhiều sự kiện ưu tiên liên tiếp.
    \item \textbf{Kiểm thử stress}: Mở rộng kịch bản stress test với gridlock cực đoan, nhiều xe ưu tiên đồng thời và lỗi hệ thống.
    \item \textbf{Tích hợp thực tế}: Kết nối với hệ thống camera, V2X, cloud API, kiểm chứng hiệu quả mô hình trên mạng lưới thực tế.
    \item \textbf{Minh bạch và tái tạo}: Công khai mã nguồn, dữ liệu thử nghiệm, cấu hình hyperparameters để cộng đồng dễ dàng tái tạo và so sánh kết quả.
\end{itemize}

\vspace{0.5cm}

\noindent\textbf{Kết luận:} Mô hình điều khiển lai APC–RL kết hợp ưu điểm của điều khiển dựa trên luật và học máy, nâng cao hiệu quả, độ linh hoạt và khả năng phục hồi cho hệ thống đèn giao thông đô thị thông minh. Đây là nền tảng vững chắc cho các nghiên cứu và ứng dụng mở rộng trong quản lý giao thông thông minh thời gian thực.
%\chapter{Kết luận}
%\section{Tóm tắt kết quả}
%\section{Tác động đến quản lý giao thông}
%\section{Ứng dụng thực tiễn}
%\section{Khuyến nghị cuối cùng}

% ===== Bibliography =====
\cleardoublepage
\phantomsection
\addcontentsline{toc}{chapter}{Tài liệu tham khảo}
\printbibliography[title={Tài liệu tham khảo}]

% ===== Appendices =====
\appendix

\chapter{Tài liệu mã nguồn}
% Đảm bảo đã thêm \usepackage{listings} và cấu hình style trong phần preamble
% Ví dụ chèn file mã nguồn chính:
% \lstinputlisting[style=py,caption={Bộ điều khiển Lane7b.py}]{Lane7b.py}

\appendix
\chapter{Tài liệu mã nguồn}
% Đảm bảo đã thêm \usepackage{listings} và cấu hình style trong phần preamble
% Ví dụ chèn file mã nguồn chính:
% \lstinputlisting[style=py,caption={Bộ điều khiển Lane7b.py}]{Lane7b.py}

% Link/ mô tả dữ liệu, cấu hình traffic demand...

%\chapter{Thuật ngữ}
% Giải thích các thuật ngữ chuyên ngành bổ sung.

% ======= Bảng phụ lục chi tiết schema =======

% File: appendix_phase_records2.tex
% Sử dụng file này như một chương phụ lục riêng sau \appendix trong main.tex

\chapter{Phụ lục: Lược đồ bảng \texttt{phase\_records}}

\section*{Lược đồ đầy đủ bảng \texttt{phase\_records}}
\addcontentsline{toc}{chapter}{Phụ lục: Lược đồ bảng phase\_records}
\label{app:phase_records_full}

Bảng \texttt{phase\_records} lưu lịch sử chi tiết mọi lần điều chỉnh pha trong hệ thống, bao gồm thông tin thời gian, điều khiển và chỉ số hiệu suất. Bảng này được sử dụng cho phân tích hậu kỳ, huấn luyện RL, kiểm toán và debug.

\vspace{0.5em}
\begin{table}[H]
\centering
\footnotesize
\begin{tabular}{@{}llp{7.2cm}@{}}
\toprule
\textbf{Cột} & \textbf{Kiểu dữ liệu} & \textbf{Mô tả} \\
\midrule
\multicolumn{3}{@{}l}{\textit{Thông tin cơ bản}}\\
\midrule
id            & BIGSERIAL      & Khóa chính tự tăng \\
tls\_id       & TEXT           & ID của nút giao thông \\
phase\_idx    & INTEGER        & Chỉ số pha (0--11) \\
sim\_time     & REAL           & Thời điểm trong mô phỏng (giây) \\
\midrule
\multicolumn{3}{@{}l}{\textit{Thông số thời gian}}\\
\midrule
duration      & REAL           & Thời lượng thực tế của pha (giây) \\
base\_duration& REAL           & Thời lượng cơ sở theo cấu hình \\
delta\_t      & REAL           & Mức điều chỉnh sau làm mượt \\
raw\_delta\_t & REAL           & Mức điều chỉnh ban đầu (trước smoothing) \\
extended\_time& REAL           & Thời gian mở rộng thêm so với base \\
\midrule
\multicolumn{3}{@{}l}{\textit{Thông tin điều khiển}}\\
\midrule
state\_str    & TEXT           & Chuỗi trạng thái (ví dụ ``GGrrGGrr'') \\
event\_type   & TEXT           & Loại sự kiện kích hoạt điều chỉnh (emergency, starvation, rl, ...) \\
weights       & JSONB          & Trọng số các yếu tố điều khiển (density, speed, wait, queue) \\
lanes         & JSONB          & Danh sách làn được phục vụ (dùng cho phân tích theo làn) \\
\midrule
\multicolumn{3}{@{}l}{\textit{Đánh giá hiệu suất}}\\
\midrule
reward        & REAL           & Phần thưởng tính cho hành động (RL) \\
penalty       & REAL           & Hình phạt cho điều chỉnh quá mức \\
bonus         & REAL           & Điểm thưởng/bonus (nếu có) \\
\midrule
\multicolumn{3}{@{}l}{\textit{Metadata \& auditing}}\\
\midrule
created\_at   & TIMESTAMPTZ    & Thời điểm tạo bản ghi \\
updated\_at   & TIMESTAMPTZ    & Thời điểm cập nhật cuối \\
\bottomrule
\end{tabular}
\caption{Lược đồ đầy đủ của bảng \texttt{phase\_records}}
\label{tab:phase_records_detailed_full}
\end{table}

\section*{Ghi chú triển khai}
\begin{itemize}
  \item \textbf{Trường JSONB (weights, lanes):} Dùng JSONB để hỗ trợ indexing (GIN) và mở rộng linh hoạt khi cần thêm trường không cố định.
  \item \textbf{Chỉ mục gợi ý:} \texttt{CREATE INDEX idx\_phase\_records\_tls\_phase ON phase\_records(tls\_id, phase\_idx);} \\
    \texttt{CREATE INDEX idx\_phase\_records\_simtime ON phase\_records(sim\_time);}
  \item \textbf{Lưu trữ/archival:} Khi triển khai thực tế, cân nhắc chính sách lưu trữ (giữ dữ liệu tần suất cao trong 3 tháng, sau đó nén/archive).
\end{itemize}

\end{document}